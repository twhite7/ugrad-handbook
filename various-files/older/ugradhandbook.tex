\documentclass[12pt,twoside]{article}

\usepackage{epsfig}
%\usepackage{url}
\usepackage{fancyheadings}
\usepackage{setspace}
\usepackage[pdftex]{hyperref}
\usepackage{ifthen}

\newenvironment{itemlist}{
\begin{itemize}
\setlength{\itemsep}{0pt}
\setlength{\parskip}{0pt}}
{\end{itemize}}

\setlength{\topmargin}{-.5in}
\setlength{\oddsidemargin}{-0.25in}
\setlength{\evensidemargin}{-0.25in}
\setlength{\textwidth}{7truein}
\setlength{\textheight}{9truein}
\setlength{\parskip}{6pt}

\newboolean{printmode}
\setboolean{printmode}{false}

\raggedbottom

\ifthenelse{\boolean{printmode}}{
\newcommand{\textsize}{\Large}
}{
\newcommand{\textsize}{\normalsize}
}

\begin{document}
\pagestyle{empty}
\vspace*{1in}

\begin{centering}
\Huge

\ifthenelse{\boolean{printmode}}{
\epsfig{figure=Eng-side-Grayscale-pc.png,height=1.25in}
}{
\epsfig{figure=Eng-side-cmyk-uncoated-pc.png,height=1.25in}
}

\vspace*{1in}

Department {\em of}\/ Computer Science

\bigskip

Undergraduate Handbook

\vspace{1in}

\LARGE
Bachelor of Science in Computer Science

Minor in Computer Science

\end{centering}

\textsize

\newpage

\vspace*{3in}

\begin{centering}

This undergraduate handbook was\linebreak
last updated on \today. \linebreak
\linebreak\linebreak\linebreak
Any version of this handbook dated\linebreak
during or after December 2007 is\linebreak
valid for the spring 2008 semester.\linebreak

\end{centering}

\newpage

\pagestyle{fancy}

%\section{Table of Contents}

\pagenumbering{roman}
\tableofcontents

\newpage

\pagenumbering{arabic}

%----------------------------------------------------------------------

%\doublespacing
%\setstretch{2.2}

\section{Introduction}

Through the development of sophisticated computer systems, processors,
and embedded applications, computer scientists have the opportunity to
change society in ways unimagined several years ago. A major
departmental goal is the education and training of a diverse body of
students who can lead this current information technology
revolution. To this end, the computer science program orients students
toward the pragmatic aspects of computer science and provides the
learning and practices to make them proficient computing
professionals. Good engineering is rooted in solid mathematics and
science, and grounding in these fundamentals is essential. Provided in
the context of the practice of computing, this early grounding forms
the basis for an education that prepares students for a computing
career.

With funding from the National Science Foundation, the Department of
Computer Science has designed and developed a curriculum focused on
the practice of computing, yet grounded in the mathematical and
scientific fundamentals of computer science. The curriculum is
structured around the introduction of modern software development
techniques in the very beginning courses, and is supported by a
set of ``closed laboratories''.

In order to provide an environment appropriate to our courses, the
department has established several laboratories with hundreds of
workstations. These machines have high-resolution graphics and are
connected to large file handlers, as well as to the University
network. The lab courses expose students to many commercial software
tools and systems, and introduce modern software development
techniques via object-oriented design and implementation.

The Department of Computer Science co-offers, with the Department of
Electrical and Computer Engineering, a degree in computer engineering.

Students have ample opportunities to participate in cutting-edge
research with department faculty members.  From the senior thesis
research project to independent study, one can pursue research in any
conceivable area.  Our former students are enrolled in all of the top
graduate programs in the country.  Our undergraduates have won many
research awards, including winning five recent CRA research award
winners.
% we had 5 in 2007

Graduates of the computer science program at the University of
Virginia will have the knowledge, skills and attitudes that will allow
them to make tangible contributions, meet new technical challenges,
contribute effectively to society, act as team members, and be
innovators in the design, analysis and application of computer
systems.

%\section{Introduction}

%Our computer science and computer engineering degrees ready students
%for careers that provide both personal and societal rewards. As
%creators of information technologies, our graduates are reaching out
%to people and the world by supporting and enhancing communication,
%health care, entertainment, scientific inquiry, transportation,
%business, and almost any other endeavor you can imagine.

%The hallmarks of our undergraduate program of studies include a high
%degree of mathematical rigor reinforced through use, an emphasis on a
%philosophy of engineering, hands-on experience with team-oriented
%laboratories, undergraduate involvement in research projects, and a
%carefully crafted progression of material as the students advance
%through the program.

%----------------------------------------------------------------------

\section{Degrees Offered}

The Department of Computer Science offers three computing degrees, as
well as a minor option.

\begin{itemlist}
\item Bachelor of Science in Computer Science
\item Bachelor of Science in Computer Engineering
\item Bachelor of Arts in Computer Science
\item Minor in Computer Science
\end{itemlist}

The computer engineering degree, handled jointly with the Department
of Electrical and Computer Engineering, focuses more on hardware-level
issues, while still giving the students experience in both software
and electrical engineering.  The computer science degree focuses more
on software, while still giving students experience in computer
hardware.  The undergraduate handbook for computer engineering can be
found at
\url{http://www.cpe.virginia.edu/ugradmainpage.html}.
Students wishing to dual major in computer science and computer
engineering should see
\url{http://www.cpe.virginia.edu/compeng_and_cs_%20combined.pdf}.

The Bachelor of Arts in Computer Science is for students in the
College of Arts and Sciences.  The requirements for the Bachelor of
Arts are different than those for the Bachelor of Science.  The
Bachelor of Arts requirements can be found at
\url{http://www.cs.virginia.edu/ba/}.

This document deals primarily with the Bachelor of Science in computer
science.  However, section \ref{minorreqs} describes the requirements
for the minor.


\subsection{ABET accreditation}

The Bachelor of Science in Computer Science degree is accredited by
the Computing Accreditation Commission of ABET, 111 Market Place,
Suite 1050, Baltimore, MD 21202-4012, telephone: (410) 347-7700. 
%The Bachelor of Science in Computer Engineering is accredited by the
%Engineering Accreditation Commission of ABET.

%----------------------------------------------------------------------
\ifthenelse{\boolean{printmode}}{}{\newpage}
\textsize

\section{Recommended Course of Study}
\label{sampleschedule}

%Below is the recommended course of study for the Bachelor of Science
%in Computer Science.  Note that if you have already completed some of
%these classes (through AP credit, for example), then your course of
%study would obviously deviate from what is shown below -- consult your
%academic advisor for details.

\ifthenelse{\boolean{printmode}}{On the next page}{Below} is the
recommended course of study for the bachelor's degree.  If you have
already completed some of these classes (through AP credit, for
example), then your course of study would deviate from what is shown
below -- consult your academic advisor for details.

There are a total of 9 electives that the student can choose from.
These electives are indicated by the footnotes below, and are
described in detail beginning on page \pageref{electiveinformation}.
Note that some of these requirements are for all SEAS students, while
others are required for the CS bachelor's degree.

Please be aware of when the classes are offered!  Some are only
offered once per year, or in a particular semester.  See page
\pageref{coursedescriptions} for details as to when courses are
offered.

\bigskip


\ifthenelse{\boolean{printmode}}{
\noindent\begin{tabular}{p{1.25in}p{4in}c}
& \bf First semester				& \bf 15 \\
APMA 111 & Single Variable Calculus 		& 4 \\
CHEM 151 & Chemistry for Engineers 		& 3 \\
CHEM 151L & Chemistry Lab 			& 1 \\
ENGR162 & Problem Solving \& Design 		& 4 \\
STS 101 & Engineering, Technology and Society	& 3 \\
\\
& \bf Second semester & \bf 17 \\
APMA 212 & Multivariate Calculus & 4 \\
PHYS 142E & Physics I & 3 \\
PHYS 142W & Physics I Workshop & 1 \\
CS 101 & Intro to Computer Science & 3 \\
??? & Science elective$^1$ & 3 \\
??? & HSS elective$^2$ & 3 \\
\\
& \bf Third semester				& \bf 16 \\
APMA & APMA elective$^5$ or APMA 310 		& 3 \\
CS 201 & Software Development Methods		& 3 \\
CS 202 & Discrete Mathematics 			& 3 \\
PHYS 241E & General Physics II 			& 1 \\
PHYS 241W & General Physics Lab I 		& 3 \\
??? & HSS elective$^2$ 				\\
\\
& \bf Fourth semester & \bf 16 \\
CS 216 & Program and Data Representation & 3 \\
CS 230 & Digital Logic Design & 3 \\
CS 302 & Theory of Computation & 3 \\
CS 290 & CS Seminar & 1 \\
STS & STS 2xx/3xx elective$^9$ & 3 \\
??? & Technical elective$^3$ & 3 \\
\end{tabular}

\noindent\begin{tabular}{p{1.25in}p{4in}c}
& \bf Fifth semester				& \bf 15 \\
CS 333 & Computer Architecture 			& 3 \\
CS 432 & Algorithms 				& 3 \\
APMA & APMA elective$^5$ or APMA 310		& 3 \\
??? & Technical elective$^3$ 			& 3 \\
??? & Unrestricted elective$^4$ 		& 3 \\
\\
& \bf Sixth semester & \bf 15 \\
CS 340 & Adv.\ Software Develop. & 3 \\
CS & CS elective$^8$ & 3 \\
APMA & APMA elective$^5$ or APMA 310 & 3 \\
??? & Gen Edu elective$^6$ & 3 \\
??? & HSS elective$^2$ & 3 \\
\\
& \bf Seventh semester				& \bf 15 \\
STS 401 & Western Tech and Culture 		& 3 \\
??? & Computer Arch. elective$^7$ 		& 3 \\
CS 414 & Operating Systems 			& 3 \\
CS & CS elective$^8$ 				& 3 \\
??? & Gen Edu elective$^6$ 			& 3 \\
\\
& \bf Eighth semester & \bf 15 \\
STS 402 & Engineer in Society & 3 \\
CS & CS elective$^8$ & 3 \\
CS & CS elective$^8$ & 3 \\
??? & Technical elective$^3$ & 3 \\
??? & Gen Edu elective$^6$ & 3 \\
\end{tabular}
}{
\normalsize
\noindent \begin{tabular}{llccllc}
& \bf First semester				& \bf 15 & & & \bf Second semester & \bf 17 \\
APMA 111 & Single Variable Calculus 		& 4 & & APMA 212 & Multivariate Calculus & 4 \\
CHEM 151 & Chemistry for Engineers 		& 3 & & PHYS 142E & Physics I & 3 \\
CHEM 151L & Chemistry Lab 			& 1 & & PHYS 142W & Physics I Workshop & 1 \\
ENGR162 & Problem Solving \& Design 		& 4 & & CS 101 & Intro to Computer Science & 3 \\
STS 101 & Engineering, Technology 		& 3 & & ??? & Science elective$^1$ & 3 \\
& and Society 	       				& & & ??? & HSS elective$^2$ & 3 \\
\\
& \bf Third semester				& \bf 16 & & & \bf Fourth semester & \bf 16 \\
APMA & APMA elective$^5$ or	 		& 3 & & CS 216 & Program and Data & 3 \\
& APMA 310 					& & & & Representation & \\
CS 201 & Software Development 			& 3 & & CS 230 & Digital Logic Design & 3 \\
& Methods 					& 3 & & CS 302 & Theory of Computation & 3 \\
CS 202 & Discrete Mathematics 			& 3 & & CS 290 & CS Seminar & 1 \\
PHYS 241E & General Physics II 			& 1 & & STS & STS 2xx/3xx elective$^9$ & 3 \\
PHYS 241W & General Physics Lab I 		& 3 & & ??? & Technical elective$^3$ & 3 \\
??? & HSS elective$^2$ 				\\
\\
& \bf Fifth semester				& \bf 15 & & & \bf Sixth semester & \bf 15 \\
CS 333 & Computer Architecture 			& 3 & & CS 340 & Adv.\ Software Develop. & 3 \\
CS 432 & Algorithms 				& 3 & & CS & CS elective$^8$ & 3 \\
APMA & APMA elective$^5$ or 			& 3 & & APMA & APMA elective$^5$ or & 3 \\
& APMA 310  		 			& & & & APMA 310 & \\
??? & Technical elective$^3$ 			& 3 & & ??? & Gen Edu elective$^6$ & 3 \\
??? & Unrestricted elective$^4$ 		& 3 & & ??? & HSS elective$^2$ & 3 \\
\\
& \bf Seventh semester				& \bf 15 & & & \bf Eighth semester & \bf 15 \\
STS 401 & Western Tech and Culture 		& 3 & & STS 402 & Engineer in Society & 3 \\
??? & Computer Arch. elective$^7$ 		& 3 & & CS & CS elective$^8$ & 3 \\
CS 414 & Operating Systems 			& 3 & & CS & CS elective$^8$ & 3 \\
CS & CS elective$^8$ 				& 3 & & ??? & Technical elective$^3$ & 3 \\
??? & Gen Edu elective$^6$ 			& 3 & & ??? & Gen Edu elective$^6$ & 3
\end{tabular}
}

\textsize

%----------------------------------------------------------------------
\section{Degree Requirement Checklist}
\normalsize
\begin{tabular}{@{}l} % outer 'holding' box

\begin{tabular}{|l|l|l|l|}
\hline
\bf Required computing \& math courses & \bf Grade & \bf Semester & \bf Comments \\ \hline \hline
CS 101 Introduction to Computer Science & & & \hspace{1.8in} \\ \hline
CS 201 Software Development Methods & & & \\ \hline
CS 202 Discrete Mathematics & & & \\ \hline
CS 216 Program \& Data Representation & & & \\ \hline
CS 230 Digital Logic Design & & & \\ \hline
CS 290 CS Seminar & & & \\ \hline
CS 302 Theory of Computation & & & \\ \hline
CS 333 Computer Architecture & & & \\ \hline
CS 340 Advanced SW Development Techniques & & & \\ \hline
CS 414 Operating Systems & & & \\ \hline
CS 432 Analysis of Algorithms & & & \\ \hline
Computer Architecture Elective (from list)  & & & Course: \\ \hline
APMA 310 Probability & & & \\ \hline
APMA 213 or 308 or 312 (circle one) & & & \hspace{1.5in} \\ \hline
APMA 213 or 308 or 312 (circle one) & & & \hspace{1.5in} \\ \hline
\end{tabular}

\bigskip

\\

\noindent\begin{tabular}{@{}ll}
\noindent\begin{tabular}{@{}l}
\noindent SEAS required courses
\\
\begin{tabular}{|l|l|l|} \hline
\bf Course & \bf Grade & \bf Semester \\ \hline \hline
APMA 111 & & \\ \hline
APMA 212 & & \\ \hline
CHEM 151 & & \\ \hline
CHEM 151L & & \\ \hline
ENGR 162 & & \\ \hline
PHYS 142E & & \\ \hline
PHYS 142W & & \\ \hline
PHYS 241E & & \\ \hline
PHYS 241W & & \\ \hline
\end{tabular} \\
\\
% cs electives
\noindent CS electives \\
\noindent\begin{tabular}{|l|l|l|l|} \hline
& \bf Course & \bf Grade & \bf Semester \\ \hline \hline
1) & \hspace{1in} & & \\ \hline
2) & & & \\ \hline
3) & & & \\ \hline
4) & & & \\ \hline
\end{tabular} \\
% tech electives
\\
\noindent Technical electives\\
\noindent\begin{tabular}{|l|l|l|l|l|} \hline
& \bf Course & \bf Grade & \bf Semester \\ \hline \hline
1) & \hspace{1in} & & \\ \hline
2) & & & \\ \hline
3) & & & \\ \hline
\end{tabular}\\

\end{tabular}

&

\begin{tabular}{l}
\noindent STS courses
\\
\begin{tabular}{|l|l|l|} \hline
\bf Course & \bf Grade & \bf Semester \\ \hline \hline
STS 101 & & \\ \hline
STS 2xx/3xx & & \\ \hline
STS 401 & & \\ \hline
STS 402 & & \\ \hline
\end{tabular} \\
\\
\noindent SEAS electives \\
\noindent\begin{tabular}{|l|l|l|l|} \hline
\bf Course & \bf Grade & \bf Semester & \bf Course \\ \hline \hline
Science elective & & & \\ \hline
HSS elective \# 1 & & & \\ \hline
HSS elective \# 2 & & & \\ \hline
HSS elective \# 3 & & & \\ \hline
Unrest.\ elective & & & \\ \hline
\end{tabular} \\

\\
\noindent General education electives (9 credits) \\
\noindent\begin{tabular}{|l|l|l|l|} \hline
& \bf Course & \bf Grade & \bf Semester \\ \hline \hline
1) & \hspace{1in} & & \\ \hline
2) & & & \\ \hline
3) & & & \\ \hline
4) & & & \\ \hline
\end{tabular}

\end{tabular}

\end{tabular}

\end{tabular} % outer 'holding' box

%----------------------------------------------------------------------

\newpage
\section{Course Requirement Flowchart}

\epsfig{figure=Cs-course-flowchart-073.png,width=\textwidth}


%----------------------------------------------------------------------
\section{Elective Information}
\label{electiveinformation}
\textsize

The numbers in the list below correspond to the footnote numbers from
the sample course schedule shown on page \pageref{sampleschedule}.

\begin{enumerate}

\item Science elective (1 required): Students must choose one of BIOL
201 (Introduction to Biology: Cell Biology and Genetics), BIOL 202
(Introduction to Biology: Organismal and Evolutionary Biology), CHEM
152 (Introductory Chemistry for Engineers), CHEM 162 (Introductory
Chemistry for Engineers), ECE 200 (Science of Information), MSE 209
(Introduction to the Science and Engineering of Materials), or PHYS
252 (Introductory Physics IV: Quantum Physics).  Additional courses in
this list can count as technical or unrestricted electives.

\item HSS electives (3 required): Studies in the humanities and social
sciences serve not only to meet the objectives of a broad education,
but also to meet the objectives of the engineering profession.  Such
course work must meet the generally accepted definitions that the
humanities are the branches of knowledge concerned with humankind and
its culture, while the social sciences are the studies of society.
See the full list of allowed courses in the SEAS Undergraduate
Handbook.  This list can be found online at
\url{http://www.seas.virginia.edu/advising/undergradhandbook.php#hss}.

\item Technical electives (3 required): Technical electives are
courses whose major emphasis is mathematics, science, or
engineering.  Technical electives can be at the 200-level, but at
least two courses (6 of the 9 credits) must be at the 300-level or
higher.  See page \pageref{techelectives} for more details,
including courses that do not count as technical electives.  Courses
where there is uncertainty as to whether or not it qualifies
should be approved by the student's advisor and recorded with a
signature.

\item Unrestricted elective (1 required): Any graded course in the
University, with a few exceptions.  From the SEAS Undergraduate
Student Handbook (found at
\url{http://www.seas.virginia.edu/advising/undergradhandbook.php}):

{\em All Unrestricted Electives may be chosen from any graded course
in the University except mathematics courses below MATH 131, including
STAT 110 and 112, and courses that substantially duplicate any others
offered for the degree, including PHYS 201, PHYS 202, CS 110, CS 111,
CS 120, or any introductory programming course. Students in doubt as
to what is acceptable to satisfy a degree requirement should obtain
the approval of their advisor and the dean's office, Thornton Hall,
Room A122. APMA 109 counts as a three credit unrestricted elective for
students.}

Band classes (such as marching band) and ROTC classes can count for
the unrestricted elective.

\item APMA elective (2 required): Must choose two from: APMA 213
(Ordinary Differential Equations), APMA 308 (Linear Algebra) or APMA
312 (Statistics).  Note that APMA 310 (Probability) is a required
course in addition to the two APMA electives.

\item General education elective (3 required): General education
courses are designed to further broaden the student beyond the three
required HSS courses.  Allowed courses include all those allowed as an
HSS elective, as well as other non-technical courses. See page
\pageref{geneduelectives} for more information. Courses where there
is uncertainty as to whether or not it qualifies should be
approved by the student's advisor and recorded with a signature.

\item \label{comparchelective} Computer architecture elective (1
required): A student must take one course from the list of approved
computer architecture electives. The current list includes: ECE 435
(Computer Organization and Design), CS 433 (Advanced Computer
Architecture), and CS 444 (Introduction to Parallel Computing).

\item CS electives (4 required): Any 3 credit CS class at the 300
level or above.  A course that is fulfilling another requirement can
not also count as a CS elective.  If you take more than four CS
electives, you can count additional CS elective course(s) as either
a technical elective or an unrestricted elective.  ECE 436 (Advanced
Digital Design) also counts as a CS elective (although not for CpE
majors, as ECE 436 is required for CpE).  There are restrictions on
counting independent study courses for your CS electives -- see your
advisor for details.  Note that a previous set of degree requirements
limited which courses above the 300 level counted as electives -- the
description here is the current requirements (changed October, 2
2006).  Note that for a class that does not meet these requirements to
count as a CS elective requires approval by the department (NOT by the
student's academic advisor).

\item STS 2xx/3xx elective (1 required): Any STS course at the
200-level or 300-level.

\end{enumerate}

Note that classes that receive no grade (such as the no longer offered
TCC 100, or classes that are audited) do not count toward your degree
requirements.

%----------------------------------------------------------------------
%\newpage
\section{Technical Elective Policies}

\label{techelectives}

\subsection{General information}

Technical electives are courses whose major emphasis is mathematics,
science, or engineering.  Three technical elective courses (9 credits)
are required for the bachelor's degree. One of the technical electives
can be at the 200-level, but at least two courses (6 of the 9 credits)
must be at the 300-level or higher.  The course should be one that
majors from that field can take, i.e., no ``physics for poets''
classes. Often the course description in the undergraduate record
identifies such courses. CS majors can use CS courses as technical
electives, once the requirements for the CS electives have been
completed.

Any course taught by another department in SEAS may count as a
technical elective as long as the department offering that course
considers it to meet the standards of a technical elective.

Beware of courses with overlap substantial.  For example, a linear
programming course overlaps with SYS 321 (Deterministic Decision
Models).  You can only receive technical elective credit for one
course if you take multiple courses whose subject material overlaps.
If you have any questions, see your academic advisor.  Courses where
there is uncertainty as to whether or not it qualifies should be
approved by the student's advisor and recorded with a signature.

\subsection{Class specific details}

\noindent Examples of what does count as a technical elective:
\begin{itemlist}
\item EVSC 3xx or higher: Environmental Science courses of 3xx or
better are accepted.
\item PHIL 542 (Symbolic Logic) is a technical elective.
\item PSYC 305 (Research Methods and Analysis), PSYC 306 (Research
  Methods and Analysis II) count as technical or general elective,
  {\bf not as a HSS elective}.
\item TMP 351 (The Technology and Product-Development Life Cycle) is a technical elective.
%\item MDST 345: MDST 345 is a technical elective % no longer offered as of 11-30-07
\item MUSI 445 (Computer Applications in Music) is a technical
elective, and does NOT count as a CS elective.
\item PSYC 220 (A Survey of the Neural Basis of Behavior) is a
technical elective.
\item Students can take accounting courses as a technical elective.  

\end{itemlist}

\noindent Examples of what does NOT count as a technical elective:
\begin{itemlist}
\item SYS 323 (Human Machine Interface) does not count as a technical
  elective.  Note that CS 305 has similar content (and CS 305 counts
  as a CS elective).
\item PHIL 242 (Introduction to Symbolic Logic) cannot be taken for
  technical elective credit, as it overlaps CS 202.
\item TMP 352 (Science and Technology Public Policy) is not a
  technical elective. %; it is a STS elective.
\item ECON 201 (Principles of Economics: Microeconomics) and ECON 202
  (Principles of Economics: Macroeconomics) count as HSS  electives (or
  general education or unrestricted), but not as technical electives. 
\item ASTR 348 (Introduction to Cosmology) is not a technical elective
  for CS majors.
%\item COMM 320: COMM 320 cannot count as a technical elective because
%of significant overlap with CS 201. % course no longer exists (as of 11-30-07)
\end{itemlist}

%\noindent What might count as a technical elective:
%\begin{itemize}
%\item ARCH 305: ARCH 305 in isolation is not considered a technical
%elective. A case can be made if the student is taking ARCH 305 as part
%of series of courses in architectural design.  See your advisor for
%details.
%\end{itemize}

\subsection{Overlap with other courses}

You can only receive technical elective credit for one course if you
take multiple courses whose subject material overlaps.  Examples of
courses that overlap include the following.

\begin{itemize}

\item Only one of ENGR 488 (Business and Technical Leadership in
  Engineering) %, MAE 400 () % no longer exists
  and CE 441 (Construction Engineering and Economics) may count as a
  technical elective because of significant overlap.

\item Students who have taken APMA 310 (Probability) cannot use APMA
  311 (Applied Probability and Statistics) as a technical
  elective. APMA 312 (Statistics) may be used as a technical elective;
  however, no student can get credit for both APMA 311 and either of
  APMA 310 or 312.  Note that students usually use APMA 312 to fulfill
  the APMA elective requirement.

\item The definition of MATH 404 (Discrete Mathematics) in the
  Undergraduate Record indicates that there is too much overlap with
  CS 202 to get credit for both. However, at least one instructor is
  doing cryptography rather than discrete mathematics in the
  course. Under those circumstances, you may get credit for both CS
  202 and MATH 404.

\item Students CAN take both SYS 321 (Deterministic Decision Models)
  and CS 457 (Computer Networks).

\item Because of substantial overlap, a student can earn credit for
  only one of SYS 202 (Data and Information Engineering), SYS 204
  (Data and Information Management), and CS 462 (Databases).  Multiple
  classes taken from this set can only be used to fulfill the
  unrestricted elective requirement.

\end{itemize}


%----------------------------------------------------------------------
%\newpage
\section{General Elective Policies}

\label{geneduelectives}

\subsection{General information}

The goal of this requirement is for our majors to take additional
courses in humanities, social sciences, arts, and other disciplines
that serve to broaden the background of the student. In this context,
``broaden'' means courses outside the areas required for the
bachelor's degree (i.e. not science, math, engineering, or
computing). Note that the general education elective courses are in
addition to the nine credits of HSS electives that all SEAS students
must take.  Courses where there is uncertainty as to whether or not it
qualifies should be approved by the student's advisor and
recorded with a signature.  See your advisor if you have any questions
on the general education requirements.

\subsection{Approved general elective courses}

The following describes what courses may or may not count toward this
requirement:

\begin{enumerate}

\item Any course that satisfies the SEAS HSS elective requirement will
count. The description and list of courses for SEAS HSS requirements
are listed online at
\url{http://www.seas.virginia.edu/advising/undergradhandbook.php#hss}.

\item Any course that could satisfy the CS elective or technical
elective requirement cannot count.

\item Other courses may count if they meet the spirit of the
requirement (see the first paragraph of this section) and they are
approved in advance by the department. The department maintains a
partial list of courses that have been approved -- see below.

\item Only one performance-oriented or skills-development courses may
be allowed to count.

\item Band classes (such as marching band) may count for 3 of the 9
required general education elective credits.  Note that band classes
may also count toward the unrestricted elective.

\item Courses on communication in the student's native language,
regardless of their level, may not be used to satisfy this
requirement.  This is the same policy as that used for the SEAS HSS
electives.

\item ROTC classes can count for 6 of the 9 general education elective
credits.  Note that you can also count ROTC classes toward the
unrestricted elective.

\end{enumerate}


The following courses that satisfy the SEAS Minor in Business, and may
also count for this requirement.

\begin{itemlist}
\item COMM 180 (Making Business Work)
\item COMM 341 (Commercial Law I)
\item COMM 351 (Fundamentals of Marketing)
\item COMM 371 (Managerial Finance I)
\item COMM 381 (Business Ethics)
\item COMM 467 (Organizational Change and Development)
\item COMM 468 (Entrepreneurship)
\item TMP 352 (Science and Technology Public Policy)
\item TMP 399 (Case Studies in Technology Management and Policy)
\item ISBU 327 (Investment Analysis)
\item ISBU 361 (Organizational Behavior)
\item ISBU 384 (International Business)
\item ISBU 485 (Strategic Management)
\end{itemlist}

Note that TMP 351 (The Technology and Product-Development Life Cycle),
COMM 201 (Introduction to Financial Accounting) and COMM 202
(Introduction to Management Accounting) satisfy the SEAS Minor in
Business, but these count as technical electives.  ECON 201
(Principles of Economics: Microeconomics) and ECON 202 (Principles of
Economics: Macroeconomics) count for either the HSS requirement or the
general education requirement, and they also count toward the
business minor.


%----------------------------------------------------------------------

%\ifthenelse{\boolean{printmode}}{\newpage}{}

\section{Minor in Computer Science}
\label{minorreqs}

The Department of Computer Science provides a minor program for
qualified students. The courses in the minor program provide a solid
foundation in computer science. The minor program is a six course,
eighteen credit curriculum. The curriculum consists of the four
required courses and two elective courses.  Full course descriptions
are at the end of this document, beginning on page
\pageref{coursedescriptions}.

\subsection{SEAS students}

All SEAS (School of Engineering and Applied Science) students are
required to take (or place out of) CS 101, as part of the SEAS
first-year curriculum.  This course is also the first required course
for the minor.

The following are the first four courses required for the minor.

\begin{itemlist}
\item CS 101: Introduction to Computer Science
\item CS 201: Software Development Methods
\item CS 202: Discrete Mathematics
\item CS 216: Program and Data Representation
\end{itemlist}

Furthermore, two additional computer science electives are
required. The elective courses must be computer science courses at the
300 level or above.  The only restriction on elective courses is a
limit to how many independent study courses one can count toward a
minor -- contact the minor advisor for details at
\url{minoradvisor@cs.virginia.edu}.

Computer science courses typically build upon each other. In
particular, CS 101 is a prerequisite of both CS 201 and CS 202.  CS
201 and CS 202 are both prerequisites of CS 216. In addition, CS 216
is a prerequisite for almost all of the computer science electives.
The Department of Computer Science also requires that its courses be
passed at a certain level (typically a C- or better) in order to take
successive courses. Be aware that the department strictly enforces its
prerequisite policy.

\subsection{Non-SEAS students}

We recommend that non-SEAS (School of Engineering and Applied Science)
students interested in taking a computer science minor start in the
fall semester. SEAS requires that its students take CS 101 in the
spring semester, and thus there are many more students enrolled in CS
101 in the spring semester.

CS 150 and CS 205 are the recommended introductory courses for
non-SEAS students. These courses count for CS 101 and CS 201,
respectively, for the minor requirements.  However, students may
choose to substitute CS 150 with CS 101 or substitute CS 205 with CS
201, but will not get credit for taking both CS 101 and CS 150 or both
CS 201 and CS 205.

The rest of the minor requirements are the same: CS 202, CS 216, and
two elective courses, as described in the previous section.

\subsection{Declaring the minor}

To declare a minor, a student should have completed CS 101 or 150, CS
201 or 205, and CS 202.  Furthermore, the student should have
completed, or at least be enrolled in, CS 216.  The student can then
declare the minor by completing the appropriate form, available from
your individual school.  Attach your current transcript, and bring it
to Peggy Reed in Olsson 223.  For additional information, please
contact the computer science minor advisor at
\url{minoradvisor@cs.virginia.edu}.


%----------------------------------------------------------------------

%\ifthenelse{\boolean{printmode}}{\newpage}{}
\section{Frequently Asked Questions}

\subsection{What kind of  advanced placement credit is available?}

Advanced placement (AP) credit is awarded by the University for most
AP tests in which the grade is a 4 or a 5.  This section only deals
with the AP computer science test.  A student's VISTAA report will
always list which courses qualify for the AP test scores (both in
computer science and in other fields).

A 5 on the computer science AB test will receive course credit for CS
101 and CS 201.  A 4 on the computer science AB test OR a 5 on the
computer science A test will receive course credit for CS 101.  If the
AP exam was not in Java, proficiency in Java must be demonstrated
prior to taking CS 201.  Note that CS 201 is required for other
majors: computer engineering, systems engineering, and electrical
engineering.  There is also a placement exam before the fall semester
that will allow the student to place out of CS 101, but does not allow
credit to be received for the course -- the student must then take
another 3 hour CS or technical course instead.  See the next section
for information about the CS 101 placement exam.

\subsection{Can I  place out of CS 101?}

There is a placement exam for CS 101, which covers all the topics
taught in the course.  For the current semester's syllabus, see the CS
101 course website.  Successful completion will allow a student to
place out of the course, but does NOT give course credit -- only a
sufficient score on the AP test or transfer credit can give course
credit for CS 101.  A student must still take CS 201 or a technical
elective to fulfill the SEAS CS 101 requirement.  The test is offered
before the beginning of the fall semester.  Note that any student who
has ever enrolled in CS 101 -- even if they later dropped or withdrew
from the course -- is not allowed to take the placement exam.  The
exam may be taken by visiting the departmental office in Olsson 204.

\subsection{Can CS courses from another college receive credit?}

We officially discourage taking major courses elsewhere. This policy
is especially true for the lab-based and required courses.  If, in
spite of this departmental policy, you still want to take a course
elsewhere, then the student needs an advisor signature AND the
signature of the current instructor of that course from UVa. To
receive the required signatures, you must bring in a detailed
syllabus, so that faculty can make informed decisions.  Note that to
receive credit for CS 216 elsewhere, you need both a data structures
course and an assembly language programming course.


\subsection{What are the Rodman Scholar requirements?}

Rodman scholars have slightly different requirements for graduation.

\begin{itemlist}
\item The requirement for STS 101 is changed to an additional HSS
elective
\item ENGR 141R counts as ENGR 162
\item STS 200R counts as their STS elective
\item PHYS 142R counts for PHYS 142E and PHYS 142W
\item PHYS 241R counts for PHYS 241E and PHYS 241W
\end{itemlist}

Furthermore, Rodman Scholars are required to complete 4 seminars (ENGR
307 or ENGR 308) prior to graduation.

\subsection{Is the number of computer majors capped?}

Although it is the norm that a student be accepted into either of the
computing programs, circumstances can occur that cause a limit in the
number of accepted applicants. Given the continuing expansion of
departmental resources, no limits on the number of majors are expected
in the immediate future. The Department of Computer Science wants you
as a major.  Both the department and the School of Engineering and
Applied Science continue their efforts to ensure that any
student, wishing to do so, can major in computing.

\subsection{Where can I get information on VISTAA?}

VISTAA is the Virginia Student Academic Audit reporting system for
assisting students and their advisors in tracking programs of study. A
student VISTAA report lists both the courses a student has completed
and those in which s/he is currently enrolled.

A VISTAA report groups the courses according to degree, major, and
minor requirements.  These reports also provide information on which
requirements are incomplete and which courses will satisfy them.
VISTAA now supports a student and faculty record request update
feature -- one can use this feature to put in an override for an
improper VISTAA assignment of a course toward a program of study
requirement.

Further information VISTAA is available from the sources listed below.

\begin{itemlist}
\item Home page for VISTAA:
\url{http://www.virginia.edu/registrar/vistaa.html}. The page contains
a list of frequently asked questions. % verified url on 7 dec 2007
\item Detailed student directions on the use of VISTAA:
\url{http://www.virginia.edu/registrar/VISTAA2.pdf}. % verified url on 7 dec 2007
\item Detailed faculty directions on the use of VISTAA:
\url{http://www.virginia.edu/registrar/vistaafsa.html}. % verified url on 7 dec 2007
\item Department faculty VISTAA liaison: \url{vistaa-cs@cs.virginia.edu} % verified url on 7 dec 2007
\item SEAS VISTAA liaison: \url{vistaa-seas@cs.virginia.edu} % verified url on 7 dec 2007
%\item Registrar office VISTAA liaison: \url{rfg@virginia.edu}
\end{itemlist}

It is recommended that you bring a copy of your VISTAA record with you
to advising.

\subsection{Can CS students study abroad?}

To get more information about studying abroad, see
\url{http://www.cs.virginia.edu/curriculum/study_abroad/}.

\subsection{How do I transfer into the CS program?}

There is also the Bachelor of Arts in Computer Science, offered
through the College (see \url{http://www.cs.virginia.edu/ba/}).
Students outside of the School of Engineering and Applied Science with
an interest in obtaining a BS (as opposed to a BA) degree in computer
science must transfer to the Engineering school.

Like other SEAS students, transfer students must formally apply to,
and be approved by, the Department of Computer Science to enroll in
the computer science program of study. To minimize loss of credit upon
transfer, students must take a rigorous program in mathematics and the
sciences. The School of Engineering and Applied Science expects a
minimum of 63 credits in the first two years, instead of the 60-credit
minimum that is customary in the College of Arts and Sciences. The
additional credits are often completed through summer courses.
Detailed information on curriculum requirements may be obtained from
the Office of the Dean of the School of Engineering and Applied
Science.



%----------------------------------------------------------------------

%\ifthenelse{\boolean{printmode}}{\newpage}{}
\section{Course Descriptions}

\label{coursedescriptions}

These course listings are from the undergraduate record
(\url{http://records.ureg.virginia.edu/content.php?catoid=11&navoid=189}).
The frequency code for each class specifies how often it is offered.
(S) means offered each (spring and fall) semester; (Y) means offered
once each academic year, and (SI) means offered upon sufficient
student interest.

%\begin{itemlist}
%\item (S): offered fall and spring semesters
%\item (Y): offered at least once every academic year (fall or spring semester)
%\item (E): offered when the fall semester occurs in an even year (e.g., 2006-2007)
%\item (O): offered when the fall semester occurs in an odd year (e.g., 2005-2006)
%\item (SI): offered upon sufficient student interest
%\item (IR): offered irregularly
%\item (SS): offered during summer session
%\item (J): offered during January session
%\end{itemlist}

\bigskip

{\bf\noindent CS 101 - Introduction to Programming} (3 credits):
Introduces the basic principles and concepts of object-oriented
programming through a study of algorithms, data structures and
software development methods in Java. Emphasizes both synthesis and
analysis of computer programs. (S)
      
{\bf\noindent CS 101E - Introduction to Programming} (3 credits):
Introduces the basic principles and concepts of object-oriented
programming through a study of algorithms, data structures and
software development methods in Java. Emphasizes both synthesis and
analysis of computer programs. (S) Prerequisite: Prior programming
experience.
      
{\bf\noindent CS 101X - Introduction to Programming} (3 credits):
Introduces the basic principles and concepts of object-oriented
programming through a study of algorithms, data structures and
software development methods in Java. Emphasizes both synthesis and
analysis of computer programs. (SI) Note: No prior programming
experience.
      
{\bf\noindent CS 110 - Introduction to Information Technology} (3
credits): Provides exposure to a variety of issues in information
technology, such as computing ethics and copyright. Introduces and
provides experience with various computer applications, including
e-mail, newsgroups, library search tools, word processing, Internet
search engines, and HTML. Not intended for students expecting to do
further work in CS. Cannot be taken for credit by students in SEAS or
Commerce. (S)
      
%{\bf\noindent CS 120 - Introduction to Business Computing} (3
%credits): Overview of modern computer systems and introduction to
%programming in Visual Basic, emphasizing development of programming
%skills for business applications. Intended primarily for pre-commerce
%students. May not be taken for credit by students in SEAS. (S)

{\bf\noindent CS 150 - From Ada and Euclid to Quantum Computing and
the World Wide Web} (3 credits): Introduction to computer science with
no previous background.  Focuses on describing and reasoning about
information processes using language and logic.  Uses motivating
examples from liberal arts and sciences areas such as art, biology,
economics, narrative, physics, and sociology. (Y)

{\bf\noindent CS 201 - Software Development Methods} (3 credits): A
continuation of CS 101, emphasizing modern software development
methods. An introduction to the software development life cycle and
processes. Topics include requirements analysis, specification,
design, implementation, and verification. Emphasizes the role of the
individual programmer in large software development projects. (S)
Prerequisite: CS 101 with a grade of C- or higher.

%{\bf\noindent CS 201K - Software Development Methods} (3 credits):
%Covers tools and techniques used to manage complexity needed to build,
%analyze, and test complex software systems including abstraction,
%analysis, and specification.

{\bf\noindent CS 202 - Discrete Mathematics} (3 credits): Introduces
discrete mathematics and proof techniques involving first order
predicate logic and induction. Application areas include sets (finite
and infinite), elementary combinatorial problems, and
probability. Development of tools and mechanisms for reasoning about
discrete problems. Cross-listed as APMA 202. (S) Prerequisite: CS 101
or 150 with a grade of C- or higher.
      
{\bf\noindent CS 205 - Engineering Software} (3 credits): Covers tools
and techniques used to manage complexity and to build, analyze, and
test complex software systems including abstraction, analysis, and
specification. (Y) Prerequisite: CS 150 with a grade of C- or higher.
Notes: Students may not receive credit for both CS 201 and CS 205.
      
{\bf\noindent CS 216 - Program and Data Representation} (3 credits):
Introduces programs and data representation at the machine level. Data
structuring techniques and the representation of data structures
during program execution. Operations and control structures and their
representation during program execution. Representations of numbers,
arithmetic operations, arrays, records, recursion, hashing, stacks,
queues, trees, graphs, and related concepts. (S) Prerequisite: CS 202
and either CS 201 or CS 205 with all grades of C- or higher.

{\bf\noindent CS 230 - Digital Logic Design} (3 credits): Includes
number systems and conversion; Boolean algebra and logic gates;
minimization of switching functions; combinational network design;
flip-flops; sequential network design; arithmetic networks. Introduces
computer organization and assembly language. Cross-listed as ECE
230. (S)
      
{\bf\noindent CS 290 - Computer Science Seminar I} (1 credit):
Provides cultural capstone to the undergraduate experience. Students
make presentations based on topics not covered in the traditional
curriculum. Emphasizes learning the mechanisms by which researchers
and practicing computer scientists can access information relevant to
their discipline, and on the professional computer scientist's
responsibility in society. (Y) Prerequisite: CS 201 or 205 with a
grade of C- or higher, as well as a computing major (BACS, BSCS, or
CpE).

{\bf\noindent CS 302 - Theory of Computation} (3 credits): Introduces
computation theory including grammars, finite state machines and
Turing machines; and graph theory. Cross-listed as APMA 302. (Y)
Prerequisite: CS 202 and either CS 201 or 250 all with grades of C- or
better.

{\bf\noindent CS 305 - HCI in Software Development} (3 credits):
Human-computer interaction and user-centered design in the context of
software engineering. Examines the fundamental principles of
human-computer interaction. Includes evaluating a systems usability
based on well-defined criteria; user and task analysis, as well as
conceptual models and metaphors; the use of prototyping for evaluating
design alternatives; and physical design of software user-interfaces,
including windows, menus, and commands. (Y) Prerequisite: CS 201 or
205 with a grade of C- or higher.

{\bf\noindent CS 333 - Computer Architecture} (3 credits): Includes
the organization and architecture of computer systems hardware;
instruction set architectures; addressing modes; register transfer
notation; processor design and computer arithmetic; memory systems;
hardware implementations of virtual memory, and input/output control
and devices. Cross-listed as ECE 333. (S) Prerequisite: CS 201 or 205
with a grade of C- or higher, and CS (or ECE) 230 with a grade of C-
or higher.
     
{\bf\noindent CS 340 - Advanced Software Development Techniques} (3
credits): Analyzes modern software engineering practice for
multi-person projects; methods for requirements specification, design,
implementation, verification, and maintenance of large software
systems; advanced software development techniques and large project
management approaches; project planning, scheduling, resource
management, accounting, configuration control, and documentation. (Y)
Prerequisite: CS 216 with a grade of C- or higher.

{\bf\noindent CS 351 - Selected Topics in Computer Science} (1 to 3
credits): Content varies annually, depending on students needs
and interests. Recent topics include the foundations of computation,
artificial intelligence, database design, real-time systems, Internet
engineering, and electronic design automation. (SI) Prerequisite:
Instructor permission.

{\bf\noindent CS 414 - Operating Systems} (3 credits): Analyzes
process communication and synchronization; resource management;
virtual memory management algorithms; file systems; and networking and
distributed systems. (S) Prerequisite: CS 216 and CS (or ECE) 333 with
grades of C- or higher.

{\bf\noindent CS 415 - Programming Languages} (3 credits): Presents
the fundamental concepts of programming language design and
implementation. Emphasizes language paradigms and implementation
issues. Develops working programs in languages representing different
language paradigms. Many programs oriented toward language
implementation issues. (Y) Prerequisite: CS 216 with a grade
of C- or higher.

{\bf\noindent CS 416 - Artificial Intelligence} (3 credits):
Introduces artificial intelligence. Covers fundamental concepts and
techniques and surveys selected application areas. Core material
includes state space search, logic, and resolution theorem
proving. Application areas may include expert systems, natural
language understanding, planning, machine learning, or machine
perception. Provides exposure to AI implementation methods,
emphasizing programming in Common LISP. (Y) Prerequisite: CS 216 with
grades of C- or higher.

{\bf\noindent CS 432 - Algorithms} (3 credits): Introduces the
analysis of algorithms and the effects of data structures on
them. Algorithms selected from areas such as sorting, searching,
shortest paths, greedy algorithms, backtracking, divide- and-conquer,
and dynamic programming. Data structures include heaps and search,
splay, and spanning trees. Analysis techniques include asymptotic
worst case, expected time, amortized analysis, and reductions between
problems. (Y) Prerequisite: CS 216 with a grade of C- or higher.

{\bf\noindent CS 433 - Advanced Computer Architecture} (3 credits):
Provides an overview of modern microprocessor design. The topics
covered in the course will include the design of super-scalar
processors and their memory systems, and the fundamentals of
multi-core processor design. (Y) Prerequisite: CS 216 and CS (or ECE)
333 with a C- or better.

{\bf\noindent CS 434 - Fault-tolerant Computing} (3 credits):
Investigates techniques for designing and analyzing dependable
computer-based systems. Topics include fault models and effects, fault
avoidance techniques, hardware redundancy, error detecting and
correcting codes, time redundancy, software redundancy, combinatorial
reliability modeling, Markov reliability modeling, availability
modeling, maintainability, safety modeling, trade-off analysis, design
for testability, and the testing of redundant digital
systems. Cross-listed as ECE 434. (SI) Prerequisite: CS (or ECE) 333,
APMA 213, and APMA 310, each with grades of C- or higher.

{\bf\noindent CS 441 - Principles of Software Design} (3 credits):
Focuses on techniques for software design in the development of large
and complex software systems. Topics will include software
architecture, modeling (including UML), object-oriented design
patterns, and processes for carrying out analysis and design. More
advanced or recent developments may be included at the instructor's
discretion. The course will balance an emphasis on design principles
with an understanding of how to apply techniques and methods to create
successful software systems.  (Y) Prerequisite: CS 216 with a C- or
better.

{\bf\noindent CS 444 - Introduction to Parallel Computing} (3
credits): Introduces the student to the basics of high-performance
parallel computing and the national cyber-infrastructure. The course
is targeted for both computer science students and students from other
disciplines who want to learn how to significantly increase the
performance of applications.  (Y) Prerequisite: CS 216 and CS (or ECE)
333 with a C- or better.

{\bf\noindent CS 445 - Introduction to Computer Graphics} (3 credits):
Introduces the fundamentals of three-dimensional computer graphics:
rendering, modeling, and animation. Students learn how to represent
three-dimensional objects (modeling) and the movement of those objects
over time (animation). Students learn and implement the standard
rendering pipeline, defined as the stages of turning a
three-dimensional model into a shaded, lit, texture-mapped
two-dimensional image. (Y) Prerequisites: CS 216 with a grade of C-.

%{\bf\noindent CS 446 - Real Time Rendering} (3 credits): Examines
%real-time rendering of high-quality interactive graphics. Studies the
%advances in graphics hardware and algorithms that are allowing
%applications such as video games, simulators, and virtual reality to
%become capable of near cinematic-quality visuals at real-time
%rates. Topics include non-photorealistic rendering, occlusion culling,
%level of detail, terrain rendering, shadow generation, image-based
%rendering, and physical simulation. Over several projects throughout
%the semester students work in small teams to develop a small 3-D game
%engine incorporating some state-of-the-art techniques. (Y)
%Prerequisite: Grade of C- or better in CS 445 or equivalent working
%knowledge.
     
{\bf\noindent CS 447 - Image Synthesis} (3 credits): Provides a broad
overview of the theory and practice of rendering. Discusses classic
rendering algorithms, although most of the course focuses on either
fundamentals of image synthesis or current methods for physically
based rendering. The final project is a rendering competition. (Y)
Prerequisite: Grade of C- or better in CS 445 or equivalent working
knowledge.

%{\bf\noindent CS 448 - Computer Animation} (3 credits): Introduces
%both fundamental and advanced computer animation techniques. Discusses
%such traditional animation topics as keyframing, procedural
%algorithms, camera control, and scene composition. Also introduces
%modern research techniques covering dynamic simulation, motion
%capture, and feedback control algorithms. These topics help prepare
%students for careers as technical directors in the computer animation
%industry and assist in the pursuit of research careers. (Y)
%Prerequisite: Grade of C- or better in CS 445 or equivalent working
%knowledge.

{\bf\noindent CS 451 - Selected Topics in Computer Science} (1 to 3
credits): Content varies annually, depending on students needs and
interests. Recent topics include the foundations of computation,
artificial intelligence, database design, real-time systems, Internet
engineering, wireless sensor networks, and electronic design
automation. (SI) Prerequisite: Instructor permission.

{\bf\noindent CS 453 - Electronic Commerce Technologies} (3 credits):
Focuses on the history of the Internet and electronic commerce on the
web; case studies of success and failure; cryptographic techniques for
privacy, security, and authentication; digital money; transaction
processing; wired and wireless access technologies; Java; streaming
multimedia; XML; Bluetooth. Defining, protecting, growing, and raising
capital for an e-business. (Y) Prerequisite: CS 216 with a grade of C-
or higher.

{\bf\noindent CS 457 - Computer Networks} (3 credits): Intended as a
first course in communication networks for upper-level undergraduate
students. Topics include the design of modern communication networks;
point-to-point and broadcast network solutions; advanced issues such
as Gigabit networks; ATM networks; and real-time communications.
Cross-listed as ECE 457. (Y) Prerequisite: CS (or ECE) 333 with a
grade of C- or higher.

{\bf\noindent CS 458 - Internet Engineering} (3 credits): An advanced
course on computer networks on the technologies and protocols of the
Internet. Topics include the design principles of the Internet
protocols, including TCP/IP, the Domain Name System, routing
protocols, and network management protocols. A set of laboratory
exercises covers aspects of traffic engineering in a wide-area
network. (Y) Prerequisite: CS 457 with a grade of C- or better.

{\bf\noindent CS 462 - Database Systems} (3 credits): Introduces the
fundamental concepts for design and development of database
systems. Emphasizes relational data model and conceptual schema design
using ER model, practical issues in commercial database systems,
database design using functional dependencies, and other data
models. Develops a working relational database for a realistic
application. (Y) Prerequisite: CS 216 with grades of C- or higher.

{\bf\noindent CS 471 - Compilers} (3 credits): Provides an
introduction to the field of compilers, which translate programs
written in high-level languages to a form that can be executed. The
course covers the theories and mechanisms of compilation
tools. Students will learn the core ideas behind compilation and how
to use software tools such as lex/flex, yacc/bison to build a compiler
for a non-trivial programming language. (Y) Prerequisite: CS 340 and
CS (or ECE) 333 with grades of C- or higher.

{\bf\noindent CS 493 - Independent Study} (1 to 3 credits): In-depth
study of a computer science or computer engineering problem by an
individual student in close consultation with departmental
faculty. The study is often either a thorough analysis of an abstract
computer science problem or the design, implementation, and analysis
of a computer system (software or hardware). (S) Prerequisite:
Instructor permission.

{\bf\noindent CS 494 - Special Topics in Computer Science} (1 to 3
credits): Content varies annually, depending on instructor interests
and the needs of the department. Similar to CS 551 and CS 751, but
taught strictly at the undergraduate level. (SI) Prerequisite:
Instructor permission; additional specific requirements vary with
topics.

{\bf\noindent CS 551 - Selected Topics in Computer Science} (1 to 3
credits): Content varies annually, depending on students needs
and interests. Recent topics included the foundations of computation,
artificial intelligence, database design, real-time systems, Internet
engineering, and electronic design automation. (SI) Prerequisite:
Instructor permission.

%{\bf\noindent CS 586 - Real-Time Systems} (3 credits): Presents the
%underlying theory, concepts, and practice for real-time systems, such
%as avionics, process control, space travel, mobile computing and
%ubiquitous computing.  (Y) Prerequisite: CS 414 with a grade of
%C- or higher.

%{\bf\noindent CS 587 - Security in Information Systems} (3 credits):
%This course focuses on security as an aspect of a variety of software
%systems. We will consider software implementations of security related
%policies in the context of operating systems, networks, and data
%bases. (Y) Prerequisite: CS 340 and either CS/ECE 457 or CS 414 with
%grades of C- or higher.

%{\bf\noindent CS 588 - Cryptology: Principles and Applications} (3
%credits): Introduces the basic principles and mathematics of
%cryptology including information theory, classical ciphers, symmetric
%key cryptosystems and public-key cryptosystems. Develops applications
%of cryptology such as anonymous email, digital cash and code
%signing. (Y) Prerequisite: CS 302 with a grade of C- or higher.


%----------------------------------------------------------------------

\ifthenelse{\boolean{printmode}}{}{\newpage}
\section{Degree Requirement Revisions}
\label{sec:degreereqrevisions}

Computer science is an evolving field, and our undergraduate
curriculum reflects this.  The department sometimes makes changes to
the requirements for the bachelor's degree.  Note that you are allowed
to graduate using ANY SINGLE set of requirements that were in effect
when you were a declared computer science major -- thus, if the
requirements change, you are allowed to complete the degree using the
older version of the requirements.  You cannot ``mix and match''
requirements from the different sets.  For example, a student using
the fall 2004 rules below (no general electives) is not allowed to
take something other than ECE 435 (Computer Organization and Design)
for the computer architecture elective.

Any changes to the requirements will occur after the spring semester
and before the following fall semester, unless the change is
considered minor.  A minor change is something that does not in any
way restrict the degree requirements.  Examples of minor changes would
be expanding the allowed courses for one of the elective types, or
clarifying what counts as a technical elective.  Note that unless the
change to the requirements directly affects the third semester (i.e.\
the first semester of the second year), a student cannot choose to
graduate using a set of requirements that were in effect during his or
her first year at UVa but that were not in effect during his or her
second year, as they were not a declared computer science major during
their first year.

The requirement revisions below describe what major changes occurred
during the previous years, and what courses students must complete to
graduate using that set of requirements.  Note that the older sets are
kept for historical reasons, even though there may not be any more
students who are eligible to graduate with those sets.

The current set of requirements, which this document reflects, became
effective in the fall of 2006.  No (non-minor) changes were made for
the fall of 2007.

\subsection{Requirement revision from fall 2005}

The main change in the requirements from the fall of 2005 to the fall
of 2006 was that ECE 435 is no longer an absolute requirement.
Instead, students must choose one course from a list of ``computer
architecture electives.''  The list of acceptable courses is described
on page \pageref{comparchelective}.  Because of the above change, a
student can now graduate with 124.5 credits.

Students graduating using the fall 2005 requirements must take ECE 435,
and are not allowed to take an alternative computer architecture
elective as described on page \pageref{comparchelective}.  However, as
this change (allowing courses other than ECE 435) only expands the
allowed courses a student can take, it is not expected that anyone
will graduate using this set of requirements.

       
\subsection{Requirement revision from fall 2004}

The main change in the requirements from the fall of 2004 to the fall
of 2005 was the addition of general education classes.  Students must
complete 9 credits of general education courses, in addition to the 9
credits of HSS required of all SEAS students. Students now only need
9 technical electives credits (at most 3 credits at 200-level) and 3
credits of unrestricted electives.

Students graduating using the fall 2004 requirements must take 12
credits of technical electives and 9 credits of unrestricted
electives.  This is in addition to the 9 credits of HSS courses
required of all SEAS students.  In addition, students must take ECE
435 (as described above, in the 'fall 2005' requirements section).

Furthermore, CS 390 was renamed to CS 290, and should be taken in the
2nd year.  However, taking either class (290 or 390) will fulfill this
requirement.


\subsection{Requirement revision from fall 2003}

The main change in the requirements from the fall of 2003 to the fall
2004 was the change in math requirements.  Students must take APMA 310 and
then must choose two from APMA 213, APMA 308 or APMA 312.   This means
a student could graduate with 125.5 credits instead of 126.5.

There are currently no students enrolled that are eligible to graduate
using these requirements.

\ifthenelse{\boolean{printmode}}{
\newpage
\section{Individual Notes}
\ 
}{}


%----------------------------------------------------------------------


%\newpage
%\section*{Questions and todos and such}

%\noindent These are things that still need to be done / changed / fixed.

%\begin{enumerate}

%\item make the study abroad e-mail point somewhere

%\item CS 451 needs to be removed from the VISTAA requirements

%\item the ugrad handbook needs to be updated with what's in here -- in
%particular, a number of courses were removed (120, 201K, real-time
%rendering, animation, 5xx).  And CS 202's name needs to change.  And
%descriptions for 205, 433, 444, etc.  And a bunch of pre-reqs.

%\end{enumerate}

%----------------------------------------------------------------------

\ifthenelse{\boolean{printmode}}{

\newpage
\pagestyle{empty}



\vspace*{2in}
\begin{tabular}{p{1.1in}p{4.1in}}
& {\em Enlighten the people generally, and tyranny and oppression of body
and mind will vanish like evil spirits at the dawn of day\ldots\/ the
diffusion of knowledge among the people is to be the instrument by
which it is to be effected. } \\
\\
& --Thomas Jefferson, 1816 \\
\end{tabular}

%\newpage
%\pagestyle{empty}
%\epsfig{figure=mls-bw.jpg,height=\textheight}

\newpage
\pagestyle{empty}
\Large
\vspace*{3in}


\begin{tabular}{c}
\hspace{6.3in} \\
Department of Computer Science \\
School of Engineering and Applied Science \\
The University of Virginia \\
151 Engineer's Way \\
P.O.\ Box 400740 \\
Charlottesville, Virginia 22904-4740 \\
434.982.2200 \\
http://www.cs.virginia.edu \\
\vspace{1in} \\
\epsfig{figure=rolg_blueseal_bw.png,width=1.5in}
%\includegraphics{rolg_blueseal_bw,width=2in}
\end{tabular}

}{}

\end{document}
