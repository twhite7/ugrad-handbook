\noindent Online: \bscsURL

\mysection{Disclaimer}

The information contained on this handbook is for informational
purposes only. The Undergraduate Record and Graduate Record represent
the official repository for academic program requirements. These
publications may be found online\myurl{http://records.ureg.virginia.edu/index.php}$^,$\myurl{http://records.ureg.virginia.edu/preview_program.php?catoid=42&poid=4962}.


\mysection{Introduction}

% this intro drafted by Mark, and sent via e-mail on 9-6-11

The Bachelor of Science degree in Computer Science is a wide-ranging
program, encompassing both the theoretical and the practical.  This
program builds upon the engineering and mathematical principles
introduced in the Engineering School's core curriculum.  Our students
are then taught to apply computing to the world around them by building
faster, smaller, and more secure software systems, exploring emerging
technologies, and working on real-world problems.  Our courses focus
on teaching students how to recognize computational challenges, create
elegant and efficient algorithms, and then use rigorous development
methodologies to build systems that can solve pressing
problems. Graduates of the BS program find successful careers with
traditional software companies, government agencies, consulting firms,
academia, and companies in other fields that have software needs.
Computing professions are often ranked near the top in ``Best Job''
lists put together by news organizations for job availability, pay,
and satisfaction.

Course work in the BS CS program starts with several courses that
introduce the basic principles of software creation, from learning
programming languages to advanced development techniques.  Once
students have mastered the basics, the bulk of our program opens up,
offering electives in several exciting fields, including networking,
security, game design, web programming, e-commerce, parallel
computing, and much more.  Students have the opportunity to take
several electives each semester, as our department offers more
electives than the other departments in the Engineering School. 

\mysection{Program Objectives}

% Loaded from the ugrad record on 7-31-16
% http://records.ureg.virginia.edu/preview_program.php?catoid=42&poid=4962

Graduates of the Bachelor of Science in Computer Science program:

\begin{itemlist}
\item Have the knowledge and skills that allow them to make tangible contributions in their profession.
\item Have the knowledge and skills that allow them meet new technical challenges.
\item Are able to contribute effectively to society.
\item Are able to work effectively as team leaders and members.
\item Have the ability to be innovators in the design, analysis and application of computer systems. 
\end{itemlist}

\noindent {\bf Grading Policy} Majors and minors are required to maintain a C average or better in their CS courses.

\mysection{Application Process}
\label{bscsapplicationprocess}

For SEAS students who started (as first years) in 2016 or later, there
are no major caps for any of the majors -- thus, the application
process is simply a matter of filling out which major one wants to
major in.  Note that this lack of major caps does not apply to the BA
CS (yet), and is for students who entered in 2016 or later.  This is
the same as the Computer Engineering degree.  Also note that the lack
of major caps applies to the first major that one selects during the
major selection period in the spring semester of one's first year.
Transfer students, individuals who want to choose Computer Science as
a second major, or change majors after their first year, may be
subject to major caps.  For details about those, please see the
Computer Science main office (Rice Hall, room 527).

Students may switch between the Computer Science and Computer
Engineering degrees at any time -- once a student is accpeted into the
computing major, they can select either one.  Note, however, that one
will still have to complete all the requirements for the major that
was just switched into.

\mysection{Curriculum}

The requirements for the computer science degree consist of a number
of required courses, as well as a series of elective choices for the
student to make.  A table of all the requirements is shown in
figure~\ref{fig:bscsreqs} (page~\pageref{fig:bscsreqs}).

\subsection{Elective Information}
\label{sec:electiveinfo}

The numbers in the list below correspond to the footnote numbers from
the sample course schedule shown in section~\ref{sec:bscsschedule}
(page~\pageref{sec:bscsschedule}).

Note that classes that receive no grade (including classes that are
audited) do not count toward your degree requirements.

\paragraph{Science elective} (1 required): Students must choose one of 
  BIOL 2010 (Introduction to Biology: Cell Biology and Genetics),
  BIOL 2020 (Introduction to Biology: Organismal and Evolutionary Biology), 
  CHEM 1620 (Introductory Chemistry for Engineers), 
  ECE 2066 (Science of Information), 
  %ENGR 2500 (Introduction to Nano\-science and Technology), 
  MSE 2090 (Introduction to the Science and Engineering of Materials), 
  or
  PHYS 2620 (Introductory Physics IV: Quantum Physics).
  Additional courses in this list can count as an unrestricted
  elective.

\paragraph{HSS (Humanities and Social Science) elective} (5 required): 
  Studies in the humanities and social sciences serve not only to meet
  the objectives of a broad education, but also to meet the objectives
  of the engineering profession. Such course work must meet the
  generally accepted definitions that the humanities are the branches
  of knowledge concerned with humankind and its culture, while the
  social sciences are the studies of society. See the full list of
  allowed courses in the SEAS Undergraduate Handbook. This list can be
  found
  online\myurlremember{hss}{http://www.seas.virginia.edu/advising/undergradhandbook.php}
  and in this handbook in section~\ref{hsselectives}. Note that there
  are a number of courses that do not count as HSS electives, but
  would count as an unrestricted elective. See the URL for details.
  Note that three of the five HSS electives are required by SEAS, and
  the other two are required by the CS department; however, all five
  must comply with the SEAS requirements for HSS electives, or must be
  approved through the petition process described in the SEAS
  undergraduate handbook\myurlrecall{hss}.

\paragraph{Unrestricted elective} (5 required): \label{csunrestricted} 
  Any graded course in the University, with a few exceptions. From the
  SEAS Undergraduate Student Handbook\myurlrecall{hss}: ``Unrestricted
  Electives may be chosen from any graded course in the University
  except mathematics courses below MATH 1310, including STAT 1100 and
  1120, and courses that substantially duplicate any others offered
  for the degree, including PHYS 2010, PHYS 2020, CS 1010, CS 1020, or
  any introductory programming course. Students in doubt as to what is
  acceptable to satisfy a degree requirement should obtain the
  approval of their adviser and the dean's office, Thornton Hall, Room
  A122. APMA 1090 counts as a three credit unrestricted elective for
  students.''  Note that Band classes (such as marching band) and ROTC
  classes can count for the unrestricted elective.

\paragraph{APMA (Applied Mathematics) elective} (2 required): Must 
  choose two from: APMA 2130 (Ordinary Differential Equations), APMA
  3080 (Linear Algebra), APMA 3120 (Statistics), or APMA 3150 (From
  Data to Knowledge). Note that APMA 3100 (Probability) is a required
  course in addition to the two APMA electives.

\paragraph{CS (Computer Science) elective} (5 required): Any 3 credit CS class at the 3000
  level or above. A course that is fulfilling another requirement
  cannot count as a CS elective. If you take more than five CS
  electives, you can count additional CS elective course(s) as
  unrestricted electives. Note that
  % ECE 4435 (Computer Architecture \& Design) and
  ECE 4440 (Embedded Systems Design) can count as a CS elective, but
  see the details in section~\ref{bscscpedoublemajors}
  (page~\pageref{bscscpedoublemajors}).  Due to substantial overlap with
  CS 3330 (Computer Architecture), ECE 4435 (Computer Architecture \&
  Design) can NOT count as a CS elective.  CS 4993
  (Independent Study) can be used at the most once for a CS elective
  (i.e. no more than 3 credits); additional CS 4993 credits can be
  used as unrestricted electives. Note that for a class that does not
  meet these requirements to count as a CS elective requires approval
  by the CS undergraduate curriculum committee (NOT by the student's
  academic advisor); this process can be initiated by emailing the BS
  CS director at \bscsdirectoremail. 
  % the following paragraph is no longer relevant if ECE 4435 does not
  % count as a CS elective any more
  %  
  % Due to substantial overlap, one cannot get credit for both ECE 4435
  % and CS 4330. Thus, if a student takes both of those classes, the
  % other one can ONLY count as a unrestricted elective.

\paragraph{STS 2xxx/3xxx elective} (1 required): Any STS course at the
  2000-level or 3000-level.
 
\subsection{Capstone Sequence}
\label{capstone-section}

All SEAS students must complete a senior thesis, which is encapsulated
in the STS 4500 and STS 4600 courses.  In addition to the STS courses,
BS CS students must complete one or two CS courses to fulfill the
computer science capstone sequence requirement.  There are two
``tracks'' to complete the capstone, described below, and students may
choose either track.

Formally, students must complete 3 credits of one of the two capstone
courses, depending on which track they choose to
fulfill: either CS 4971 (Capstone Practicum II) for the Capstone
Practicum track, or CS 4980 (Research Capstone) for the Research
Capstone track.  But note that CS 4970 (Capstone Practicum I), which
is a CS elective, is a strict prerequisite to CS 4971!  Also note
that STS 4500 and STS 4600 must {\bf still} be taken; the CS courses
are in addition to, not instead of, the STS courses.

\subsubsection{Research Capstone Track}

This track is intended for students who are interested in performing
an independent project, either a research-based project or an
implementation-based project.  The student must seek out a faculty
member, who will agree to advise the project.  The requirements of the
project, workload, etc., are to be agreed upon by the student and
advisor.  Students will receive 3 credits for CS 4980 (Research
Capstone), which is what formally fulfills the capstone requirement
for the degree.  Faculty advisors may decide to assign the three
credits in a single semester, or spread the three credits across
multiple semesters; as long as three credits are eventually earned,
then the requirement will be fulfilled.  Group projects are up to the
discretion of the advisor, but are certainly permissible.  Large
projects may receive additional credit through CS 4993 (Independent
Study), but this is solely up to the advisor.  Note that there is a
maximum of 3 credits (1 course) of CS electives that that CS 4993 may
count toward; any additional credits count toward the unrestricted
elective requirement.

This track is essentially how the senior thesis technical requirement
worked previously, except that students now receive three credits for
the technical work performed.

\subsubsection{Capstone Practicum Track}

This track is intended for any students who are not planning on
performing an independent project.  Students under this track will
register for CS 4970 (Capstone Practicum I) in the fall of their 4th
year, and CS 4971 (Capstone Practicum II) in the spring of their 4th
year~-- specifically, those classes are to be taken concurrently with
STS 4500 and STS 4600, respectively.  Both of these CS courses are 3
credits.

These two courses form a year-long project implementation course.
Students will be grouped into teams, will have requirements, real
customers to interact with, real deadlines, and will need to complete
real deliverables.  While the domain of the course is up to the
instructor, the current implementation of the courses is a Service
Learning Practicum\myurl{http://www.cs.virginia.edu/~asb/slp/}.

Note that only CS 4971 counts toward the capstone requirement; CS 4970
is a CS elective.  However, CS 4970 is a {\bf strict} prerequisite to
CS 4971.  In particular, because the projects are year-long
group-based projects, there will be absolutely no allowances for
students to join the sequence for just CS 4971 in the spring semester.


\subsubsection{Double Majors and the Capstone}

A SEAS student who is double majoring with the BS CS and another SEAS
degree must complete the capstone or major design experience (MDE)
course requirements in their non-CS major {\bf in addition to} the
capstone sequence in the computer science major.  This means that they
will still have to take either CS 4980 (Research Capstone) or the
practicum capstone sequence (CS 4970 and CS 4971). Students can
negotiate with their capstone/MDE instructors and their STS instructor
which degree program's work will be used to satisfy the STS thesis
portfolio requirements. However, the student should be aware that the
capstone/MDE instructors in both departments will almost certainly
require some kind of documentation of the technical work done for that
program's requirement.



\subsection{Degree Requirements Checklist}

The degree requirement checklist is shown in figure~\ref{fig:bscsreqs}
(page~\pageref{fig:bscsreqs}), and is available 
online\myurl{http://handbook.cs.virginia.edu/} in the PDF version of this handbook in a full
(letter-sized) page format (see page~\pageref{fig:bscsreqs}).

\begin{figure*}
\subsection*{BS CS Degree Requirements Checklist}
\label{fig:bscsreqs}
\begin{center}
\ifthenelse{\boolean{lettersize}}{\large}{\small}
\begin{tabular}{|l|l|l|p{1.2in}|} \hline
\bf Required Computing \& Math courses & \bf Grade & \bf Semester &
\bf Comments \\ \hline \hline
CS 1110: Intro. to Computer Science & & & \\ \hline
CS 2110: Software Development Methods & & & \\ \hline
CS 2102: Discrete Mathematics I & & & \\ \hline
CS 2150: Program \& Data Representation & & & \\ \hline
CS/ECE 2330: Digital Logic & & & \\ \hline
CS 2190: CS Seminar I & & & \\ \hline
CS 3102: Theory of Computation & & & \\ \hline
CS 3330: Computer Architecture & & & \\ \hline
CS 3240: Advanced SW Devel. Tech. & & & \\ \hline
CS 4414: Operating Systems & & & \\ \hline
CS 4102: Analysis of Algorithms & & & \\ \hline
Capstone course (circle: CS 4971 or 4980) & & & \\ \hline
APMA 3100: Probability & & & \\ \hline
APMA 2130, 3080, 3120, or 3150 (circle one) & & & \\ \hline
APMA 2130, 3080, 3120, or 3150 (circle one) & & & \\ \hline
\end{tabular}

\noindent\begin{tabular}{@{}ll}
\noindent\begin{tabular}{@{}l}
\\
SEAS required courses \\
\begin{tabular}{|p{1.2in}|l|l|}\hline
\bf Course & \bf Grade & \bf Semester \\ \hline \hline
APMA 1110 & & \\ \hline
APMA 2120 & & \\ \hline
CHEM 1610 & & \\ \hline
CHEM 1611 & & \\ \hline
ENGR 1620 & & \\ \hline
ENGR 1621 & & \\ \hline
PHYS 1425 & & \\ \hline
PHYS 1429 & & \\ \hline
PHYS 2415 & & \\ \hline
PHYS 2419 & & \\ \hline
\end{tabular} \\
\\
Science elective \\
\begin{tabular}{|p{1.2in}|l|l|} \hline
\bf Course & \bf Grade & \bf Semester \\ \hline \hline
\hspace{0.7in} & & \\ \hline
\end{tabular}
\end{tabular}

&

\begin{tabular}{@{}l}
\\
STS courses \\
\begin{tabular}{|p{1.2in}|l|l|} \hline
\bf Course & \bf Grade & \bf Semester \\ \hline \hline
STS 1500 & & \\ \hline
STS 2xxx/3xxx & & \\ \hline
STS 4500 & & \\ \hline
STS 4600 & & \\ \hline
\end{tabular} \\
\vspace{0.15in} \\
CS Electives (5) \\
\begin{tabular}{|l|p{0.9in}|l|l|} \hline
& \bf Course & \bf Grade & \bf Semester \\ \hline \hline
1) & \hspace{0.55in} & & \\ \hline
2) & & & \\ \hline
3) & & & \\ \hline
4) & & & \\ \hline
5) & & & \\ \hline
\end{tabular}
\end{tabular}
\\
& \\
HSS electives (5) & Unrestricted electives (5) \\
\begin{tabular}{|l|p{0.9in}|l|l|} \hline
& \bf Course & \bf Grade & \bf Semester \\ \hline \hline
1) & \hspace{0.45in} & & \\ \hline
2) & & & \\ \hline
3) & & & \\ \hline
4) & & & \\ \hline
5) & & & \\ \hline
\end{tabular}
& 
\begin{tabular}{|l|p{0.9in}|l|l|} \hline
& \bf Course & \bf Grade & \bf Semester \\ \hline \hline
1) & \hspace{0.575in} & & \\ \hline
2) & & & \\ \hline
3) & & & \\ \hline
4) & & & \\ \hline
5) & & & \\ \hline
\end{tabular}
\\
\end{tabular}
\caption{BS CS Degree Requirements Checklist}

\end{center}
\end{figure*}

\normalsize

\subsection{Sample BS CS Course Schedule}
\label{sec:bscsschedule}

Below is the recommended course of study for the bachelor's degree. If
one has already completed some of these classes (through AP credit,
for example), then your course of study would deviate from what is
shown below~-- consult your academic advisor for details.

There are a total of six types of electives that the student can
choose from. These electives are indicated by the footnotes below, and
are described in detail in section~\ref{sec:electiveinfo}
(page~\pageref{sec:electiveinfo}). Note that some of these
requirements are for all SEAS students, while others are required for
the CS bachelor's degree. Please be aware of when the classes are
offered! Some are only offered once per year, or in a particular
semester. See section~\ref{sec:coursedesc}
(page~\pageref{sec:coursedesc}) for details as to when courses are
offered.

The recommended schedule shown below has changed slightly each year as
the degree requirements have evolved. As discussed in the Degree
Requirement Revisions (section~\ref{sec:degreerevisions},
page~\pageref{sec:degreerevisions}), a student can graduate using any
set of requirements that were in effect when they became a declared
computer science major. Thus, as long as all the major requirements
are met, students can follow a previous version of the recommended
course schedule.

Academic requirements are managed by SIS (UVa's Student Information
System), which is where your individual set of requirements can be found.
%A sample of the BS CS requirement listing can be found
%online\myurl{http://www.cs.virginia.edu/bscs/bscs-reqs-in-sis.pdf}.
You may also want to see the FAQ question about how HSS requirements
list in the SIS report (section~\ref{sec:sishssissue},
page~\pageref{sec:sishssissue}).

The sample course schedule below exactly matched the undergraduate
course
record\myurl{http://records.ureg.virginia.edu/preview_program.php?catoid=42&poid=4962}
on July 31, 2016.

\newcolumntype{P}[1]{>{\raggedright\arraybackslash\hangindent=0.15in}p{#1}}

\ifthenelse{\boolean{lettersize}}{
\newcommand{\samplescheduletableheader}{\noindent\begin{tabular}{P{1in}P{2in}c}}
}{
\newcommand{\samplescheduletableheader}{\noindent\begin{tabular}{P{1.25in}P{2.5in}c}}
}

\vspace{0.15in}

%\ifthenelse{\boolean{lettersize}}{\begin{multicols}{2}}{}

\samplescheduletableheader
\und{First semester} & & \und{15} \\
APMA 1110 & Single Variable Calculus & 4 \\
CHEM 1610 & Intro Chemistry I for Engineers & 3 \\
CHEM 1611 & Intro Chem.\ I for Engineers Lab & 1 \\
ENGR 1620 & Introduction to Engineering & 3 \\
ENGR 1621 & Intro.\ to Engineering Lab & 1 \\
STS 1500 {\em or} HSS elective & Science, Tech.,\ \& Contemporary Issues {\em or} HSS elective$^1$ & 3 \\
\\
\end{tabular}\newline\samplescheduletableheader
\und{Second semester} & & \und{17} \\
SCI elective & Science elective$^2$ & 3 \\
HSS elective {\em or} STS 1500 & HSS elective$^1$ {\em or} Science, Tech.,\ \& Contemporary Issues & 3 \\
APMA 2120 & Multivariate Calculus & 4 \\
PHYS 1425 & Physics I: Mechanics, Thermo. & 3 \\
PHYS 1429 & Physics I Workshop & 1 \\
CS 1110 {\em or} 1111 {\em or} 1112 {\em or} 1113 {\em or} 1120 & Introduction to Programming (CS 1110-1113) {\em or} Introduction to Computing (CS 1120) & 3 \\
\\
\end{tabular}\newline\samplescheduletableheader
\und{Third semester} & & \und{16} \\
APMA course & APMA elective$^3$ or APMA 3100 & 3 \\
HSS elective & HSS  elective$^1$ & 3 \\
CS 2110 & Software Develop.\ Methods & 3 \\
CS 2102 & Discrete Mathematics & 3 \\
PHYS 2415 & General Physics II: E\&M \& Lab & 3 \\
PHYS 2419 & General Physics II Workshop & 1 \\ 
\\
\end{tabular}\newline\samplescheduletableheader
\und{Fourth semester} & & \und{16} \\
STS 2xxx/3xxx & STS 2xxx/3xxx elective & 3 \\
UE elective & Unrestricted elective$^4$ & 3 \\
CS 2150 & Prog.\ \& Data Representation & 3 \\
CS/ECE 2330 & Digital Logic Design & 3 \\
CS 2190 & CS Seminar$^6$ & 1 \\
CS 3102 & Theory of Computation & 3 \\
\\
\end{tabular}\newline\samplescheduletableheader
\und{Fifth semester} & & \und{18} \\
APMA course & APMA elective$^3$ or APMA 3100 & 3 \\
HSS elective & HSS elective$^1$ & 3 \\
UE elective & Unrestricted elective$^4$ & 3 \\
CS elective & CS elective & 3 \\
CS 3330 & Computer Architecture & 3 \\
CS 4102 & Algorithms & 3 \\
\\
\end{tabular}\newline\samplescheduletableheader
\und{Sixth semester} & & \und{15} \\
APMA course & APMA elective$^3$ or APMA 3100 & 3 \\
UE elective & Unrestricted elective$^4$ & 3 \\
HSS elective & HSS elective$^1$ & 3 \\
CS elective & CS elective & 3 \\
CS 3240 & Advanced Software Develop. & 3 \\
\\
\end{tabular}\newline\samplescheduletableheader
\und{Seventh semester} & & \und{15} \\
CS elective & CS elective & 3 \\
CS 4970 {\em or} CS elective & Capstone Practicum I$^5$ {\em or} CS elective  & 3 \\
UE elective & Unrestricted elective$^4$ & 3 \\
CS 4414 & Operating Systems & 3 \\
STS 4500 & STS and Engineering Practice & 3 \\
\\
\end{tabular}\newline\samplescheduletableheader
\und{Eighth semester} & & \und{15} \\
CS 4971 {\em or} 4980 & Capstone Practicum II {\em or} Capstone Research & 3 \\
CS elective & CS elective & 3 \\
UE elective & Unrestricted elective$^4$ & 3 \\
HSS elective & HSS elective$^1$ & 3 \\
STS 4600 & Engineer, Ethics, \& Prof.\ Society & 3 \\
\\
\end{tabular}

\noindent{\bf Footnotes}

The footnotes below correspond to the footnote numbers in the sample schedule, above; you can see them described in more detail in section~\ref{sec:electiveinfo}, page~\pageref{sec:electiveinfo}.

\begin{numlist}
\item Chosen from the approved list available in A122 Thornton Hall.
\item Chosen from: BIOL 2010, 2020; CHEM 1620; ECE 2066; MSE 2090; and PHYS 2620.
\item Chosen from APMA 2130, 3080, 3100, 3120 or 3150 (but cannot take both 3120 and 3150).
\item Unrestricted electives may be chosen from any graded course in the University except mathematics courses below MATH 1310 and courses that substantially duplicate any others offered for the degree, including PHYS 2010, 2020; CS 1100, 1200; or any introductory programming course. Students in doubt as to what is acceptable to satisfy a degree requirement should get the approval of their advisor and the dean’s office, located in A122 Thornton Hall. APMA 1090 counts as a three-credit unrestricted elective.
\item The CS capstone experience 4970 and 4971 requires 4th year standing.
\item CS 2190 requires second- or third-year standing. 
\end{numlist}


%\ifthenelse{\boolean{lettersize}}{\end{multicols}}{}

%\ifthenelse{\boolean{lettersize}}{
%\begin{wrapfigure}{r}{0.6\textwidth}
%\vspace{-0.75in}
%\epsfig{figure=diagrams/bs-cs.pdf,width=4.5in}
%\end{wrapfigure}
%}{}

\mysection{Miscellaneous Information}

\subsection{GPA Requirement}

Students must have a 2.0 GPA in all computer science courses in order to complete the degree.  This means all courses with a ``CS'' courses as well as the ECE versions of CS 2330 (Digital Logic Design) and CS 4457 (Computer Networks).  A cumulative GPA less than 2.0 in these courses will prevent successful completion of the degree.  Note that the University policy for Engineering students is such that if you take a course multiple times, then {\em all} of the grades received will count toward the GPA.

\subsection{CS 2190 Specific Details}

While students can take courses in any semester, there is an issue to
consider with CS 2190: this course should be taken in the second year
or (less preferably) the third year. If a student reaches his/her
fourth year without taking the course, then s/he must take a 3 credit
course in ethics and technology in its place (even though CS 2190 is
only 1 credit). This course taken in place of CS 2190 does not count
toward any other requirement except to replace CS 2190.

%\ifthenelse{\boolean{lettersize}}{
%\mysection{Course Requirements\newline Flowchart}
%See the figure to the right.
%}{
\clearpage%\mysection{Course Requirements Flowchart}
\begin{figure*}[h!]
\mysection{Course Requirements Flowchart}
{\em (Updated April 2015)}
\begin{center}
\ifthenelse{\boolean{useflowchartimages}}{
\ifthenelse{\boolean{lettersize}}
{\epsfig{figure=diagrams/bs-cs.png,width=5.5in}}
{\epsfig{figure=diagrams/bs-cs.png,width=4.5in}}
}{
\ifthenelse{\boolean{lettersize}}
{\epsfig{figure=diagrams/bs-cs.pdf,width=5.5in,natwidth=3.40in,natheight=4.87in}}
{\epsfig{figure=diagrams/bs-cs.pdf,width=4.5in,natwidth=3.40in,natheight=4.87in}}
}
\end{center}
\end{figure*}
%}
