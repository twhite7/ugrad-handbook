\noindent Online: \url{http://www.cs.virginia.edu/}

\mysection{Introduction}

Through the development of sophisticated computer systems, processors,
and embedded applications, computer scientists have the opportunity to
change society in ways unimagined several years ago. Our goal is the
education and training of a diverse body of students who can lead this
information technology revolution. To this end, the computing programs
orient students toward the pragmatic aspects of computing and provides
the learning and practices to make them proficient computing
professionals. Computational thinking is rooted in solid mathematics
and science, and grounding in these fundamentals is essential. Our
laboratory environment exposes students to many commercial software
tools and systems, and introduces modern software development
techniques.  In the context of the practice of computing, this early
grounding forms the basis for an education that prepares students for
a computing career.

%With funding from the National Science Foundation, the Department of
%Computer Science has designed and developed a curriculum focused on
%the practice of computing, yet grounded in the mathematical and
%scientific fundamentals of computer science. The curriculum is
%structured around the introduction of modern software development
%techniques in the very beginning courses, and is supported by a set of
%``closed laboratories''.

%In order to provide an environment appropriate to our courses, the
%department has several laboratories with hundreds of workstations.
%These machines have high-resolution graphics and are connected to
%large file handlers, as well as to the University network.  The lab
%courses expose students to many commercial software tools and systems,
%and introduce modern software development techniques via
%object-oriented design and implementation.

%The Department of Computer Science co-offers, with the Department of
%Electrical and Computer Engineering, a degree in Computer Engineering. 

Students have many opportunities to participate in cutting-edge research
with department faculty members. From the senior thesis research
project to independent study, students can pursue research in any
conceivable area. Our former students are enrolled in top graduate
programs across the country. Our undergraduates have won many research
awards, including multiple CRA (Computing Research Association)
awards in recent academic years. In fact, of all US
institutions, UVa is third in overall CRA research awards won.

All graduates of our three computing programs will have the knowledge
and skills to be practitioners and innovators in computing and other
fields.  They will be able to apply computational thinking in the
analysis, design and implementation of computing solutions, whether
working alone or as part of a team. The knowledge and skills acquired
from our degree programs will give students the ability to make
contributions after graduation in their own field as well as to
society at large.

A recent Bureau of Labor Statistics Occupational Outlook Handbook
states that ``very favorable opportunities'' (more numerous job
openings compared to job seekers) can be expected for college
graduates with at least a bachelor's degree in computer
engineering. It also projects an employment increase of over 38\% by
2016 for occupations available to graduates with a bachelor's degree
in computer engineering\myurl{http://www.bls.gov/oco}.

\subsection{Diversity Statement}

The members of the department envision an environment where a
diversity of capable, inspired individuals congregate, interact and
collaborate, to learn and advance knowledge, without barriers. We
embrace this vision because:

\begin{itemlist}
\item We wish to be leaders and role models in reaping and sharing the
 benefits of diversity.
\item We seek to improve the intellectual environment and creative
 potential of our department.
\item We expect to produce happier, more capable and more broadly
 educated computer science graduates.
\item We wish to contribute to social justice and economic well-being
 for all citizens.
\end{itemlist}

\mysection{Degrees Offered}

The Department of Computer Science offers three computing degrees, as well as a minor.

\begin{itemlist}
\item Bachelor of Science (BS in CS) in Computer Science, available to
  students in the School of Engineering and Applied Sciences (SEAS).
\item Bachelor of Arts in Computer Science (BA in CS), available to
  students in the College of Liberal Arts and Sciences (CLAS).
\item Bachelor of Science in Computer Engineering (BS in CpE),
  available to students in the School of Engineering and Applied
  Sciences (SEAS). This degree is shared with the Department of
  Electrical and Computer Engineering.
\item Minor in Computer Science.  Note that the minor is restricted in
  whom can apply for it; see section~\ref{minorapplicationprocess}
  (page~\pageref{minorapplicationprocess}) for details.
\end{itemlist}

Details of the degrees are provided later in this document, but in
this section we explain the differences between computer science and
computer engineering. This explanation is adopted from the ACM and
IEEE's Computing Curricula 2005: The Overview
Report\myurl{http://www.acm.org/education/curricula-recommendations}. We
also give a high-level overview of the difference between our BS and
BA degrees in computer science.

\subsection{What is Computer Science?}

Computer science spans a wide range, from its theoretical and
algorithmic foundations to cutting-edge developments in graphics,
intelligent systems, cybersecurity, and other exciting areas. We can
think of the work of computer scientists as falling into three
categories.

\begin{itemlist}
\item They design and implement software. Computer scientists take on
  challenging programming jobs. They also supervise other programmers,
  keeping them aware of new approaches.
\item They devise new ways to use computers. Progress in the CS areas
  of networking, database, and human-computer-interface enabled the
  development of the World Wide Web. Now CS researchers are working
  with scientists from other fields to develop control physical
  sensors and devices, to use databases to create new knowledge, and
  to use computers to help doctors solve complex problems in medical
  care.
\item They develop effective ways to solve computing problems. For
  example, computer scientists develop the best possible ways to store
  information in databases, send data over networks, and display
  complex images. Their theoretical background allows them to
  determine the best performance possible, and their study of
  algorithms helps them to develop new approaches that provide better
  performance.
\end{itemlist}

Computer science spans the range from theory through
programming. While some universities offer computing degree programs
that are more specialized (such as software engineering,
bioinformatics, etc.), a degree in computer science offers a
comprehensive foundation that permits graduates to adapt to new
technologies and new ideas.

\subsection{Comparison of the BA \& BS Computer Science
  Degrees}

At the University of Virginia, we offer two different computer science
degrees:

\begin{itemlist}
\item The Bachelor of Science (BS) degree, through the School of
  Engineering and Applied Sciences (SEAS), and
\item The Interdisciplinary Major in Computer Science, a Bachelor of
  Arts (BA) degree, through the College of Liberal Arts and Sciences
  (CLAS).
\end{itemlist}

The following gives a high-level comparison of these two degrees.

The BS in Computer Science degree program includes the set of core
courses required of every other engineering degree in SEAS. These
include an introduction to engineering, physics, chemistry, calculus,
courses focused on the engineer's role in society, and at least five
courses in the humanities or social sciences. Like other engineering
majors, all students in our BS program complete a year-long project
leading to a senior thesis in their fourth year. Students in the BS
program can minor in another engineering discipline or applied
math. It is also possible to minor in a subject from the College of
Arts and Sciences (but it's more difficult to have a second major in a
College subject). Students in the BS program must complete at least 46
credits of computer science courses. The Bachelor of Science in
Computer Science is accredited by the Computing Accreditation
Commission of ABET\myurl{http://www.abet.org}.

The BA in Computer Science degree program includes the same general
requirements (known as core and competency requirements) as all other
liberal arts and science degrees in CLAS. These include courses in
foreign language, writing, historical studies, social science,
humanities, and non-western perspectives. These general requirements
also include natural science and mathematics, but fewer courses than
are required for the BS CS in engineering. Students in the BA program
are in a good position to major or minor in another subject in
CLAS. Students with a GPA of 3.4 or better may apply to the
Distinguished Majors Program, in which students complete a thesis
based on two semesters of empirical or theoretical research. Students
in the BA program must complete at least 27 credits of computer
science courses along with 12 additional credits of "integration
electives", which are computing-related courses taught by another
department other than the CS department. Students in the BA program
have the option of taking a version of the first two computing courses
that differ from those taken by the BS students, but otherwise
students from both degree programs share the same CS courses.

Graduates of both programs have been accepted to the best graduate
programs, have received job offers from leading companies, etc. A few
employers have shown a preference for graduates from one program or
the other, but in general both degrees prepare students for excellent
opportunities after graduation.

Students who apply to the University of Virginia must choose to apply
for admission to either SEAS (the engineering school) or CLAS (the
College of Liberal Arts and Sciences). It is possible to transfer from
one unit to the other after admission, and since we offer degrees in
both units a student can major in computer science in either.

\subsection{What is Computer Engineering?}

Computer engineering is concerned with the design and construction of
computers and computer-based systems. It involves the study of
hardware, software, communications, and the interaction among
them. Its curriculum focuses on the theories, principles, and
practices of traditional electrical engineering and mathematics and
applies them to the problems of designing computers and computer-based
devices.

Computer engineering students study the design of digital hardware
systems including communications systems, computers, and devices that
contain computers. They study software development, focusing on
software for digital devices and their interfaces with users and other
devices. At the University of Virginia, the CpE degree has a balanced
emphasis on hardware and software.

At the University of Virginia, computer engineering degrees are
jointly designed and administered by the Department of Computer
Science and the Department of Electrical and Computer Engineering. The
degree program is composed of courses from both departments. 

\subsection{ABET accreditation}

The Bachelor of Science in Computer Science is accredited by the
Computing Accreditation Commission of
ABET\myurlremember{abet}{http://www.abet.org}.  The Bachelor of
Science in Computer Engineering is accredited by the Engineering
Accreditation Commission of ABET\myurlrecall{abet}.


\ifthenelse{\boolean{lettersize}}{}{\clearpage}

\mysection{Major Course Requirements Comparison}

\begin{figure}[h!]
\label{fig:coursereqscomparison}
\begin{center}
\epsfig{figure=diagrams/venn-diagram.pdf,width=3.9in,natwidth=7.22in,natheight=7.43in}
\end{center}
\ifthenelse{\boolean{lettersize}}{\caption{Course Requirements Comparison}}{}
\end{figure}

\ifthenelse{\boolean{lettersize}}{\noindent See figure~\ref{fig:coursereqscomparison}.}{\noindent\begin{tabular}{p{2in}p{2.25in}}}
The SEAS school requirements consist of:
\begin{itemlist}
\item APMA 1110 \& 2120
\item CHEM 1610 \& 1611
\item ENGR 1620 \& ENGR 1621
\item PHYS 1425 \& 1429
\item PHYS 2415 \& 2419
\item Science elective
\item STS 1500
\item STS 2xxx/3xxx elective
\item STS 4500 \& 4600
\end{itemlist}

\ifthenelse{\boolean{lettersize}}{\noindent}{&}
The CLAS school requirements consist of:
\begin{itemlist}
\item First \& second writing requirements
\item Foreign language requirement
\item 6 credits of social sciences
\item 6 credits of humanities
\item 3 credits of historical studies
\item 3 credits of non-western perspectives
\item 12 credits of natural science and math
\end{itemlist}

\ifthenelse{\boolean{lettersize}}{}{\end{tabular}}

\ifthenelse{\boolean{lettersize}}{}{\noindent} A ``CS 1 class'' is CS
1110, CS 1111, or CS 1112.  CLAS majors can take CS 1120, provided
they have Java experience.  Placement is available; see
sections~\ref{applacement} (page~\pageref{applacement}) \&
\ref{101placement} (page~\pageref{101placement}).
