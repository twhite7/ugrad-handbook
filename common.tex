\noindent Online: \csugradURL



\mysection{Cybersecuirty Focal Path}

A focal path is a selection of courses that a student can take to
fulfill the various elective requirements, which are described in
detail in the sections on elective information for the various
majors. They do not change any of the requirements, and students are
not required to follow a focal path. They are included simply to give
prospective majors an idea about various classes that they can take to
fulfill an interest that they may have in computing. %Not all focal
%paths have classes to fulfill each elective requirement. Some will
%list more classes than are needed for the given requirement.


The cybersecurity focal path courses are:

%In an effort to keep down the space for each listing, the reason for
%each class is not listed~-- if interested, speak to a CS faculty
%member in that particular area. Also, as BA CS students may be
%interested in these focal paths, a line listing the BA CS requirements
%may also shown below, if the requirements differ.

%There are a number of other areas for which focal paths are being
%developed, and we expect to include them in future editions of this
%handbook. Those areas are: Systems, Parallel \& Distributed Computing,
%Graphics, Languages \& Compilers, Software Engineering, Hardware, and
%Security \& Privacy.

%\subsection{Cybersecurity}

\iffalse
\begin{itemlist}
\item APMA/STAT elective: Statistics (APMA 3120 or STAT 3120)
\item Theory courses: Theory of Computation (CS 3102), Algorithms (CS
  4102).  Note that one or both may be required, depending on the
  computing major.
\item CS upper level courses: Operating Systems (CS 4414), Computer
  Networks (CS 4457), Defense Against the Dark Arts (CS 4630),
  Database Systems (CS 4750), Network Security (CS 4760) CS 4414:
  Operating Systems.  Some of these are required courses for the
  different computing majors.
\end{itemlist}
\fi

\begin{itemlist}
\item APMA/STAT 3120 Statistics
\item CS 111x Introduction to Programming
\item CS 2150 Program and Data Representation
\item CS 3102 Theory of Computation
\item CS 4102 Algorithms
\item CS 4414 Operating Systems
\item CS/ECE 4457 Computer Networks
\item CS 4630 Defense Against the Dark Arts
\item CS 4750 Database Systems
\item CS 4760 Network Security
\end{itemlist}

Once you have completed these courses, you can apply for the Letter of
Completion of the Cybersecurity Focal Path by filling out
the online application\myurl{https://cyberinnovation.virginia.edu/cybersecurity-focal-path-letter-completion-application}.
Note that this is currently the only focal path we offer that has a 
Letter of Completion.
%(the letter is also a requirement for the CAE
%certification\myurl{https://www.nsa.gov/resources/educators/centers-academic-excellence/cyber-defense/}).

\iffalse

\subsection{Software Engineering}
\begin{itemlist}
\item Science elective (1): Science of Information (ECE 2066)
\item HSS electives (5): See the note (right below this list)
  regarding the Engineering Business Minor.
\item Unrestricted elective (5): See the note (right below this list)
  regarding the Engineering Business Minor.
\item CS electives (5): Advanced Software Development (CS 3240), if it
  is not required for your major.  Principles of Software Design (CS
  4240) and HCI in Software Development (CS 3205) have a strong
  software engineering focus. Other courses that include significant
  development projects would be appropriate, such as Databases (CS
  4750), Web and Mobile Systems (CS 4720), and Computer Game Design
  (CS 4730).
\item STS 2xxx/3xxx elective (1): Any offerings related to technology
  in society or ethics would be appropriate
\item Capstone sequence: the Practicum Capstone track
\end{itemlist}

Notes: If special topics courses were offered in software testing,
software quality, or formal methods, these would be good choices for
this focal path. Also, the Engineering Business Minor would be a good
addition for this focus. Finally, experience within a software company
through a summer internship will increase your understanding of this
area.

%\iffalse

\subsection{Game Design}
\begin{itemlist}
\item Science elective (1): N/A
\item HSS electives (5): anything relating to asset development
  (sound, images, video, etc.)
\item Unrestricted elective (5): digital art classes, such as ARTS
  2220, 2222, 3220, 3222, 4220, and 4222; sound design courses, such
  as DRAM 2620 and DRAM 3640; modeling classes such as ARCH 3410.
  Also consider additional CS electives.
\item APMA electives (2): linear algebra (APMA 3080)
\item CS electives (5): game development courses (offered as special
  topics courses, CS 4501); graphics (CS 4810), artificial
  intelligence (CS 4710), networks (CS 4457), databases (CS 4750),
  parallel computing (CS 4444)
\item STS 2xxx/3xxx elective (1): N/A
\item Notes: You will need a lot of C++ experience upon graduation
\end{itemlist}

\subsection{Theory}
\begin{itemlist}
\item Science elective (1): ECE 2066 (Science of Information)
\item HSS electives (5): mathematical economics (ECON 3090),
  psycho-linguistics (PSYC 4110)
\item Unrestricted electives (5): game theory (ECON 4010), various math
  courses (MATH 4452, MATH 5700, STAT 3010)
\item Unrestricted elective (1): N/A
\item APMA electives (2): linear algebra (APMA 3080)
\item CS electives (5): programming languages (CS 4610), artificial
  intelligence (CS 4710), cryptography (offered as special topics
  courses, CS 4501)
\item STS 2xxx/3xxx elective (1): N/A
\item Notes: BA students need to take CS 3102 which is critical for a
f  theory focal path, but is not (at this time) a required course for
  the BA
\end{itemlist}

\subsection{Networks (including wireless networks)}
\begin{itemlist}
\item Science elective (1): ECE 2066 (Science of Information)
\item HSS (5): N/A
\item Unrestricted electives (5): ECE 2630 (Circuits), ECE 3750
  (Signals), ECE 4710 (Communications), ECE 4209 (Wireless Systems),
  ECE 4785 (Optical Communications)
\item APMA electives (2): APMA 3120 (Statistics)
\item CS electives (5): CS 4457 (Networks), CS 4458 (Internet
  Networks), all wireless sensor networks courses offered as special
  topics, CS 4753 (Electronic Commerce), CS 4720 (Web and Mobile
  Systems), CS 4444 (Parallel Computing)
\item STS 2xxx/3xxx elective (1): N/A
\item Notes: The wireless networking class is often offered as a graduate class (called wireless sensor networks) and can be added with instructor permission. 
%\item Science elective (1): N/A
%\item HSS electives (5): N/A
%\item Unrestricted electives (5): electronic commerce (SYS 2057),
%  hardware based communications (ECE 4710, ECE 4784)
%\item Unrestricted elective (1): electronic commerce (COMM 4240)
%\item APMA electives (2): N/A
%\item CS electives (5): networks (CS 4457), Internet networks (CS
%  4458), wireless networks (offered as special topics courses, CS
%  4501), electronic commerce (CS 4753), parallel computing (CS 4444)
%\item STS 2xxx/3xxx elective (1): N/A
%\item Notes: the wireless networking class is often offered as a
%  graduate class (called wireless sensor networks), and can be added
%  with instructor permission
\end{itemlist}

\subsection{Web Technologies}
\begin{itemlist}
\item Science elective (1): ECE 2066 (Science of Information)
\item HSS (5): Digital art classes, ECON 2010 (Micro Economics), ECON 2020 (Macro Economics), PSYC 1010 (Intro to Psychology)
\item Unrestricted electives (5): STS 4110 (Business of New Product Development)
\item APMA electives (2): N/A
\item CS electives (5): CS 4753 (Electronic Commerce), CS 4457 (Networks), CS 4720 (Web and Mobile Systems), CS 4750 (Database Systems), CS 4240 (Software Design)
\item STS 2xxx/3xxx elective (1): STS 2160 (Intellectual Property)
\item Notes: There are a number of IT classes that are relevant, including courses in web design, technology, and marketing.  However, these are not allowed as unrestricted electives per SEAS policy. 
%\item Science elective (1): ECE 200
%\item HSS and unrestricted electives (10): digital art classes
%\item Unrestricted electives (5): COMM 424 (if no 453), TMP 351
%\item APMA electives (2): N/A
%\item CS electives (5): Electronic commerce (CS 5753); networks (CS
%  4457); Internet networks (CS 4458); web-based courses (offered as
%  special topics courses, CS 4501)
%\item STS 2xxx/3xxx elective (1): STS 2160
%\item Notes: There are a number of IT classes that are relevant,
%  including IT 323 (Web design), IT 332 (Web Tech), IT 334 (Web
%  marketing). However these are not allowed as unrestricted electives
%  as per SEAS policy.
\end{itemlist}


\fi

\mysection{HSS Electives}
\label{hsselectives}

Humanities and Social Science (HSS) courses are required by both of
the Engineering majors: Computer Science requires 5, and Computer
Engineering requires three.  Note that the BA CS curriculum includes
the College
requirements\myurl{http://college.artsandsciences.virginia.edu/requirements/area},
which contain a more in-depth set of HSS type requirements.

The following course mnemonics are generally acceptable for HSS
elective credit. A student may normally take any course under any one
of these mnemonics, with the exception of those listed below.

\setlength\columnsep{5pt}
\begin{multicols}{5}
\small
\begin{itemlist}
\item AAS
\item AMEL
\item AMST
\item AMTR
\item ANTH
\item AR
\item ARAB
\item ARTH
\item ARTR
\item ASL
\item BULG
\item CCFA
\item CCIA
\item CCLT
\item CCSS
\item CHIN
\item CHTR
\item CLAS
\item CPLT
\item CZ
\item EAST
\item ECON
\item ENAM
\item ENCR
\item ENEC
\item ENGL
\item ENGN
\item ENLS
\item ENLT
\item ENMC
\item ENMD
\item ENNC
\item ENRN
\item ENSP
\item ENWR
\item ETP$^*$
\item FREN
\item FRTR
\item GDS
\item GERM
\item GETR
\item GREE
\item HEBR
\item HIAF
\item HIEA
\item HIEU
\item HILA
\item HIME
\item HIND
\item HISA
\item HIST
\item HIUS
\item ITAL
\item ITTR
\item JAPN
\item JPTR
\item KOR
\item LATI
\item LING
\item LNGS
\item MDST
\item MEST
\item MSP
\item MUSI
\item PERS
\item PETR
\item PHIL
\item PLAD
\item PLAP
\item PLCP
\item PLIR
\item PLPT
\item POL
\item PORT
\item POTR
\item PSYC
\item RELA
\item RELB
\item RELC
\item RELG
\item RELH
\item RELI
\item RELJ
\item RELS
\item RUSS
\item RUTR
\item SANS
\item SATR
\item SCAN
\item SLAV
\item SLFK
\item SLTR
\item SOC
\item SPAN
\item SPTR
\item SRBC
\item STS
\item SWAG
\item SWAH
\item SWED
\item TBTN
\item TURK
\item UKR
\item URDU
\item YIDD
\end{itemlist}
\end{multicols}
\setlength\columnsep{10pt} % back to the default

Note that only ETP 2020, 2030, 3870, 4800 can count, as well as EDLF
5000 (but not EDLF 5001).

\paragraph{Exceptions to the approved list.} These are courses in the
acceptable mnemonics that are not suitable for HSS elective credit,
generally because of their specialized nature for majors in that field
or because they are predominantly skills courses:

\begin{itemlist}
\item ANTH: 1090, 3810, 3820, 4991, 4993, 4998, 4999, 5080, 5800, 5870,
5880, 5989
\item ECON: 3710, 3720, 4010, 4350, 4710, 5090, 5100
\item ENSP: 1600
\item GDS: 1100, 4951, 4952
\item MDST: 3702
\item MUSI: 1310, 1993, 2993, 3310, 3320, 3360, 3390, 3993, 4575
\item PSYC: 2200, 2210, 2220, 3005, 3006, 3210, 3870, 3590, 4111, 4125,
4200, 4290, 4330, 4500, 4910, 4970, 4930, 4940, 4980, 5200, 5210,
5260, 5330, 5350, 5401
\item SOC: 4800, 4810, 4820, 4970, 5100, 5110, 5120, 5595, 5596
\item STS: 4110
\end{itemlist}

Elective credit for ANY course with a temporary course number
(often x559 or x595 or x599) must be requested by petition, and
the course syllabus must be attached to the petition.

This list is mainted by the office of the undergraduate dean of SEAS
(Thornton A122), and only that office may grant petitions and
exceptions.

\mysection{Frequently Asked Questions}

\subsection{What computer science student groups exist?}
\label{sec-student-groups} 

At least six computer science student groups at UVa.
Other groups are created by students and faculty from time to time;
if you know of a group we are missing, please submit a pull request
to this handbook.\myurl{https://github.com/uva-cs/ugrad-handbook}

The {\bf Association for Computing Machinery} (ACM) Chapter at the
University of Virginia is a student chapter of the national parent
Association for Computing Machinery. The Chapter is a Contracted
Independent Organization (CIO) at the University of Virginia, and
serves students, faculty, and staff of the University. Numerous events
are held each year, including technical talks, workshops, social
events, hack-a-thons, etc.  Any student at UVa may become a member of
the chapter.  Also see their web
site\myurl{http://acm.cs.virginia.edu/}.

{\bf ACM-W} is the ACM committee on Women in Computing. It celebrates,
informs and supports women in computing, and works with the ACM-W
community of computer scientists, educators, employers and policy
makers to improve working and learning environments for women. Also
see their web site\myurl{http://www.cs.virginia.edu/~acm-w/}.

The {\bf Computer and Network Security Club (CNS)} at UVa is a 
student-run group dedicated to bringing together students interested 
in cybersecurity. In addition to weekly hands-on workshops or industry 
speakers, the club also competes in a variety of CTF and security 
competitions including the Collegiate Cyber Defense Competition (CCDC). 
For more information about the club, please see their 
website\myurl{https://cnsatuva.github.io}.   

{\bf PARFAIT} is a student-run club focused on mobile application
development. It provides a way for student developers to meet and
create project teams so that they can improve their mobile development
skills, develop a portfolio of apps, and provide the University with a
much needed mobile development resource. Find out more using the "contact
owners" link at the parfait mailing list
page\myurl{https://lists.virginia.edu/sympa/info/parfait}.

The {\bf Programming Contest Teams} at UVa participates in the
International Collegiate Programming Contest (ICPC)~-- there is a
regional contest in the fall, and potentially the World Finals in the
late spring.  UVa has had great success recently with earning a World
Finals berth.  The same group of students also host UVa's High School
Programming Contest in the spring, which is the largest such contest
in the mid-Atlantic region.  More information about the programming
contest teams can be found
online\myurl{http://www.cs.virginia.edu/~asb/icpc/}.

The {\bf Student Game Developers} seeks to bring together students who
are interested in learning and experiencing the art of computer game
development. They have resources available for programmers as well as
non-programmers, weekly informative meetings, and many industry
contacts for lectures, resume building, and networking. Also see their
web site\myurl{http://gamedevatuva.blogspot.com/}.

\subsection{What is ICPC, the International Collegiate Programming
  Contest, and how do I get involved?}

The International Collegiate Programming Contests, abbreviated ICPC,
is a world-wide contest of computer programming for college
students. UVa has a very active programming contest team. Regional
contests occur in the fall.  Our region is the nearest 6 (or so)
states and Washington, D.C. The top team(s) from each regional contest
advance to the world finals, which consists of the top 100 teams from
around the world. UVa has qualified for the world finals twice in the
recent years: for the 2009 world finals in Stockholm, Sweden, and the
2010 world finals in Harbin, China. We typically have seven teams (of
three students each) compete in the regional contest. Our programming
contest teams practice throughout the year. If you are interested in
more information, you can either contact UVa's local ACM
chapter\myurl{http://acm.cs.virginia.edu} or ACM's advisor,
\acmadvisor\ (\acmadvisoremail).

\subsection{What kind of advanced placement credit is available?}
\label{applacement}

Advanced placement (AP) credit is awarded by the University for most
AP tests in which the grade is a 4 or a 5. This section only deals
with the AP Computer Science test. A student's SIS report will
always list which courses qualify for the AP test scores (both in
computer science and in other fields).

Students who receive a 4 or 5 on the Computer Science A test will
receive credit for CS 1110.  Students who took the higher-level Computer
Science course through International Baccalaureate will get credit for
CS 1110 with a 5 and for both CS 1110 and CS 2110 with a 6 or 7.

Note that CS 2110 is required for other majors: computer engineering,
systems engineering, and electrical engineering. There is also a
placement exam before the fall semester that will allow the student to
place out of CS 1110, but does not allow credit to be received for the
course~-- the student must then take another 3 hour CS or technical
course (see your advisor for details about a 'technical course')
instead. See the next question and answer for information about the CS
1110 placement exam.

\subsection{Can I place out of CS 1110? What about CS 2110?}
\label{101placement}

There is a placement exam for CS 1110, which covers all the topics
taught in the course. For the current semester's syllabus, see the
current CS 1110 course web site. Successful completion of the test
will allow a student to place out of the course, but does NOT give
course credit~-- only a sufficient score on an AP or IB test, or
transfer credit, can give course credit for CS 1110.

Because course credit is not awarded for the placement exam, another
course must be taken instead in order to fulfill the graduation
requirements.  The different majors have different requirements for
what needs to be taken in lieu of CS 1110.

\begin{itemlist}
\item The BS CS requires a technical course instead (see your advisor
for details about a 'technical course').
\item The BA CS does not require CS 1110 as a major requirement;
  instead, it is a prerequisite, so there is no need to replace it.
\item The BS CpE requires an unrestricted elective instead; see
section~\ref{csunrestricted} (page~\pageref{csunrestricted}) for
details on unrestricted electives; note that while that section is in
the BS CS section, the requirements for unrestricted electives are
defined by the school, and are the same for all SEAS majors.
\item Other Engineering majors will have different rules as to what
  can be substituted; consult your advisor in that department for
  details.
\end{itemlist}

The test is offered before the beginning of the fall semester. Note
that any student who has enrolled in CS 1110 or equivalent (CS 1111 or
CS 1112) and got a letter grade~-- including a 'W'~-- is not allowed
to take the placement exam (in other words, if you enroll and then
drop the course without a 'W', you may still take the placement
exam).

More information about the CS 1110 placement exam may be found online
\myurl{http://www.cs.virginia.edu/cs1110/placement.php}, or you can
visit the computer science departmental office in Rice Hall, room 527.

Computing majors may place out of CS 2110, but they must take another
CS course of a greater number (i.e., greater than CS 2110) instead.
For information about the placement exam for CS 2110, please contact
the current CS 2110 instructor.

\subsection{How does SEAS handle transfer credit?}

The Engineering School handles transfer credit, such as from an AP
course or transfer from another school. The credit will appear on your
SIS report, along with the UVa courses that you received credit
for. Note that the credit amounts need to match - so if you are
getting credit for APMA 2120 (Multivariate Calculus), which is a
4 credit course, the number of credits you transfer in should
(ideally) also be 4 credit hours. If it does not (your equivalent
course at another school was only 3 credits), you will have to take
another math or technical course (see your advisor for details about a
'technical course') to make up for the discrepancy. Note that placing
out of a course (such as CS 1110, APMA 2120, etc.) through
the respective placement exam does not give credit – and thus the
credits need to be made up through other courses (in the case of CS
1110, 3 credits of a technical course will fill that spot; in
the case of APMA 2120, 4 credits of math or a technical elective
will fill that spot). AP exams do give course credit.

Note that half of the 128 credits that one uses to graduate must be
earned at UVa. Thus, if you transfer with more than 64 credits, you
must still take 64 credits at UVa.

\subsection{Can CS courses from another college receive credit?}

We officially discourage taking major courses elsewhere. This policy
is especially true for the lab-based and required courses. If, in
spite of this departmental policy, you still want to take a course
elsewhere, then the student needs an advisor signature AND the
signature of the current instructor of that course from UVa. To
receive the required signatures, you must bring in a detailed
syllabus, so that faculty can make informed decisions. Note that to
receive credit for CS 2150 elsewhere, you need a course (or
multiple courses) that cover(s) data structures, C++, and assembly
language programming.

\subsection{What are the Rodman Scholar requirements?}

Rodman Scholars\myurl{http://seas.virginia.edu/students/rodmans/} have
slightly different
requirements\myurlremember{rodmanreqs}{http://seas.virginia.edu/students/rodmans/curriculum.php}
for graduation.  While those requirement differences are included
here, one should check at that URL\myurlrecall{rodmanreqs} for the
most up-to-date information, as that URL will supersede the
information in this section.

\begin{itemlist}
\item Introduction to Engineering: Instead of ENGR 1620 and ENGR 1621,
  Rodman Scholars take ENGR 1410 (Synthesis Design I) in the fall of
  their first year, and ENGR 1420 (Synthesis Design II) in the spring
  of their first year.  Note that this sequence is 6 credits, whereas
  the ENGR 1620 and ENGR 1621 sequence is only 4 credits.  Because
  ENGR 1410 and ENGR 1420 this is a fall-spring sequence, mid-year
  applicants are not required to take it, and can use ENGR 1620/1621
  credit instead.
\item A total of four 1-credit seminars (ENGR 3580) must be taken
  prior to graduation.
\item The requirements in place of STS 1500 are in a state of flux.
  For the classes of 2014 and earlier, STS 1500 is not required, but
  an HSS elective (or appropriate AP credit) must be taken in its
  stead.  For the classes of 2015 and beyond, STS 1500 is required,
  but with a separate Rodman-only discussion section.
\item Physics: Instead of PHYS 1425 and PHYS 2415 (and their
  associated labs), Rodmans may choose
  to take a three-course sequence for Physics majors: PHYS 1610, PHYS
  1620, and PHYS 2610.
\end{itemlist}

\subsection{What is a SIS exception and how can I get one?}
\label{sec:sisexceptions}

SIS, which is UVa's Student Information System, keeps track of the
requirements that all students must fulfill in order to graduate.  If
any student has an approved reason to differ from the stated
guidelines, then a SIS exception has to be entered.  Examples would
include using a different course for a BA CS integration elective
(section~\ref{sec:bacs-integrationelectives},
page~\pageref{sec:bacs-integrationelectives}), or a BS CS \& BS CpE
double major wanting to count ECE 4440 (Embedded Systems Design) as a CS
elective (section~\ref{bscscpedoublemajors},
page~\pageref{bscscpedoublemajors}).  Note that SIS exceptions are only
entered with the appropriate approval.  Your academic advisor can
request approval from the appropriate degree committee, and have such
an exception entered by contacting the SEAS undergraduate office.

\iffalse 
\subsection{Why are ECE 4435 and ECE 4440 not showing up in my list of
  fulfilled CS electives?}
\label{sec:sisece4435issue}

ECE 4435 (Computer Architecture \& Design) does not count as a CS
elective, as there is too much overlap with CS 3330 (Computer
Architecture); see section~\ref{bscscpedoublemajors}
(page~\pageref{bscscpedoublemajors}) for details.

As for ECE 4440 (Embedded Systems Design), this has to do with a
restriction in how SIS handles the CS elective requirements.  While it
can count as a CS elective, in order for this to happen a SIS
exception will need to be entered. 
office.  Note that this course can only count as one elective each,
even though it is a 4.5 credit course (meaning that only 3 credits
counts toward the CS elective requirement).

\fi

\subsection{Why do the SIS requirements for the BS CS major list 6 HSS
  electives, and not 5?}
\label{sec:sishssissue}

This has to do with how SIS (the Student Information System, UVa's
system for handling academic requirements and registration) handles
major requirements, and is done to allow for people to place out of
STS 1500. If one does not place out of STS 1500, then STS 1500 will
list both in the STS 1500 requirement, and in the HSS requirement,
thus requiring students to take 5 additional HSS courses. If one does
place out of STS 1500, they need to take an additional HSS course in
its place. So the credit to place out of STS 1500 will appear in the
STS 1500 requirements, and will still require 6 (not 5) HSS courses.
We think this is all a bit bizarre as well, but that is how SIS
handles requirements.

%A sample of the BS CS requirements can be found
%online\myurl{http://www.cs.virginia.edu/bscs/bscs-reqs-in-sis.pdf}~--
%your individual one can be found via SIS.

\subsection{Can CS students study abroad?}

Yes! To get more information about studying abroad, see
online\myurl{http://www.cs.virginia.edu/~horton/study_abroad/}
for more details.
 
\subsection{How do I transfer into the computing program?}

Students must decide which school they want to receive their degree
in: either in the Engineering School (which will yield a BS CS or BS
CpE degree; chapter~\ref{bscschapter} (page~\pageref{bscschapter})
details the BS CS degree, and chapter~\ref{bscpechapter}
(page~\pageref{bscpechapter}) details the BS CpE degree) or the
College of Arts and Sciences (which will yield a BA CS degree,
detailed chapter~\ref{bacschapter} (page~\pageref{bacschapter}).
Students must then apply for the degree in that school; the
application process is described in
section~\ref{bscsapplicationprocess}
(page~\pageref{bscsapplicationprocess}) for the BS CS and CpE degrees,
or section~~\ref{bacsapplicationprocess}
(page~\pageref{bacsapplicationprocess}) for the BA CS degree.

%Like other SEAS students, transfer students must formally apply to,
%and be approved by, the Department of Computer Science to enroll in
%the computer science program of study. To minimize loss of credit upon
%transfer, students must take a rigorous program in mathematics and the
%sciences. The School of Engineering and Applied Science expects a
%minimum of 63 credits in the first two years, instead of the 60-credit
%minimum that is customary in the College of Arts and Sciences. The
%additional credits are often completed through summer
%courses. Detailed information on curriculum requirements may be
%obtained from the Office of the Dean of the School of Engineering and
%Applied Science.
%
%There is also the Bachelor of Arts in Computer Science, offered
%through the College; also see their main
%web page\myurl{http://www.cs.virginia.edu/ba/}. Students
%outside of the School of Engineering and Applied Science with an
%interest in obtaining a BS (as opposed to a BA) degree in computer
%science must transfer to the Engineering school.


\subsection{Where can I find out about the Business minor?}

The courses for the Engineering Business Minor can be found on the web
page for the Technology Entrepreneurship Program.  These courses can
be worked into the various electives for the BS CS. More details can
be found online\myurl{http://techentrepreneurship.seas.virginia.edu/}.


\subsection{What CS electives can be taken without having completed CS
  2150?}

There are a few CS electives that one can take having only taken CS
2110.  They include:
\begin{itemlist}
\item CS 3102, Theory of Computation (this is a required class for the
  BS CS, and an elective for the BA CS and the BS CpE)
\item CS 3205, Human-Computer Interaction
\end{itemlist}

Note that CS 2150 is now a pre-requisite for CS 3330, so any course
that requires CS 3330 (such as CS 4434 and CS 4457) still require CS
2150.

\subsection{Why is CS 2330, Digital Logic Design, not offered
  in the spring?}
\label{cs2330}

CS 2330, Digital Logic Design, is cross-listed with ECE 2330.  Either
course counts for this requirement, and it does not matter which one
you take.  For unknown reasons, it is not cross-listed with CS 2330 in
the spring, but it is in the fall.  We don't understand why, either.
But you can take ECE 2330 to fulfill this requirement, as it's all the
same course.

\subsection{How do I conquer the world?}

You are on your own on that one.  But getting a computing degree from
UVa is a good start!


\ifthenelse{\boolean{lettersize}}{}{\clearpage}

\mysection{Course Descriptions}
\label{sec:coursedesc}

%Table~\ref{table:course-offering-history}
%(page~\pageref{table:course-offering-history}) gives this history of
%which courses were taught when for the last few years.

\subsection{Course Offering History}

A history of the course offerings by the department can be found
online\myurl{http://rabi.phys.virginia.edu/mySIS/CS2/page.php?Type=Group&Group=CompSci}.
These course listings in this section are from the undergraduate
record\myurl{http://records.ureg.virginia.edu/content.php?catoid=11\&navoid=189}.

Most of the lower-level courses (1000-level and 2000-level) are
offered every semester.  The exception is CS 2190, the 1 credit
Computer Science Seminar, which is only offered in the spring.  Note
that CS 1120 is no longer regularly offered by the department.

Required courses in the 3000-level and 4000-level are typically
offered every semester.  These include: CS 3102 (Theory of
Computation), CS 3240 (Advanced Software Development Techniques), CS
3330 (Computer Architecture), CS 4102 (Algorithms), and CS 4414
(Operating Systems).

The department tries to offer each elective at least once a year.
Note that the department cannot guarantee this, as when an elective is
offered is dependent on the available faculty.  Popular electives tend
to be offered every semester.  Examples of the popular elecives
include: CS 4630 (Defense Against the Dark Arts), CS 4720 (Web and
Mobile Systems), and CS 4750 (Database Systems).

Special topics courses (CS 1501, 2501, 3501, and 4501) are offered on
varying schedules.  Because CS 4501 is tyipcally used for new
electives, there are often multiple offerings each semester, each one
on a different topic.  Independent project courses are available every
semester, but they are only available by instructor permission.  These
independent courses include: CS 4980 (Capstone Research), CS 4993
(Independent Study), and CS 4998 (Distinguished BA Majors Research).

There are a few courses that are no longer regularly offered by the
department.  These include: CS 4434 (Dependable Computing Systems) and
CS 4458 (Internet Engineering).

\iffalse

\ifthenelse{\boolean{lettersize}}{The table below shows which courses
  were offered in recent semesters.}{ See
  table~\ref{table:course-offering-history} on
  page~\pageref{table:course-offering-history}.  The table shows which
  courses were offered in recent semesters.}

\newcommand{\X}{\checkmark}
\newcommand{\courseofferingtable}{
\begin{tabular}{l|cccccccccccccc}
\label{table:course-offering-history}
Course  & 08 & 09 & 09 & 10 & 10 & 11 & 11 & 12 & 12 & 13 & 13 & 14 & 14 \\
        & F  & S  & F  & S  & F  & S  & F  & S  & F  & S  & F  & S  & F  \\ \hline\hline
CS 1010 & \X & \X & \X & \X & \X & \X & \X & \X & \X & \X & \X & \X & \X \\
CS 1110 & \X & \X & \X & \X & \X & \X & \X & \X & \X & \X & \X & \X & \X \\
CS 1111 & \X & \X &    & \X & \X & \X & \X & \X & \X & \X & \X & \X &    \\
CS 1112 & \X & \X & \X & \X & \X & \X & \X & \X & \X & \X & \X & \X & \X \\
CS 1120 &    & \X & \X & \X & \X & \X & \X & \X & \X & \X &    & \X &    \\
CS 1501 &    &    &    &    &    &    &    &    &    &    &    & \X &    \\
CS 2102 & \X & \X & \X & \X & \X & \X & \X & \X & \X & \X & \X & \X & \X \\
CS 2110 & \X & \X & \X & \X & \X & \X & \X & \X & \X & \X & \X & \X & \X \\
CS 2150 & \X & \X & \X & \X & \X & \X & \X & \X & \X & \X & \X & \X & \X \\
CS 2190 &    & \X &    & \X &    & \X &    & \X &    & \X &    & \X &    \\
%CS 2220 & \X &    & \X &    & \X &    & \X &    &    &    &   & \X & \X \\
CS 2330 & \X & \X & \X & \X & \X & \X & \X & \X & \X & \X & \X & \X & \X \\
CS 2501 & \X &    &    &    & \X &    & \X &    &    &    & \X & \X &    \\
CS 3102 &    & \X & \X & \X & \X & \X & \X & \X & \X & \X & \X & \X & \X \\
CS 3205 & \X &    &    &    &    & \X &    & \X &    & \X & \X & \X & \X \\
CS 3240 &    & \X &    & \X &    & \X &    & \X & \X & \X & \X & \X & \X \\
CS 3330 & \X & \X & \X & \X & \X & \X & \X & \X & \X &    & \X & \X & \X \\
CS 4102 & \X &    & \X & \X & \X & \X & \X & \X & \X & \X & \X & \X & \X \\
CS 4240 &    &    & \X &    & \X &    & \X &    & \X &    & \X &    & \X \\
CS 4330 &    & \X &    &    &    &    &    &    &    &    &    &    &    \\
CS 4414 & \X & \X & \X & \X & \X & \X & \X & \X & \X & \X & \X & \X & \X \\
CS 4434 &    &    &    & \X &    &    &    & \X &    &    &    &    &    \\
CS 4444 & \X &    & \X &    & \X &    & \X &    &    & \X &    &    &    \\
CS 4457 & \X & \X & \X & \X & \X & \X & \X & \X & \X & \X & \X &    & \X \\
CS 4458 & \X &    &    &    &    &    &    &    &    &    &    &    &    \\
CS 4501 &    &    &    &    & \X & \X &    &    & \X & \X &    & \X & \X \\
CS 4610 &    &    & \X &    &    &    &    & \X &    &    &    & \X &    \\
CS 4620 &    &    &    &    &    &    &    &    &    &    & \X &    & \X \\
CS 4630 &    & \X & \X & \X &    & \X &    & \X &    & \X & \X & \X &    \\
CS 4710 &    & \X &    &    &    & \X &    & \X &    & \X &    & \X &    \\
CS 4720 &    &    & \X &    & \X &    & \X &    & \X &    & \X & \X & \X \\
CS 4730 &    &    &    & \X &    &    & \X &    &    &    &    & \X &    \\
CS 4740 &    &    &    &    &    &    &    & \X &    & \X &    &    &    \\
CS 4750 & \X &    &    & \X &    & \X &    & \X &    & \X &    & \X &    \\
CS 4753 & \X &    & \X &    & \X &    & \X &    & \X &    & \X &    & \X \\
CS 4810 & \X &    &    & \X &    &    & \X &    & \X &    &    & \X & \X \\
CS 4970 &    &    &    &    &    &    &    & \X & \X &    & \X &    & \X \\
CS 4971 &    &    &    &    &    &    &    &    &    & \X &    & \X &    \\
\end{tabular}
}

\begin{table}[h!]
\begin{center}
\rowcolors{2}{gray!25}{white} % from http://tex.stackexchange.com/questions/5363/how-to-create-alternating-rows-in-a-table

\ifthenelse{\boolean{lettersize}}
{\resizebox{3.625in}{!}{\courseofferingtable}}
{\courseofferingtable}

\caption{Computer Science Course Offering History.  Note that many courses were offered via their ECE cross-listed course or via a 4501 course.}
\end{center}
\end{table}

\fi

\subsection{1000 Level CS Courses}

\course{CS 1010}{Introduction to Information Technology}{3
  credits}{Provides exposure to a variety of issues in information
  technology, such as computing ethics and copyright. Introduces and
  provides experience with various computer applications, including
  e-mail, newsgroups, library search tools, word processing, Internet
  search engines, and HTML. Not intended for students expecting to do
  further work in CS. Cannot be taken for credit by students in SEAS
  or Commerce.}{}

%\course{CS 1020}{Introduction to Business Computing}{3
%  credits}{Overview of modern computer systems and introduction to
%  programming in Visual Basic, emphasizing development of programming
%  skills for business applications. Intended primarily for
%  pre-commerce students. May not be taken for credit by students in
%  SEAS.}{}

\course{CS 1110}{Introduction to Programming}{3 credits}{A first 
course in programming, software development, and computer science. 
Introduces computing fundamentals and an appreciation for 
computational thinking. No previous programming experience 
required. Note: CS 1110, 1111, 1112, 1113, and 1120 provide 
different approaches to teaching the same core material; students 
may only receive credit for one of these courses.}{}

\course{CS 1111}{Introduction to Programming}{3 credits}{A first 
course in programming, software development, and computer science. 
Introduces computing fundamentals and an appreciation for 
computational thinking.}{Prerequisite: Prior programming experience.}

\course{CS 1112}{Introduction to Programming}{3 credits}{A first 
course in programming, software development, and computer science. 
Introduces computing fundamentals and an appreciation for 
computational thinking. Note: No prior programming experience
  allowed.}{}

\ifthenelse{\boolean{lettersize}}{}{\clearpage}

\course{CS 1120}{Introduction to Computing: Explorations in Language,
  Logic, and Machines}{3 credits}{This course is an introduction to
  the most important ideas in computing. It focuses on the big ideas
  in computer science including the major themes of recursive
  definitions, universality, and abstraction.  It covers how to
  describe information processes by defining procedures using the
  Scheme and Python programming languages, how to analyze the costs
  required to carry out a procedure, and the fundamental limits of
  what can be computed.}{}

\course{CS 1501}{Special Topics in Computer Science}{1
  to 3 credits}{Content varies annually, depending on students needs
  and interests. Recent topics include the foundations of computation,
  artificial intelligence, database design, real-time systems,
  Internet engineering, wireless sensor networks, and electronic
  design automation.}{Prerequisite: Instructor permission.}

\subsection{2000 Level CS Courses}

\course{CS 2102}{Discrete Mathematics}{3 credits}{Introduces discrete
  mathematics and proof techniques involving first order predicate
  logic and induction. Application areas include finite and infinite
  sets and elementary combinatorial problems. Development of tools and
  mechanisms for reasoning about discrete problems.}{Prerequisite: CS
  1110, 1111, 1112 or 1120 with a grade of C- or higher.}

\course{CS 2110}{Software Development Methods}{3 credits}{A
  continuation of CS 1010, emphasizing modern software development
  methods. An introduction to the software development life cycle and
  processes. Topics include requirements analysis, specification,
  design, implementation, and verification. Emphasizes the role of the
  individual programmer in large software development
  projects.}{Prerequisite: CS 1010, 1111, or 1112 with a grade of C-
  or higher.}

\course{CS 2150}{Program and Data Representation}{3 credits}
{Introduces programs and data representation at the machine level.
  Data structuring techniques and the representation of data
  structures during program execution. Operations and control
  structures and their representation during program execution.
  Representations of numbers, arithmetic operations, arrays, records,
  recursion, hashing, stacks, queues, trees, graphs, and related
  concepts.}{Prerequisite: CS 2102 and CS 2110, both with grades of C-
  or higher.}

\course{CS 2190}{Computer Science Seminar}{1 credit}{Provides cultural
  capstone to the undergraduate experience. Students make
  presentations based on topics not covered in the traditional
  curriculum. Emphasizes learning the mechanisms by which researchers
  and practicing computer scientists can access information relevant
  to their discipline, and on the professional computer scientist's
  responsibility in society.}{Prerequisite: CS 2102 and 2110, both
  with a grade of C- or higher.}

\course{CS/ECE 2330}{Digital Logic Design}{3 credits}{Includes number
  systems and conversion; Boolean algebra and logic gates;
  minimization of switching functions; combinational network design;
  flip-flops; sequential network design; arithmetic networks.
  Introduces computer organization and assembly language. Cross-listed
  as ECE 2330.}

\course{CS 2501}{Special Topics in Computer Science}{1
  to 3 credits}{Content varies annually, depending on students needs
  and interests. Recent topics include the foundations of computation,
  artificial intelligence, database design, real-time systems,
  Internet engineering, wireless sensor networks, and electronic
  design automation.}{Prerequisite: Instructor permission.}
  
  
\course{CS 2910}{CS Education Practicum}{1 credit}{An overview of computer science education for undergraduate students. Topics include ethics, diversity, tutoring and teaching techniques, and classroom management. Students enrolled in this course serve as a teaching assistant for a computer science course as part of their coursework.}{Prerequisite: Hired as a TA for a CS course.}


\subsection{3000 Level CS Courses}

\course{CS 3102}{Theory of Computation}{3 credits}{Introduces
  computation theory including grammars, finite state machines and
  Turing machines; and graph theory.}{Prerequisites: CS 2102 and CS
  2110 both with grades of C- or higher.}

\course{CS 3205}{HCI in Software Development}{3
  credits}{Human-computer interaction and user-centered design in the
  context of software engineering. Examines the fundamental principles
  of human-computer interaction. Includes evaluating a system's
  usability based on well-defined criteria; user and task analysis, as
  well as conceptual models and metaphors; the use of prototyping for
  evaluating design alternatives; and physical design of software
  user-interfaces, including windows, menus, and commands.}
{Prerequisite: CS 2110 with a grade of C- or higher.}

\course{CS 3240}{Advanced Software Development Techniques}{3
  credits}{Analyzes modern software engineering practice for
  multi-person projects; methods for requirements specification,
  design, implementation, verification, and maintenance of large
  software systems; advanced software development techniques and large
  project management approaches; project planning, scheduling,
  resource management, accounting, configuration control, and
  documentation.}{Prerequisite: CS 2150 with a grade of C- or higher.}


\course{CS 3250}{Software Testing}{3 credits}{An introduction to testing for assuring software quality. Covers concepts and techniques for testing software, including testing at the unit, module, subsystem, and system levels; automatic and manual techniques for generating and validating test data; the testing process; static vs. dynamic analysis; functional testing; inspections; testing in specific application domains; and reliability assessment.}{Prerequisite: CS 2150}

\course{CS 3330}{Computer Architecture}{3 credits}{Includes the
  organization and architecture of computer systems hardware;
  instruction set architectures; addressing modes; register transfer
  notation; processor design and computer arithmetic; memory systems;
  hardware implementations of virtual memory, and input/output control
  and devices.}{Prerequisite: CS 2150 with a grade of C- or higher. CS
  2330 recommended. Students may not receive credit for both CS 3330
  and ECE 3430.}

\course{CS 3501}{Special Topics in Computer Science}{1
  to 3 credits}{Content varies annually, depending on students needs
  and interests. Recent topics include the foundations of computation,
  artificial intelligence, database design, real-time systems,
  Internet engineering, wireless sensor networks, and electronic
  design automation.}{Prerequisite: Instructor permission.}

\course{CS 3710}{Introduction to Cybersecurity}{3 credits}{Introduces students to the fields of cybersecurity.  Both non-technical issues, such as ethics and policy, and technical issues are covered. Students see and experiment with a wide range of areas within cybersecurity, including: binary exploitation, encryption, digital forensics, networks, and modern threats.}{Prerequisite: CS 2110; have taken or concurrently enrolled in CS 2150}


\subsection{4000 Level CS Courses}

\course{CS 4102}{Algorithms}{3 credits}{Introduces the analysis of
  algorithms and the effects of data structures on them. Algorithms
  selected from areas such as sorting, searching, shortest paths,
  greedy algorithms, backtracking, divide- and-conquer, and dynamic
  programming. Data structures include heaps and search, splay, and
  spanning trees. Analysis techniques include asymptotic worst case,
  expected time, amortized analysis, and reductions between problems.}
  {Prerequisite: CS 2102 and 2150 with grades of C- or higher. Also
  APMA 1090 or MATH 1210 or MATH 1310.}

\course{CS 4240}{Principles of Software Design}{3 credits}{Focuses on
  techniques for software design in the development of large and
  complex software systems. Topics will include software architecture,
  modeling (including UML), object-oriented design patterns, and
  processes for carrying out analysis and design. More advanced or
  recent developments may be included at the instructor's discretion.
  The course will balance an emphasis on design principles with an
  understanding of how to apply techniques and methods to create
  successful software systems.}{Prerequisite: CS 2150 with grade of C-
  or higher.}
  
  
\course{CS 4260}{Internet Scale Applications}{3 credits}{A survey of methods for building large-scale internet websites and mobile apps, with a focus on how theory meets practice. Topics covered include performance engineering, scaling, security, and large team software engineering. Results in students building a working scalable online application.}{Prerequisite: CS 3240}


\course{CS 4330}{Advanced Computer Architecture}{3 credits}{Provides
  an over\-view of modern microprocessor design. The topics covered in
  the course will include the design of super-scalar processors and
  their memory systems, and the fundamentals of multi-core processor
  design.}{Prerequisite: CS 2150 and CS 3330, both with grades of C-
  or higher.}

\course{CS 4414}{Operating Systems}{3 credits}{Analyzes process
  communication and synchronization; resource management; virtual
  memory management algorithms; file systems; and networking and
  distributed systems.}{Prerequisite: CS 2150 with grade of C- or
  higher, CS/ECE 2330 with a grade of C- or higher, and CS 3330 or ECE
  3430 with a grade of C- or higher.}

\course{CS 4434}{Dependable Computing Systems}{3 credits}{Focuses on
  the techniques for designing and analyzing dependable computer-based
  systems. Topics include fault models and effects, fault avoidance
  techniques, hardware redundancy, error detecting and correcting
  codes, time redundancy, software redundancy, combinatorial
  reliability modeling, Markov reliability modeling, availability
  modeling, maintainability, safety modeling, trade-off analysis,
  design for testability, and the testing of redundant digital
  systems.}

\course{CS 4444}{Introduction to Parallel Computing}{3
  credits}{Introduces the student to the basics of high-performance
  parallel computing and the national cyber-infrastructure. The course
  is targeted for both computer science students and students from
  other disciplines who want to learn how to significantly increase
  the performance of applications.}{Prerequisites: CS 2110 with grade
  of C- or higher, CS/ECE 2330 with a grade of C- or higher, CS3330 or
  ECE 3430 with a grade of C- or higher, APMA 3100 and APMA 3110.}

\course{CS/ECE 4457}{Computer Networks}{3 credits}{ Topics include the
  design of modern communication networks; point-to-point and
  broadcast network solutions; advanced issues such as Gigabit
  networks; ATM networks; and real-time communications.  Cross-listed
  as ECE 4457.}{Prerequisites: CS 2110 with grade of C- or higher, and
  CS 3330 or ECE 3430 with a grade of C- or higher.}

\course{CS 4458}{Internet Engineering}{3 credits}{An advanced course
  on computer networks on the technologies and protocols of the
  Internet. Topics include the design principles of the Internet
  protocols, including TCP/IP, the Domain Name System, routing
  protocols, and network management protocols. A set of laboratory
  exercises covers aspects of traffic engineering in a wide-area
  network.}{Prerequisite: CS 4457 with a grade of C- or better.}

\course{CS 4501}{Special Topics in Computer Science}{1
  to 3 credits}{Content varies annually, depending on students needs
  and interests. Recent topics include the foundations of computation,
  artificial intelligence, database design, real-time systems,
  Internet engineering, wireless sensor networks, and electronic
  design automation.}{Prerequisite: Instructor permission.}

\course{CS 4610}{Programming Languages}{3 credits}{Presents the
  fundamental concepts of programming language design and
  implementation. Emphasizes language paradigms and implementation
  issues. Develops working programs in languages representing
  different language paradigms. Many programs oriented toward language
  implementation issues.}{Prerequisite: CS 2150 with grade of C- or
  higher.}

\course{CS 4620}{Compilers}{3 credits}{Provides an introduction
  to the field of compilers, which translate programs written in
  high-level languages to a form that can be executed. The course
  covers the theories and mechanisms of compilation tools. Students
  will learn the core ideas behind compilation and how to use software
  tools such as lex/flex, yacc/bison to build a compiler for a
  non-trivial programming language.}{Prerequisite: CS2150 with grade
  of C- or higher. CS3330 recommended.}

\course{CS 4630}{Defense Against the Dark Arts}{3 credits}{Viruses,
  worms, and other malicious software are an ever-increasing threat to
  computer systems. There is an escalating battle between computer
  security specialists and the designers of malicious software. This
  course provides an essential understanding of the techniques used by
  both sides of the computer security battle.}{Prerequisite: CS 3710}

\course{CS 4640}{Programming Languages for Web Applications}{3 credits}{Presents programming languages and implementations used in developing web applications. Both client and server side languages are presented as well as database languages. In addition, frameworks that enable interactive web pages are discussed as well as formatting languages. Language features and efficiencies including scoping, parameter passing, object orientation, just in time compilation and dynamic binary translation are included.}{Prerequisite: CS 2150 with a grade of C- or higher.}

\course{CS 4710}{Artificial Intelligence}{3 credits}{Introduces
  artificial intelligence. Covers fundamental concepts and techniques
  and surveys selected application areas. Core material includes state
  space search, logic, and resolution theorem proving. Application
  areas may include expert systems, natural language understanding,
  planning, machine learning, or machine perception. Provides exposure
  to AI implementation methods, emphasizing programming in Common
  LISP.}{Prerequisite: CS 2150 with grade of C- or higher.}

\course{CS 4720}{Mobile Application Development}{3 credits}{Mobile computing devices have become ubiquitous in our communities. In this course, we focus on the creation of mobile solutions for various modern platforms, including major mobile operating systems. Topics include mobile device architecture, programming languages, software engineering, user interface design, and app distribution.}{Prerequisite: CS 2150 with a grade of C- or higher.}

\course{CS 4730}{Computer Game Design}{3 credits}{This course will
  introduce students to the concepts and tools used in the development
  of modern 2-D and 3-D real-time interactive computer video games.
  Topics covered in this include graphics, parallel processing,
  human-computer interaction, networking, artificial intelligence, and
  software engineering.}{Prerequisite: CS 2150 with a grade of C- or
  higher.}

\course{CS 4740}{Cloud Computing}{3 credits}{Investigates the
  architectural foundations of the various cloud platforms, as well as
  examining both current cloud computing platforms and modern cloud
  research.  Student assignments utilize the major cloud
  platforms.}{Prerequisite: CS 2150 with a grade of C- or higher.}

\course{CS 4750}{Database Systems}{3 credits}{Introduces the
  fundamental concepts for design and development of database systems.
  Emphasizes relational data model and conceptual schema design using
  ER model, practical issues in commercial database systems, database
  design using functional dependencies, and other data models.
  Develops a working relational database for a realistic
  application.}{Prerequisite: CS 2150 with grades of C- or higher.}

\course{CS 4753}{Electronic Commerce Technologies}{3 credits}{History
  of Internet and electronic commerce on the web; case studies of
  success and failure; cryptographic techniques for privacy, security,
  and authentication; digital money; transaction processing; wired and
  wireless access technologies; Java; streaming multimedia; XML;
  Bluetooth. Defining, protecting, growing, and raising capital for an
  e-business.}{Prerequisite: CS 2150 with a grade of C- or higher.}

\course{CS 4760}{Network Security}{3 credits}{This course covers the principles of secure network communications and the application of network security. Topics include: attack types, attack surfaces, attack phases, network security devices.(a)symmetric key encryption, cryptographic hash function, authentication/identification techniques, key distribution, and data integrity assurance. Also, currently used security mechanisms and protocols will be discussed.}{Prerequisite: CS 3710 and CS/ECE 4457}

\course{CS 4774}{Machine Learning}{3 credits}{An introduction to machine learning: the study of algorithms that improve their performance through experience. Covers both machine learning theory and algorithms. Introduces algorithms, theory, and applications related to both supervised and unsupervised learning, including regression, classification, and optimization and major algorithm families for each.}{Prerequisites: CS 2150; and either Math 3350 or APMA 3080; and one of APMA 3100, APMA 3110, MATH 3100, or equivalent.}

\course{CS 4780}{Information Retrieval}{3 credits}{An introduction to modern information retrieval technologies. Topics include indexing, query processing, document ranking, query recommendation, personalization, and other current topics in information retrieval. Students develop a custom search engine as part of this course.}{Prerequisites: CS2150 and one of APMA 3100, APMA 3110, MATH 3100, or equivalent.}


\course{CS 4810}{Introduction to Computer Graphics}{3
  credits}{Introduces the fundamentals of three-dimensional computer
  graphics: rendering, modeling, and animation. Students learn how to
  represent three-dimensional objects (modeling) and the movement of
  those objects over time (animation). Students learn and implement
  the standard rendering pipeline, defined as the stages of turning a
  three-dimensional model into a shaded, lit, texture-mapped
  two-dimensional image.}{Prerequisite: CS 2150 with a grade of
  C- or higher.}

%\course{CS 4820}{Real Time Rendering}{3 credits}{Examines real-time
%  rendering of high-quality interactive graphics. Studies the advances
%  in graphics hardware and algorithms that are allowing applications
%  such as video games, simulators, and virtual reality to become
%  capable of near cinematic-quality visuals at real-time rates. Topics
%  include non-photorealistic rendering, occlusion culling, level of
%  detail, terrain rendering, shadow generation, image-based rendering,
%  and physical simulation. Over several projects throughout the
%  semester students work in small teams to develop a small 3-D game
%  engine incorporating some state-of-the-art
%  techniques.}{Prerequisite: Grade of C- or better in CS 4810 or
%  equivalent working knowledge.}

%\course{CS 4830}{Computer Animation}{3 credits}{Introduces both
%  fundamental and advanced computer animation techniques. Discusses
%  such traditional animation topics as keyframing, procedural
%  algorithms, camera control, and scene composition. Also introduces
%  modern research techniques covering dynamic simulation, motion
%  capture, and feedback control algorithms. These topics help prepare
%  students for careers as technical directors in the computer
%  animation industry and assist in the pursuit of research
%  careers.}{Prerequisite: Grade of C- or better in CS 4810 or
%  equivalent working knowledge.}

\course{CS 4970}{Capstone Practicum I}{3 credits}{This course is one
  option in the CS Senior Thesis track. Under the Practicum track,
  students will take two 3-credit courses, CS 4970 and CS 4971. These
  courses would form a year-long group-based and project-based
  practicum class. There would be an actual customer, which could be
  either internal (the course instructor, other CS professors, etc.)
  or external (local companies, local non-profits,
  etc.).}{Prerequisite: CS 2150 with a grade of C- or higher.}

\course{CS 4971}{Capstone Practicum II}{3 credits}{This course is one
  option in the CS Senior Thesis track and is the continuation from CS
  4970. Under the Practicum track, students will take two 3-credit
  courses, CS 4970 and CS 4971. These courses would form a year-long
  group-based and project-based practicum class. There would be an
  actual customer, which could be either internal (the course
  instructor, other CS professors, etc.) or external (local companies,
  local non-profits, etc.).}{Prerequisite: CS 4970.}

\course{CS 4980}{Capstone Research}{1 to 3 credits}{This course is one
  option in the CS Senior Thesis track. Students will seek out a
  faculty member as an advisor, and do an independent project with
  said advisor. Instructors can give the 3 credits across multiple
  semesters, if desired. This course is designed for students who are
  doing research, and want to use that research for their senior
  thesis. Note that this track could also be an implementation
  project, including a group-based project.}{Prerequisite: CS 2150
  with a grade of C- or higher.}

\course{CS 4993}{Independent Study}{1 to 3 credits}{In-depth study of
  a computer science or computer engineering problem by an individual
  student in close consultation with departmental faculty. The study
  is often either a thorough analysis of an abstract computer science
  problem or the design, implementation, and analysis of a computer
  system (software or hardware).}{Prerequisite: Instructor
  permission.}

\course{CS 4998}{Distinguished BA Majors Research}{3 credits}{Required
  for Distinguished Majors completing the Bachelor of Arts degree in
  the College of Arts and Sciences. An introduction to computer
  science research and the writing of a Distinguished Majors
  thesis.}{Prerequisites: CS 2150 with a grade of C- or higher AND a
  CLAS student.}
 
\subsection{Selected ECE Courses}

This section is not meant to be an exhaustive list of all courses in
the Electrical Engineering department.  Instead, it is meant to list the
required courses for the Computer Engineering majors.  Information
about the other Electrical Engineering courses offered can be found
online\myurl{http://www.ece.virginia.edu/}.  Note that cross-listed
courses (CS/ECE 2330 (Digital Logic Design), CS 3330 (Computer
Architecture), and CS/ECE 4457 (Networks)) are only listed above.

\vspace{0.25in}

\course{ECE 2630}{Introductory Circuit Analysis}{3 credits}{Elementary
  electrical circuit concepts and their application to linear circuits
  with passive elements; use of Kirchhoff's voltage and current laws
  to derive circuit equations; solution methods for first- and
  second-order transient and DC steady-state responses; AC
  steady-state analysis; frequency domain representation of signals;
  trigonometric and complex Fourier series; phasor methods; complex
  impedance; transfer functions and resonance; Thevenin / Norton
  equivalent models; controlled sources. Six laboratory
  assignments.}{Prerequisite: APMA 1110.}

\course{ECE 2660}{Electronics I}{3 credits}{Studies the modeling,
  analysis, design, computer simulation, and measurement of electrical
  circuits which contain non-linear devices such as junction diodes,
  bipolar junction transistors, and field effect transistors. Includes
  the gain and frequency response of linear amplifiers, power
  supplies, and other practical electronic circuits. Three lecture and
  three laboratory hours.}{Prerequisite: ECE 2630.}


\course{ECE 3430}{Introduction to Embedded Computing Systems}{3
  credits}{ An embedded computer is designed to efficiently and
  (semi-) au\-ton\-o\-mous\-ly perform a small number of tasks, interacting
  directly with its physical environment. This lab-based course
  explores architecture and interface issues relating to the design,
  evaluation and implementation of embedded systems . Topics include
  hardware and software organization, power management, digital and
  analog I/O devices, memory systems, timing and
  interrupts.}{Prerequisite: CS/ECE 2330, CS 2110, ECE 2660--if ECE
  3430 offered in spring}

\course{ECE 3750}{Signals and Systems I}{3 credits}{Develops tools for
  analyzing signals and systems operating in continuous-time, with
  applications to control, communications, and signal processing.
  Primary concepts are representation of signals, linear
  time-invariant systems, Fourier analysis of signals, frequency
  response, and frequency-domain input/output analysis, the Laplace
  transform, and linear feedback principles. Practical examples are
  employed throughout, and regular usage of computer tools (Matlab,
  CC) is incorporated. Students cannot receive credit for both this
  course and BIOM 3310.}{Prerequisite: ECE 2630 and APMA 2130.}

\course{ECE 4435}{Computer Architecture \& Design}{3 credits}{
  Introduces computer architecture and provides a foundation for the
  design of complex synchronous digital devices, focusing on: 1)
  Established approaches of computer architecture, 2) Techniques for
  managing complexity at the register transfer level, and 3) Tools for
  digital hardware description, simulation, and synthesis. Includes
  laboratory exercises and significant design activities using a
  hardware description language and simulation.}{Prerequisite: ECE
  3430}

\course{ECE 4440}{Embedded System Design}{3 credits}{ Modeling,
  analysis and design of embedded computer systems. Tradeoff analysis
  and constraint satisfaction facilitated by the use of appropriate
  analysis models. Includes a semester-long design of an embedded
  system to meet a specific need. Counts as MDE (major design
  experience) for both electrical and computer engineering
  students.}{}


\mysection{Course Numbering}
\label{course-numbering}

Starting with the fall 2009 semester, the University of Virginia
changed all course numbers to 4-digit numbers from the old 3-digit
number system. Earlier versions of this handbook listed both the both
the 3-digit number and the 4-digit number, in the form of ``CS 1110
(101)'' to aid the transition, as well as a full table mapping the
3-digit course numbers to the 4-digit course numbers.  The current
version no longer lists the courses that way.  This handbook no longer
lists the old 3-digit course numbers, but they can be found
online\myurl{http://www.virginia.edu/registrar/atoz.html\#CS}.

The new 4-digit course numbers follow a system developed by the
department. The first digit is the year that the course is expected to
be taken. The second digit specifies the type of course, as shown
below. The third and fourth digits attempted to keep the previous last
two digits of the 3-digit course number, although that was not always
possible.


The 2nd digit numbering scheme is:

\begin{itemlist}
\item x000: service courses, courses for non-majors, general interest
\item x100: core, fundamentals, theoretical (a broad category)
\item x200: software development-oriented courses (note in ECE, this will
 be for electronics courses)
\item x300: hardware, architecture, etc.
\item x400: computer systems
\item x500: by University rule: ``special-topics and variable one-time
 offerings''
\item x600: languages, compilation, etc.
\item x700: application areas including AI, databases, etc.
\item x800: computer graphics
\item x900: by University rule: thesis, dissertation, independent
 study, capstone, etc.
\end{itemlist}

Note that currently cross-listed courses with ECE fall in the x300 and
x400 categories.

\mysection{Degree Requirement Revisions}
\label{sec:degreerevisions}

Computer science is an evolving field, and our undergraduate
curriculum reflects this. The department sometimes makes changes to
the requirements for the bachelor's degree. Note that you are allowed
to graduate using ANY SINGLE set of requirements that were in effect
when you were a declared computer science major~-- thus, if the
requirements change, you are allowed to complete the degree using the
older version of the requirements. You cannot ``mix and match''
requirements from the different sets. Whatever set of requirements is
completed, it must be {\em all} the requirements from that set.

Any changes to the requirements will typically occur after the spring
semester and before the following fall semester, unless the change is
considered minor. A minor change is something that does not in any way
restrict the degree requirements. Examples of minor changes would be
expanding the allowed courses for one of the elective types, or
clarifying what counts as a given elective. Note that unless the
change to the requirements directly affects the third semester (i.e.
the first semester of the second year), a student cannot choose to
graduate using a set of requirements that were in effect during his or
her first year at UVa but that were not in effect during his or her
second year, as they were not a declared computer science major during
their first year.

The requirement revisions below describe which major changes occurred
during the previous years, and what courses students must complete to
graduate using that set of requirements. Note that the older sets are
kept for historical reasons, even though there may not be any more
students who are eligible to graduate with those sets.

The current set of requirements, which this document reflects, became
effective in the spring of 2013 (and before then-first year SEAS
majors declared the BS CS and CpE majors).

\subsection{Prior requirements revisions}

There are no known currently enrolled students who are eligible to graduate
using any set of requirements prior to those listed above, and thus
prior sets of revisions are not included in this handbook.  Earlier
editions of this handbook, available from the department, describe
the prior sets of requirements.

\iffalse

\subsection{Requirements revision from spring 2013}

The inclusion BS CS capstone requirement added 3 additional credits
(CS 4971 (Capstone Practicum II) or CS 4980 (Research Capstone))
needed to complete the BS CS degree.  In addition, CS 4970 (Capstone
Practicum I) was added as a CS elective.  Details about the capstone
can be found in section~\ref{capstone-section}
(page~\pageref{capstone-section}).

For the BS CpE, the embedded systems requirement was added, as that
was approved in the spring of 2013.  As a result of those changes, ECE
4435 (Computer Architecture \& Design) can no longer count as a CS
elective, as the content is to similar to that of CS 3330 (Computer
Architecture).

CS 2330 (Digital Logic Design) was removed as a prerequisite for
CS/ ECE 3330 (Computer Architecture), and added as a prerequisite for
CS 4330 (Advanced Computer Architecture) and CS 4434 (Dependable
Computing).  Note that because of the Embedded Systems requirement
change for CpE, BS CpE majors no longer will be taking CS 3330.
This requirement change also affects the BA CS, as they no longer need
to take CS 2330 as a CS elective in order to enroll in CS 3330.

UVa moved to 4-digit course numbers in the fall of 2009, and previous
versions of this handbook listed both the 3-digit and 4-digit numbers
for the courses.  The current version no longer lists the 3-digit
versions throughout the document.  However, the course numbering
section (section~\ref{course-numbering},
page~\pageref{course-numbering}) still lists the mapping, for
reference.  Anybody interested in the original 3-digit course numbers
can find the mapping
online\myurl{http://www.virginia.edu/registrar/atoz.html\#CS}.

\subsection{Requirements revision from spring 2010}

In January of 2010, the elective structure was changed. Previously,
majors were required to take 3 HSS electives, 3 general education
electives, 3 technical electives, and 1 unrestricted elective. With
the change, these 10 elective courses are now split into 5 HSS
electives and 5 unrestricted electives. Students wishing to graduate
using the old rules (if you were a declared major prior to 2010)
should see the previous editions of this handbook for the description
of what constitutes general education electives and technical
electives. However, the new requirements are more general, and we
expect most students to graduate using these updated requirements. The
old versions of this handbook are available from the department.

\subsection{Requirements revision from fall 2009}

In addition to the course numbering change, the change in the
requirements was that the computer architecture elective was replaced
with an additional CS elective, to bring the total number of required
CS electives to 5. The previous computer architecture requirement had
the students take one class from a set of 3: CS 4444
(Introduction to Parallel Computing), CS 4330 (Advanced Computer
Architecture) or ECE 4435 (Computer Architecture and
Design). Since all of those three courses count as CS electives,
students who have already fulfilled this old requirement will still
fulfill the CS elective that replaced it.

Focal paths were also added to the undergraduate handbook, although
they do not change the major requirements.

\fi

\clearpage
\vspace*{3in}
\begin{center}
This page intentionally left blank.
\end{center}

\clearpage
\vspace*{3in}
\begin{center}
This page intentionally left blank.
\end{center}

\clearpage
\vspace*{3in}
\begin{center}
This page intentionally left blank.
\end{center}
