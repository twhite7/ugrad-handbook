\noindent Online: \bacsURL

\mysection{Introduction}

Computer Science is the study of information processes. Computer
scientists learn how to describe information processes, how to reason
about and predict properties of information processes, and how to
implement information processes elegantly and efficiently in hardware
and software. The Computer Science major concentrates on developing
the deep understanding of computing and critical thinking skills that
will enable graduates to pursue a wide variety of possible fields and
to become academic, cultural, and industrial leaders in areas that
integrate the arts and sciences with computing. The Computer Science
major is designed to provide students entering the University without
previous background in computing with an opportunity to major in
Computer Science, while taking courses in arts, humanities, and
sciences to develop broad understanding of other areas and their
connections to computing. Computing connects closely with a wide range
of disciplines including, but not limited to, the visual arts, music,
life sciences including biology and cognitive science, the physical
sciences, linguistics, mathematics, and the social sciences. The core
curriculum focuses on developing methods and tools for describing,
implementing, and analyzing information processes and for managing
complexity including abstraction, specification, and recursion. 


\mysection{Application Process}
\label{bacsapplicationprocess}

The Department of Computer Science is experiencing tremendous student
interest in our degree programs. Our goal is to accommodate as many BA
in Computer Science (BA CS) majors as possible.  However, because of
current resources, the Department has had to institute a cap on the
number of students who can declare the BA CS major.


{\bf Requirements to declare the major:} In order to apply for the
major, students must have taken one introductory computer science
course (either CS 1110, CS 1111, or CS 1112, or CS 1120) with a grade
of C+ or better, and must be enrolled in CS 2110 and CS 2102 (or must
have already completed CS 2110 and CS 2102 with a grade of C+ or
better). Students are accepted into the major in the spring semester
of their second year upon review of their applications. This is a
selective process which takes into account the applicant's GPA and
application essay, as well as other factors.

{\bf Application information:} Applications must be completed in the
spring semester (normally the student's fourth semester).  Deadlines
are posted in the Computer Science Department office and on the
departmental web site; the deadline will be on or about March 15.  Due
to prerequisite dependencies, it is difficult for rising third year
students who have not completed CS 2110 and CS 2102 to complete the
major in the 4 remaining semesters; however, in exceptional cases,
students in that situation may apply for the major by petition to the
Chair.

Students apply to the Computer Science major by completing a form
available on the departmental web
site\myurlFormatted{\bacsDeclaringURL}. Students
list all CS courses taken or in progress and discuss any career goals
or aspirations, computing-related extra-curricular activities,
internships or experience. The essay invites students to reflect on
their intellectual objectives in wishing to pursue the major and asks
students to consider their career goals and how the BA in CS advances
those goals. Applications are read and evaluated by the core faculty
in Computer Science. All applicants will be notified of admission
decisions by April 1.

{\bf Second majors:} College of Arts and Sciences students who wish to
declare the BA CS as a second major must follow the same application
process described here. Only College of Arts and Sciences students are
eligible to apply for the BA CS degree as a second major.

{\bf Transfer students from outside the University:} Students
transferring into the University from other institutions must apply to
the department to be allowed to declare the BA CS major. Qualified
applicants will be considered on a space-available basis, given our
target caps for each class year. Applications will be considered the
summer before a transfer student begins classes, and the application
process will be discussed during the summer orientation session. If an
incoming transfer does not attend summer orientation, they must meet
with a CS advisor before classes begin to discuss applying.  Due to
prerequisite dependencies, it is difficult for rising third year
students who have not completed CS 2110 and CS 2102 to complete the
BA CS in the 4 remaining semesters. It is important that students
transferring to the University as third years complete the equivalent
of these courses before coming to UVa. In exceptional cases, students
in this situation may apply for the major, but the ability to complete
the degree in a timely fashion is considered in determining if
you are accepted into the degree program.




\mysection{Curriculum}

%\paragraph{Prerequisites:}

Before declaring the computer science major, all students should have
taken one introductory computer science course (either CS 1110, CS
1111, or CS 1112; CS 1120 is also allowed if the student has Java
experience) with a grade of C+ or better, or have comparable
experience. Students may be permitted to declare the major while they
are currently taking the introductory course.

The major requires the College Competency and Area
Requirements\myurl{http://artsandsciences.virginia.edu/college/requirements/index.html}
as well as at least 30 credits in Computer Science courses and 12
credits in Integration Electives.

\subsection{Required Core Courses}
\label{sec:bacs-corecourses}

The following courses are required for all BA CS majors.  Full
descriptions can be found in the Course Descriptions section
(section~\ref{sec:coursedesc}, page~\pageref{sec:coursedesc}).

\begin{itemlist}
\item CS 2110, Software Development Methods
\item CS 2102, Discrete Math
\item CS 2150, Program and Data Representation
\item CS 3330, Computer Architecture
\item CS 4102, Algorithms
\item One of the following four courses:
  \begin{itemlist}
  \item CS 3102, Theory of Computation
  \item CS 3240, Advanced Software Development Techniques
  \item CS 4414, Operating Systems
  \item CS 4610, Programming Languages 
  \end{itemlist}
  Additional courses from this list can be taken as CS electives.
\end{itemlist}

Note that any CS1 class, either CS 111x, Introduction to Programming,
or CS 1120, Introduction to Computing: Language, Logic, and Machines,
is required to enroll in CS 2110.

This list changed in Summer 2019. Students who declared their major before then should consult their Academic Requirements report in SIS to verify that their plans fit the requirements of the major as it was when they declared it.

\subsection{CS Electives}
\label{sec:bacs-cselectives}

Four computing intensive electives are to be selected from a list of
approved courses. The list of approved courses includes all 
current Computer Science courses at 3000-level or 4000-level. Additional 
courses that may be jointly offered by CLAS and CS
departments will be added to the list of approved computing electives
based on approval by the BA committee.

There are a few restrictions on which upper-level CS courses can count
as a CS elective:

\begin{itemlist}
\item CS 4971 (Capstone Practicum II) does not count as a CS elective,
as it is part of the BS CS capstone requirement.  Note that its
prerequisite (CS 4970, Capstone Practicum I) {\em does} count as a CS
elective.
\item CS 4980 (Capstone Research) does not count as a CS elective, as
it is part of the BS CS capstone requirement.
\item CS 4993 (Independent Study) credits can only count for at most 1
CS elective (i.e., 3 credits).
\item CS 4998 (Distinguished BA Majors Research) is a separate
  requirement for the DMP (see section~\ref{sec:badmp},
  page~\pageref{sec:badmp}), and thus does not count as a CS elective.
\end{itemlist}

\subsection{Integration Electives}
\label{sec:bacs-integrationelectives}

Four courses selected with the approval of the student's advisor from
the list of computing-related courses approved by the BA CS
committee. These courses are offered by departments other than
Computer Science, and should either provide fundamental computing
depth and background or explore applications of computing to arts and
sciences fields. 

This is a list of the courses that are generally approved as
integration electives. This list is not meant to be exhaustive: if you
find a course that is not on the list that appears to satisfy the
goals of an integration elective, discuss with your advisor or the BA
Program Director if it should count as an integration elective for
you.

Some of these courses are not offered regularly, and some courses may
have prerequisites. The list of integration electives may change
slightly from year to year.  You can always check the current list of
integration electives on SIS.  The list below is according to SIS as
of September 2019.

\paragraph{American Studies}
\begin{itemlist}
\item AMST 3463: Language \& New Media
\end{itemlist}

\paragraph{Anthropology}
\begin{itemlist}
\item ANTH 3171: Culture of Cyberspace: Digital Fluency for an Internet-Enabled Society
\item ANTH 3490: Language and Thought
\end{itemlist}


\paragraph{Studio Art}
\begin{itemlist}
\item ARTS 2220: Introduction to New Media I
\item ARTS 2222: Introduction to New Media II
\item ARTS 3220: Intermediate New Media 
\item ARTS 3222: Intermediate New Media II
\item ARTS 4220: Advanced New Media I
\item ARTS 4222: Advanced New Media II
\end{itemlist}

\paragraph{Biology}
\begin{itemlist}
\item BIOL 4230: Bioinformatics and Functional Genomics
\end{itemlist}


\paragraph{Chemistry}
\begin{itemlist}
\item CHEM 3240: Coding in Matlab/Mathematica with Applications
\end{itemlist}


\paragraph{Drama}
\begin{itemlist}
\item DRAM 2110: Lighting Technology
\item DRAM 2210: Scenic Technology
\item DRAM 2240: Digital Design: Re-making and Re-imagining
\item DRAM 3825: Media Design Studio
\end{itemlist}


\paragraph{Economics}
\begin{itemlist}
\item ECON 3720: Econometric Methods 
\item ECON 4010 Game Theory
\item ECON 4020: Auction Theory and Practice
\item ECON 4720: Econometric Methods
\end{itemlist}

\paragraph{English Writing \& Rhetoric}
\begin{itemlist}
\item ENWR 2640: Composing Digital Stories and Essays
\item ENWR 3640: Writing with Sound
\end{itemlist}

\paragraph{Environmental Science}
\begin{itemlist}
\item EVSC 3020: GIS Methods
\item EVSC 4010: Introduction to Remote Sensing
\item EVSC 4070: Advanced GIS
\end{itemlist}


\paragraph{History}
\begin{itemlist}
\item HIST 2212: Maps in World History
\item HIUS 3162: Digitizing America
\end{itemlist}


\paragraph{Linguistics}
\begin{itemlist}
\item LING 3400: Structure of English
\item LNGS 3250: Intro to Linguistic Theory
\end{itemlist}

\paragraph{Mathematics}
\begin{itemlist}
\item MATH 3100: Intro Mathematical Probability
\item MATH 3120: Intro Mathematical Statistics
\item MATH 3315: Advanced Linear Algebra and Differential Equations
\item MATH 3350: Applied Linear Algebra
\item MATH 3351: Elementary Linear Algebra
\item MATH 4080: Operations Research
\item MATH 4300: Elementary Numerical Analysis
\end{itemlist}

\paragraph{Media Studies}
\begin{itemlist}
\item MDST 2010: Introduction to Digital Media
\item MDST 3050: History of Media
\item MDST 3102: Copyright, Commerce and Culture
\item MDST 3404: Democratic Politics in the New Media Environment
\item MDST 3500: Comparative Histories of the Internet
\item MDST 3701: New Media Culture
\item MDST 3702: Computers and Languages
\item MDST 3703: Digital Liberal Arts
\item MDST 3704: Games and Play
\item MDST 3750: Money, Media and Technology
\item MDST 3751: Values, Value, and Valuation
\item MDST 3755: Social Media and Society
\item MDST 4101: Privacy \& Surveillance
\item MDST 4700: Theory of New Media
\item MDST 4803: Computational Media
\end{itemlist}


\paragraph{Music}
\begin{itemlist}
\item MUSI 2350: Technosonics: Digital Music \& Sound Art Composition
\item MUSI 2390: Intro to Music \& Computers
\item MUSI 3390: Intro to Music \& Computers
\item MUSI 4535: Interactive Media
\item MUSI 4540: Computer Sound Generation
\item MUSI 4543: Sound Studio
\item MUSI 4545: Computer Applications in Music
\item MUSI 4610: Sound Synthesis
\item MUSI 4600: Performance with Computers
\end{itemlist}

\paragraph{Philosophy}
\begin{itemlist}
\item PHIL 1410: Forms of Reasoning
\item PHIL 1510: Ethics of Computing
\item PHIL 2330: Computers, Minds and Brains
\item PHIL 2340: The Computational Age
\item PHIL 2420: Introduction to Symbolic Logic
\end{itemlist}


\paragraph{Physics}
\begin{itemlist}
\item PHYS 2660: Fundamentals Scientific Computing
\end{itemlist}


\paragraph{Psychology}
\begin{itemlist}
\item PSYC 2150: Introduction to Cognition
\item PSYC 2200: Survey of the Neural Basis of Behavior
\item PSYC 2300: Introduction to Perception
\item PSYC 4110: Psycholinguistics
\item PSYC 4111: Language Development \& Disorders
\item PSYC 4125: Psychology of Language
\item PSYC 4150: Cognitive Processes
\item PSYC 4200: Neural Mechanisms of Behavior
\item PSYC 4300: Theories of Perception
\item PSYC 4400: Approaches to Quantitative Methods in Psychology
\item PSYC 4682: Mobile Technology in Mental Health Research
\end{itemlist}


\paragraph{Statistics}
\begin{itemlist}
\item STAT 1100: Chance: Intro to Statistics
\item STAT 1120: Intro to Statistics
\item STAT 2020: Statistics for Biologists
\item STAT 2120: Intro to Statistical Analysis
\item STAT 3010: Statist Computing \& Graphics
\item STAT 3080: From Data to Knowledge
\item STAT 3120: Intro to Mathematical Statistics
\item STAT 3220: Introduction to Regression Analysis
\item STAT 3240: Coding in Matlab/Mathematica with Applications
\item STAT 4220: Applied Analytics for Business
\item STAT 4260: Databases (only if CS 4750 has not been taken)
\item STAT 4630: Statistical Machine Learning
\end{itemlist}



\paragraph{Using other courses.}  If a student would like to use a
course not on the above list as an integration elective, they should
first contact their academic advisor.  Their advisor can work with the
student to come up with a good argument as to why the course should
qualify, and once the advisor approves it, send it to the BA CS
Director at \bacsdirectoremail.  Alternatively, if the advisor
prefers, s/he can just send the student to BA CS director to get
approval for a requirement exception.  This will require a SIS
exception to be entered for the student;
\ifthenelse{\boolean{ba-int-list-sis}}{see the full CS undergraduate handbook}{see section~\ref{sec:sisexceptions} (page~\pageref{sec:sisexceptions})}
for the manual SIS exception process.



\mysection{Miscellaneous Information}

%\subsection{Declaring the Major}

%Before declaring the computer science major, students should have
%taken one introductory computer science course (CS 111x (101, 101E,
%101x), Introduction to Programming, or CS 1120, Introduction to
%Computing: Language, Logic, and Machines) with a grade of C+ or
%better, or have comparable experience. Students may be permitted to
%declare the major while they are currently taking the introductory
%course.
%
%To declare the major:
%
%\begin{numlist}
%
%\item Satisfy the major prerequisite by taking one of the introductory
%  computer science courses. CS 1120 is the recommended course
%  for most BA CS majors, but the other introductory courses (CS 1110
%  CS1111, and CS 1112) can also be used to satisfy the
%  prerequisite. You may declare the major before completing the course
%  as long as you are on track to complete the course successfully. If
%  you believe you have comparable experience in some other way, you
%  may also be able to declare the major.
%
%\item Pick up a Major Declaration Form from the Dean's office, and
%  fill out the top half.
%
%\item Arrange to meet with \badup\ (\badupemail), Director of the
%  Undergraduate Program (DUP). You can email \baduppronountwo to arrange
%  a meeting time, or drop by during \baduppronoun office hours.
%
%\end{numlist}


\subsection{Distinguished Majors Program}
\label{sec:badmp}

%Bachelor of Arts Computer Science 
BA CS majors who have completed 18 credit hours toward their major
and who have a cumulative GPA of 3.4 or better may apply to the
Distinguished Majors Program. Students who are accepted must complete
a thesis based on two semesters of empirical or theoretical
research. The Distinguished Majors Program features opportunities for
students and advisors to collaborate on creative research; it is not a
lock-step thesis program with strict content requirements. Upon
successful completion of the program, students will likely be
recommended for a baccalaureate award of Distinction, High
Distinction, or Highest Distinction.

Students applying to the DMP must have a minimum cumulative GPA of 3.4
and have completed 18 credit hours toward their Computer Science major
by the end of the semester in which they apply. These 18 credit hours
can can come from any course used to fulfill the core course
requirement (section~\ref{sec:bacs-corecourses},
page~\pageref{sec:bacs-corecourses}), CS electives
(section~\ref{sec:bacs-cselectives},
page~\pageref{sec:bacs-cselectives}), or the integration electives
(section~\ref{sec:bacs-integrationelectives},
page~\pageref{sec:bacs-integrationelectives}).
%Interdisciplinary Major in Computer Science Curriculum.  curriculum.
Exceptions to the 18 credit hours rule may be granted at the
discretion of the Distinguished Majors Program Director.

In addition to the normal requirements for the computer science major,
they must register for two semesters of supervised research (CS 4998
for 3 credits each semester). Students may apply to the DMP before
completing this supervised research, but students must complete the
supervised research to complete the DMP. Based on their independent
research, students must complete, to the satisfaction of their advisor
and the Distinguished Major Program Director, a project at least one
month prior to graduation.

Please note: The CS 4998 DMP credits do not apply toward the credit
hours required for the major. That is, they cannot be used to fulfill
any requirement listed on the BA CS curriculum.

For more information on the DMP, see
online\myurlFormatted{\bacsURLDMP}.
You may also contact the BA CS director at \bacsdirectoremail, who is
in charge of the BA DMP program.

\subsection{Double majors in CLAS}

From the CLAS web site on
majors\myurl{http://college.artsandsciences.virginia.edu/majortypes}
regarding double majors:

\begin{quotation}
\noindent You may major in two subjects, in which case the application
for a degree must be approved by both departments or inter\-departmental
programs. Students who double major must submit at least 18 credits in
each major; these credits may not be duplicated in the other
major. There is no triple major.
\end{quotation}

However, you should be aware of the application process described in
section~\ref{bacsapplicationprocess}
(page~\pageref{bacsapplicationprocess}).

\subsubsection{Timing of Double Majors}

The UVA administration dislikes students who have finished the
requirements for one degree remaining a student for an additional
semester to add a minor or second major. In some cases they may even
force you to graduate early instead of allowing you to remain to
complete your other degree plans. If you plan to get a double major,
please plan to complete both majors' requirements in the same
semester.


\mysection{Course Requirements Flowchart}

\begin{figure*}[h!]
\label{fig:barequirementsflowchart}
\ifthenelse{\boolean{lettersize}}{\section*{Course Requirements Flowchart}}{}
{\em (Updated April 2015)}
\begin{center}
\ifthenelse{\boolean{useflowchartimages}}{
\ifthenelse{\boolean{lettersize}}
{\epsfig{figure=diagrams/ba-cs.png,width=5.5in}}
{\epsfig{figure=diagrams/ba-cs.png,width=4.25in}}
}{
\ifthenelse{\boolean{lettersize}}
{\epsfig{figure=diagrams/ba-cs.pdf,width=5.5in}}
{\epsfig{figure=diagrams/ba-cs.pdf,width=4.25in}}
}
\end{center}
\end{figure*}

\ifthenelse{\boolean{lettersize}}{See
  figure~\ref{fig:barequirementsflowchart} on page~\pageref{fig:barequirementsflowchart}.}{\noindent}
Note that CS 2102 requires {\em either} CS 111x or CS 1120 as a
prerequisite.
