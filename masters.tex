\noindent Online: \csmsURL

\mysection{Disclaimer}

The information contained on this handbook is for informational
purposes only. The Undergraduate Record and Graduate Record represent
the official repository for academic program requirements. These
publications may be found online\myurl{https://www2.virginia.edu/registrar/records.html}.

The record is updated every year; you want the version under which you were admitted,
as noted in SIS under the Academics section as your ``Admit Term.''

- Fall 2019 and Spring 2020: \myurl{http://records.ureg.virginia.edu/preview_program.php?catoid=48&poid=6159}
- Fall 2018 and Spring 2019: \myurl{http://records.ureg.virginia.edu/preview_program.php?catoid=46&poid=5856}
- Fall 2017 and Spring 2018: \myurl{http://records.ureg.virginia.edu/preview_program.php?catoid=44&poid=5470}



% From a conversation with Jack Davidson on 8-20-13, the warning is
% being removed, as the department is now accepting masters students

%{\Large\em Warning: The department typically does not admit more than
%  one or two UVa students per year into the five-year Master's program
%  described in this chapter. Thus students may not be able to complete
%  this program and should discuss it with an advisor when planning.}

% this information is via Wes who spoke with Kathy Thornton in January 2011

\mysection{Introduction}

There are multiple ways to pursue graduate degrees. If you uncertain about
the benefits or general meaning of a graduate degree (e.g., ``what
\emph{is} a Master's degree and why might I want one?'') speak to your
advisor or attend various graduate school information sessions, such as
those produced by the local Association for Computing Machinery Chapter
(see Section~\ref{sec-student-groups}). 

Master's degrees are the most common graduate degrees and typically
requires three to four semesters to complete. Ph.D. students typically
initially complete a Master's degree and then complete the Ph.D. degree
four years after that. An additional option is available to UVa
undergraduates: any credits not used to satisfy UVA undergraduate
requirements can be applied to UVa graduate requirements, allowing for the
completion of both a Bachelor's degree and also a Master's degree in a
total of five years (instead of the usual six). 

The department maintains graduate program information
online\myurlFormatted{\csmsURL}; that
web site contains more complete information than this chapter.  This
section pertains solely to obtaining a Master's in five years. This type
of Masters degree is usually one of the ``terminal'' Masters degrees
discussed in the graduate handbook, such as the ``terminal coursework-based
MCS degree''. Other type of Masters degrees are not discussed here.

Students seeking a Ph.D.\ are often better served going to a different
school for Ph.D.\ graduate work. Every school has biases and ways of doing
things, and spending an entire academic career at one institution precludes
one from seeing other approaches to CS research. A student seeing a 
Ph.D.\ should consider applying directly to another school's graduate
program or completing a Master's degree here and then applying to another
school's Ph.D.\ program. 

In recent years UVa CS has made an explicit push to grow the Master's
program and our five-year Bachelor's and Master's approach is likely to
be a good fit for students who want to obtain a strong Master's degree 
before entering the workforce. 

Students that are aiming for a five-year Master's must still follow the
same rules and guidelines for all Computer Science and SEAS graduate
students. That is, even though you are ``already'' a UVa CS student, you
must officially apply to the UVa CS graduate program at the usual time
(typically by December 15th in your penultimate undergraduate semester). 
These rules and guidelines can be found in the graduate handbook (see
above), and
online.\myurl{http://www.seas.virginia.edu/admissions/graduate.php}

We would like to stress again that this chapter focuses solely on
earning a Master's degree in five years, where the Bachelors also was
earned at UVa.

\subsection{When to Apply}

Although you are earning both degrees in five years, there is still will
be a formal switch between undergraduate and graduate student status.
See also the graduate applications web
page\myurl{http://www.seas.virginia.edu/admissions/graduate.php}.

A local undergraduate student applies to the UVa graduate program in CS
like any other prospective graduate student, but mentions in the
application packet (specifically, in the statement of purpose) that 
this is an application for the five-year combined Bachelor's and Master's
program. Students seeking only a Master's degree (and not a follow-on
Ph.D.\ at UVa or another school) should also indicate explicitly that they
are applying for a terminal Master's degree. 

Note that in the application process, you are \emph{not} considered a
transfer student, even if you already have taken some graduate courses at
UVa.

The easiest time to apply is in the fall of your fourth year (i.e., during
your seventh semester).  This would allow you to finish up eight full
semesters as an undergraduate, and then have two full semesters (plus
summers, potentially) as a graduate student. Applications to the Computer 
Science graduate program are typically due on December
15th\myurl{http://www.cs.virginia.edu/admissions/grad/}. 

However, nothing requires that you apply at that time.  You can apply
in the spring of your third year (i.e., sixth semester) or even earlier.
Students who decide to apply earlier (sixth semester) may graduate
early in 3.5 years (i.e., seven semesters). Such students graduate one
semester early from undergraduate, then enroll in their eighth semester
as a Master's student.

The benefit of applying a semester early is that one can have an
additional semester to work exclusively on graduate courses with full
institutional support (e.g., an office in the CS building). This must be
balanced with the concern of completing their undergraduate degree in seven
semesters.

Note that if you are a SEAS student and are graduating one semester
early, you must still write a senior thesis, and take STS 4500 and STS
4600.

One can apply earlier as well, such as in the fall of one's third year
(i.e. in your fifth semester).  This would mean that one would complete
the undergraduate degree in three years (following the same time line as
completion of the undergraduate degree in eight semesters, but accelerated
by one year), and complete the Master's in the fourth year.

\subsection{Degrees Offered}

The Department of Computer Science offers two different Master's
degrees.  The first is a Master's of Computer Science (MCS), and the
second is a Master's of Science in Computer Science (MS). Both may be
obtained in five years, although most students will opt for the MCS.

From the perspective of employers, the two degrees are, for the most part,
equivalent.  The primary difference is that a MS requires a full
Master's thesis, with a complete faculty committee that looks for a
significant amount of work to have been accomplished.  The MCS degree
requirements can be satisfied entirely via standard coursework or via
a three-credit project, and that is judged only by the student's advisor.
The MCS degree is a good fit for a terminal Master's degree. See the
UVa CS graduate handbook for information about the various combinations
of Master's degrees (e.g., terminal or non-terminal; thesis, coursework or
project). 

\mysection{Curriculum}

The full curriculum for a Master's degree is listed in the graduate
handbook.\myurl{https://github.com/uva-cs/grad-handbook/}
This section is only intended as a summary.

A coursework-based Master's degree typically involves ten classes (thirty
credits). One or two of these courses often corresponds to an MCS project
or an MS thesis course.  Graduate students often take three or four courses
for each of three semesters, completing the degree in two years. A UVa
undergraduate student with three or four graduate classes beyond the usual
undergraduate requirements (e.g., not used to satisfy any undergraduate
major requirements or general rules like the 120 credit requirement; see
below) can apply those credits toward a UVa CS graduate degree, allowing
for completion in fewer semesters. 

Any course that counts toward the graduation requirements for your
undergraduate degree may {\em\bf NOT} count toward the graduation
requirements for your Master's~-- even if it is as an unrestricted
elective.  Thus, if you want to take graduate classes as an undergraduate,
and you want them to count toward your Master's graduation requirements,
you must ensure that you take enough classes so that you could have
graduated {\em\bf without} those graduate class(es).  Thus, one must
carefully work out which courses will count for which degree. 

Terminal Master's students do not need to take (or pass) a qualification
exam, as that exam is required for the Ph.D.\ degree only.  If you decide
to later transfer into the Ph.D.\ degree, then you will need to take (or
have taken) the qualification exam.

One can certainly take more than five years to complete the Master's~--
you are paying tuition, after all~-- but typically students aim to
complete their Master's after five full years. Students will sometimes use
the summer (either before or after their fifth year) to complete some
Master's requirements.  

\mysection{Miscellaneous Information}

Generally, terminal Master's students are not funded.  Thus, students will
pay tuition (and room/board, as appropriate).  The costs are analogous to
undergraduate rates: lower for in-state residents, and higher for
out-of-state residents.

The ``easiest''~-- and most typical~-- path to a five-year Master's is to
apply in seventh semester, and already have some graduate classes that are
\emph{not} counting toward your undergraduate degree. You will have had to
have talked to your undergraduate advisor about your plans. For example,
if you eventually desire a Ph.D., you should discuss who you might work
with for a Master's project or thesis. By contrast, if you desire a
terminal Master's degree, you can obtain it entirely via coursework. 
You would then complete the degree in one full additional year (summer,
fall, and spring), aiming for a May graduation date.

For any addition questions, please do not hesitate to contact the chair of
the graduate program committee (currently Westley Weimer) or the chair of
the graduate admissions committee (currently Marty Humphrey).
