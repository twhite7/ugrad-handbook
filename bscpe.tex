\noindent Online: \bscpeURL

\mysection{Disclaimer}

The information contained on this handbook is for informational
purposes only. The Undergraduate Record and Graduate Record represent
the official repository for academic program requirements. These
publications may be found online\myurl{http://records.ureg.virginia.edu/index.php}$^,$\myurl{http://records.ureg.virginia.edu/preview_program.php?catoid=42&poid=4960}.


\mysection{Introduction}

Computer Engineering is an exciting field that spans topics across
electrical engineering and computer science.  Students learn and
practice the design and analysis of computer systems, including both
hardware and software aspects and their integration. Careers in
Computer Engineering (CpE) are as wide and varied as computer systems
themselves, which range from embedded computer systems found in
consumer products or medical devices, to control systems for
automobiles, aircraft, and trains, to more wide-ranging applications
in entertainment, telecommunications, financial transactions, and
information systems.

%\begin{quotation}
%Computer Engineering gives you a great working knowledge
%  and balance in both CS \& ECE. With the freedom to choose electives
%  in either department, you are in full control of your educational
%  experience and how you wish to enhance your knowledge. -- Kevin Chang,
%  08
%\end{quotation}

\subsection{Program Objectives}

Graduates of the Computer Engineering program at the
University of Virginia utilize their academic preparation
to become successful practitioners and innovators in
computer engineering and other fields. They analyze,
design and implement creative solutions to problems
with computer hardware, software, systems and
applications. They contribute effectively as team
members, communicate clearly and interact responsibly
with colleagues, clients, employers and society.

Faculty from the Computer Science and Electrical \& Computer
Engineering departments jointly administer the CpE undergraduate
degree program at the University of Virginia.

The Computer Engineering program does not offer a minor.

%\begin{quotation}
%It's the future.  Everything is digitized and computer
%  engineering allows you to keep up with changing technology. It's a
%  complex field with many great opportunities for advancement.  -- Rob
%  Yip, '08
%\end{quotation}

\mysection{Application Process}
\label{bscpeapplicationprocess}

The application process for the Computer Engineering degree are the
exact same as with the BS Computer Science degree (section
\ref{bscsapplicationprocess}, page~\pageref{bscsapplicationprocess}),
and thus it is not repeated here.


\mysection{General Curriculum Details} % called ``Disciplines'' in the original

The curriculum has been carefully designed to ensure that students
obtain an excellent background in both Computer Science and Electrical
Engineering, providing breadth across these disciplines as well as
depth in at least one. All Computer Engineering students work through
an extended sequence of introductory, intermediate and advanced
courses:

\begin{itemlist}
\item CS 1110 Introduction to Computer Science
\item CS 2110 Software Development Methods
\item CS 2102 Discrete Math
\item ECE 2630 Introductory Circuit Analysis
\item ECE 2660 Electronics I
\item CS 2150 Program and Data Representation
\item ECE/CS 2330 Digital Logic Design
\item ECE 3750 Signals \& Systems I
\item CS 3240 Advanced Software Development
\item CS 3430 Introduction to Embedded Computer Systems
\item CS 4414 Operating Systems
\item ECE 4435 Computer Architecture \& Design
\item ECE 4440 Embedded Systems Design
\item CS/ECE 4457 Computer Networks
\end{itemlist}

%\paragraph{Please Note:} Course numbers changed in 2009. The old 
%course numbers are shown in parentheses.

\noindent In addition to providing breadth across the two areas,
this core of the Computer Engineering program provides
depth in the following areas:

%\ifthenelse{\boolean{lettersize}}{\begin{multicols}{2}}{}

\paragraph{Circuits}
\begin{itemlist}
\item ECE 2630: Introductory Circuit Analysis
\item ECE 2660: Electronics I
\end{itemlist}

\paragraph{Software Engineering}
\begin{itemlist}
\item CS 2110: Software Development Methods
\item CS 3240: Advanced Software Development
\end{itemlist}

\paragraph{Digital Logic}
\begin{itemlist}
\item ECE/CS 2330: Digital Logic Design
\item CS 2102: Discrete Math
\end{itemlist}

\paragraph{Computer Systems}
\begin{itemlist}
\item CS 2150: Program and Data Representation
%\item CS 3330: Computer Architecture (see \S\ref{embedded})
\item CS 3430 Introduction to Embedded Computer Systems
\item CS 4414: Operating Systems
\item ECE 4435: Computer Architecture \& Design
\item ECE 4436: Embedded Systems Design
\item CS/ECE 4457: Computer Networks
\end{itemlist}

%\ifthenelse{\boolean{lettersize}}{\end{multicols}}{}

\subsection{Grade Requirement}
To complete their program of study, computer engineering majors must
achieve a C average or better in their Computer Science and Electrical
Engineering courses.


\mysection{Curriculum} % called ``Disciplines'' in the original

Below is the recommended course of study for the bachelor's degree. If
one has already completed some of these classes (through AP credit,
for example), then your course of study would deviate from what is
shown below~-- consult your academic advisor for details.

\vspace{0.15in}

%\ifthenelse{\boolean{lettersize}}{\begin{multicols}{2}}{}

\samplescheduletableheader
\und{First semester} & & \und{15} \\
APMA 1110 & Single Variable Calculus & 4 \\
CHEM 1610 & Intro Chemistry I for Engineers & 3 \\
CHEM 1611 & Intro Chem.\ I for Engineers Lab & 1 \\
ENGR 1620 & Introduction to Engineering & 3 \\
ENGR 1621 & Intro.\ to Engineering Lab & 1 \\
STS 1500 & Science, Tech.,\ \& Contemporary Issues & 3 \\
\\
\end{tabular}

\samplescheduletableheader
\und{Second semester} & & \und{17} \\
SCI elective & Science elective$^2$ & 3 \\
HSS elective & HSS elective$^1$ & 3 \\
APMA 2120 & Multivariate Calculus & 4 \\
PHYS 1425 & Physics I: Mechanics, Thermo. & 3 \\
PHYS 1429 & Physics I Workshop & 1 \\
CS 1110 {\em or} 1111 {\em or} 1112 {\em or} 1113 & Introduction to Programming & 3 \\
\\
\end{tabular}

\samplescheduletableheader
\und{Third semester} & & \und{17} \\
HSS elective & HSS elective$^1$ & 3 \\
APMA 2130 & Ordinary Differential Eq. & 4 \\
CS 2110 & Software Develop.\ Methods & 3 \\
CS 2102 & Discrete Mathematics & 3 \\
ECE 2630 & ECE Fundamentals I & 4 \\
\\
\end{tabular}

\samplescheduletableheader
\und{Fourth semester} & & \und{16} \\
STS 2xxx/3xxx & STS 2xxx/3xxx elective & 3 \\
UE elective & Unrestricted elective$^3$ & 3 \\
CS 2150 & Prog.\ \& Data Representation & 3 \\
ECE 2660 & ECE Fundamentals II & 4 \\
CS/ECE 2330 & Digital Logic Design & 3 \\
\\
\end{tabular}

\samplescheduletableheader
\und{Fifth semester} & & \und{15} \\
CS/ECE & CS/ECE elective$^4$ & 3 \\
ECE 3430 & Intro to Embed.\ Systems & 4 \\
ECE 3750 & ECE Fundamentals III & 4 \\
PHYS 2415 & General Physics II: E\&M \& Lab & 3 \\
PHYS 2419 & General Physics II Workshop & 1 \\ 
\\
\end{tabular}

\samplescheduletableheader
\und{Sixth semester} & & \und{18} \\
CS/ECE & CS/ECE elective$^4$ & 3 \\
UE elective & Unrestricted elective$^3$ & 3 \\
ECE 4550 & Applied Research \& Design Labd & 1.5 \\
APMA 3100 & Probability & 3 \\
CS 3240 & Advanced Software Develop. & 3 \\
ECE 4435 & Computer Arch.\ \& Design & 4.5 \\
\\
\end{tabular}

\samplescheduletableheader
\und{Seventh semester} & & \und{15} \\
CS/ECE & CS/ECE elective$^4$ & 3 \\
UE elective & Unrestricted elective$^3$ & 3 \\
ECE 4440 & Embedded Systems Design & 3 \\
STS 4500 & STS \& Engineering Practice & 3 \\
CS/ECE 4457 & Computer Networks & 3 \\
\\
\end{tabular}

\samplescheduletableheader
\und{Eighth semester} & & \und{15} \\
CS/ECE & CS/ECE elective$^4$ & 3 \\
UE elective & Unrestricted elective$^3$ & 3 \\
UE elective & Unrestricted elective$^3$ & 3 \\
CS 4414 & Operating Systems & 3 \\
STS 4600 & Engineer, Ethics, \& Prof.\ Society & 3 \\
\end{tabular}

%\ifthenelse{\boolean{lettersize}}{\end{multicols}}{}

\paragraph{Footnotes:}

\label{sec:cpeunrestricted2017}

\begin{numlist}
\item Chosen from the approved list available in A122 Thornton Hall.
\item Chosen from: among BIOL 2100, 2200; CHEM 1620; ECE 2066; MSE 2090; and PHYS 2620.
\item Unrestricted electives may be chosen from any graded course in the University except mathematics courses below MATH 1310 including STAT 1100 and STAT 1120 and courses that substantially duplicate any others offered for the degree including PHYS 2010, 2020; CS 1010, 1020; or any introductory programming course. Students in doubt as to what is acceptable to satisfy a degree requirement should get the approval of their advisor and the dean’s office, located A122 Thornton Hall. APMA 1090 counts as three-credit unrestricted elective.
\item Chosen from CS/ECE courses at the 3000 level or higher. Two CS/ECE electives must be 4000 level or above.
\end{numlist}



\mysection{Miscellaneous Information}

There are three CS capstone courses: CS 4970 (Capstone Practicum I), 
CS 4971 (Capstone Practicum II), and CS 4980 (Capstone Research).
Only the first one (CS 4970) counts as a CS/ECE elective for the
Computer Engineering degree; the other two can only count as an
unrestricted elective.

Please refer to the Undergraduate
Record\myurl{http://records.ureg.virginia.edu/} for detailed
information about SEAS Academic Rules and Regulations including HSS
electives. In addition, guidelines such as course load, academic
probation and academic suspension can also be found in the Record.

%The Registrar web site provides a Course Renumbering Crosswalk to
%assist with the transition from 3 to 4 digit course
%numbers\myurl{http://www.virginia.edu/registrar/search.php}.

\subsection{Double BS CS \& BS CpE majors}
\label{bscscpedoublemajors}

Due to substantial overlap with CS 3330 (Computer Architecture), ECE
4435 (Computer Architecture \& Design) can NOT count as a CS elective.
However, double majors may have ECE 4435 count as their CS 3330
requirement, although this will require a manual SIS exception to do
so; see section~\ref{sec:sisexceptions}
(page~\pageref{sec:sisexceptions}) for the SIS exception process.

ECE 4440 (Embedded Systems Design) can count as a CS elective, but
this also requires a SIS exception to be entered~-- see
section~\ref{sec:sisexceptions} (page~\pageref{sec:sisexceptions}).
Note that even though ECE 4440 is a 4.5 credit course, it can only
count as one CS elective (i.e., only 3 credits).

Computer engineering majors are allowed to take ECE 3209
(Electromagnetic Fields) in place of PHYS 2415/2419.  While this
option is only open to computer engineering majors, it also applies to
dual CS/CpE majors as well.

The BS CpE web site\myurl{http://www.cpe.virginia.edu/ugrads/} has a
sample course schedule for double majors.

%\clearpage
%\mysection{Course Requirements Flowchart}
%\begin{figure}[h!]
%\epsfig{figure=flowcharts/cpe-flowchart.png,width=4.5in}
%\end{figure}

\ifthenelse{\boolean{lettersize}}{}{\clearpage}

\begin{figure*}[h!]
\mysection{Course Requirements Flowchart}
{\em (Updated October 2013)}
\begin{center}
\ifthenelse{\boolean{useflowchartimages}}{
\ifthenelse{\boolean{lettersize}}
{\epsfig{figure=diagrams/bs-cpe.png,width=6in}}
{\epsfig{figure=diagrams/bs-cpe.png,width=4.5in}}
}{
\ifthenelse{\boolean{lettersize}}
{\epsfig{figure=diagrams/bs-cpe.pdf,width=6in}}
{\epsfig{figure=diagrams/bs-cpe.pdf,width=4.5in}}
}
\end{center}
\end{figure*}
