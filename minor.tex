\noindent Online: \csminorURL

%{\Large\em Warning: Due to limited resources, the department is not
%accepting any minor applications from non-SEAS students, and only a
%limited number of applications from SEAS students are accepted.}

\mysection{Introduction}

The Department of Computer Science provides a minor program for
qualified students. The courses in the minor program provide a solid
foundation in computer science. The minor program is a six course,
eighteen credit curriculum. The curriculum consists of the four
required courses and two elective courses. Full course descriptions
are at the end of this document in section~\ref{sec:coursedesc}
(page~\pageref{sec:coursedesc}).

%In the past, there were separate requirements for the minor for SEAS
%students and non-SEAS students.  These requirements have been
%streamlined into a single set of requirements for everybody.
 
\mysection{Application Process}
\label{minorapplicationprocess}

The department can only allow a limited number of SEAS students to
declare a minor in Computer Science due to a rapidly growing demand
for computing courses.  Unfortunately, at this time the University is
only able to accept SEAS students as CS minors. This situation will be
re-evaluated before each year's application deadline, and if a change
is made the CS minor page\myurlFormatted{\csminorURL} will be
updated to reflect the change. The CS department continues to work
with the University to obtain resources that will allow more students
to declare the Computer Science minor.

Students wishing to declare the minor normally apply in the spring of
their first or second year. Applications from third and fourth year
students will be considered only if there are still available spaces
that were not taken earlier. The normal deadline is March 1. The
application form can be found
online\myurl{http://www.cs.virginia.edu/~horton/cs-minor-apply}.  All
applicants will be notified if they have been accepted as a CS minor
by April 1.

{\bf BS in Computer Engineering majors:} When the CpE program was created,
it was decided by the two departments that CpE students could not
declare the minor in CS. Because the CpE combines CS and EE, graduates
with this degree will automatically have the equivalent of the minor
in CS.

\mysection{Curriculum}

All SEAS (School of Engineering and Applied Science) students are
required to take (or place out of) CS 1110, as part of the SEAS
first-year curriculum. This course is also the first required course
for the minor.

The following are the first four courses required for the minor.

\begin{itemlist} 
\item CS 1110, CS 1111, or CS 1112: Introduction
  to Computer Science
\item CS 2110: Software Development Methods
\item CS 2102: Discrete Mathematics
\item CS 2150: Program and Data Representation
\end{itemlist}

Note that CS 1120 (From Ada and Euclid to Quantum Computing and the
World Wide Web) can replace the CS 111x requirement.  However, all
SEAS students are required to take a CS 111x course regardless, so
courses, so taking CS 1120 would not help at all.

Note that if you place out of CS 1110 via the placement exam,
you still have to take 6 CS courses; if you receive course credit for
it via the AP exam or transfer credit, then you need not substitute a
course in its place.

Furthermore, two additional computer science electives are
required. The elective courses must be computer science courses at the
3000 level or above. The only restriction on elective courses is a
limit to how many independent study courses one can count toward a
minor~-- contact the minor advisor for details at
\csminoradvisoremail.

Computer science courses typically build upon each other. In
particular, CS 1110 is a prerequisite of both CS 2110 and
CS 2102. CS 2110 and CS 2102 are both prerequisites
of CS 2150. In addition, CS 2150 is a prerequisite for
almost all of the computer science electives. The Department of
Computer Science also requires that its courses be passed at a certain
level (typically a C- or higher) in order to take successive
courses. Be aware that the department strictly enforces its
prerequisite policy.


\mysection{Miscellaneous Information}
The UVA administration dislikes students who have finished the
requirements for one degree remaining a student for an additional
semester to add a minor or second major. In some cases they may even
force you to graduate early instead of allowing you to remain to
complete your other degree plans. If you plan to get a CS minor,
please plan to complete its requirements no later than the semester
in which you finish the requirements of your major.

%\subsection{Declaring the Minor}
%
%To declare the minor:
%
%\begin{numlist}
%
%\item A student should have completed CS 1110 or 1120, CS 2110, and CS
%  2102. Furthermore, the student should have completed, or at least be
%  enrolled in, CS 2150.
%
%\item Complete the minor declaration form, which is available in the
%  Computer Science department's front office (Rice Hall, room 527).
%  The form has the title, ``School of Engineering and Applied Science,
%  Minor Declaration''~-- this is the form for everybody; the SEAS
%  school title at the top is because the CS department is in SEAS.
%
%\item Meet with the CS department's minor advisor(s), currently
%  \csminoradvisor\ (\csminoradvisoremail).  Bring a the minor
%  declaration form and your transcript (unofficial is fine).
%
%\item Assuming the form is approved, it will be processed by the
%  department.
%
%\end{numlist}
