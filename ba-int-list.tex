Four courses selected with the approval of the student's advisor from
the list of computing-related courses approved by the BA CS
committee. These courses are offered by departments other than
Computer Science, and should either provide fundamental computing
depth and background or explore applications of computing to arts and
sciences fields. 

This is a list of the courses that are generally approved as
integration electives. This list is not meant to be exhaustive: if you
find a course that is not on the list that appears to satisfy the
goals of an integration elective, discuss with your advisor or the BA
Program Director if it should count as an integration elective for
you.

Some of these courses are not offered regularly, and some courses may
have prerequisites. The list of integration electives may change
slightly from year to year.  You can always check the current list of
integration electives on SIS.  The list below is according to SIS as
of September 2019.

\paragraph{American Studies}
\begin{itemlist}
\item AMST 3463: Language \& New Media
\end{itemlist}

\paragraph{Anthropology}
\begin{itemlist}
\item ANTH 3171: Culture of Cyberspace: Digital Fluency for an Internet-Enabled Society
\item ANTH 3490: Language and Thought
\end{itemlist}


\paragraph{Studio Art}
\begin{itemlist}
\item ARTS 2220: Introduction to New Media I
\item ARTS 2222: Introduction to New Media II
\item ARTS 3220: Intermediate New Media 
\item ARTS 3222: Intermediate New Media II
\item ARTS 4220: Advanced New Media I
\item ARTS 4222: Advanced New Media II
\end{itemlist}

\paragraph{Biology}
\begin{itemlist}
\item BIOL 4230: Bioinformatics and Functional Genomics
\end{itemlist}


\paragraph{Chemistry}
\begin{itemlist}
\item CHEM 3240: Coding in Matlab/Mathematica with Applications
\end{itemlist}


\paragraph{Drama}
\begin{itemlist}
\item DRAM 2110: Lighting Technology
\item DRAM 2210: Scenic Technology
\item DRAM 2240: Digital Design: Re-making and Re-imagining
\item DRAM 3825: Media Design Studio
\end{itemlist}


\paragraph{Economics}
\begin{itemlist}
\item ECON 3720: Econometric Methods 
\item ECON 4010 Game Theory
\item ECON 4020: Auction Theory and Practice
\item ECON 4720: Econometric Methods
\end{itemlist}

\paragraph{English Writing \& Rhetoric}
\begin{itemlist}
\item ENWR 2640: Composing Digital Stories and Essays
\item ENWR 3640: Writing with Sound
\end{itemlist}

\paragraph{Environmental Science}
\begin{itemlist}
\item EVSC 3020: GIS Methods
\item EVSC 4010: Introduction to Remote Sensing
\item EVSC 4070: Advanced GIS
\end{itemlist}


\paragraph{History}
\begin{itemlist}
\item HIST 2212: Maps in World History
\item HIUS 3162: Digitizing America
\end{itemlist}


\paragraph{Linguistics}
\begin{itemlist}
\item LING 3400: Structure of English
\item LNGS 3250: Intro to Linguistic Theory
\end{itemlist}

\paragraph{Mathematics}
\begin{itemlist}
\item MATH 3100: Intro Mathematical Probability
\item MATH 3120: Intro Mathematical Statistics
\item MATH 3315: Advanced Linear Algebra and Differential Equations
\item MATH 3350: Applied Linear Algebra
\item MATH 3351: Elementary Linear Algebra
\item MATH 4080: Operations Research
\item MATH 4300: Elementary Numerical Analysis
\end{itemlist}

\paragraph{Media Studies}
\begin{itemlist}
\item MDST 2010: Introduction to Digital Media
\item MDST 3050: History of Media
\item MDST 3102: Copyright, Commerce and Culture
\item MDST 3404: Democratic Politics in the New Media Environment
\item MDST 3500: Comparative Histories of the Internet
\item MDST 3701: New Media Culture
\item MDST 3702: Computers and Languages
\item MDST 3703: Digital Liberal Arts
\item MDST 3704: Games and Play
\item MDST 3750: Money, Media and Technology
\item MDST 3751: Values, Value, and Valuation
\item MDST 3755: Social Media and Society
\item MDST 4101: Privacy \& Surveillance
\item MDST 4700: Theory of New Media
\item MDST 4803: Computational Media
\end{itemlist}


\paragraph{Music}
\begin{itemlist}
\item MUSI 2350: Technosonics: Digital Music \& Sound Art Composition
\item MUSI 2390: Intro to Music \& Computers
\item MUSI 3390: Intro to Music \& Computers
\item MUSI 4535: Interactive Media
\item MUSI 4540: Computer Sound Generation
\item MUSI 4543: Sound Studio
\item MUSI 4545: Computer Applications in Music
\item MUSI 4610: Sound Synthesis
\item MUSI 4600: Performance with Computers
\end{itemlist}

\paragraph{Philosophy}
\begin{itemlist}
\item PHIL 1410: Forms of Reasoning
\item PHIL 1510: Ethics of Computing
\item PHIL 2330: Computers, Minds and Brains
\item PHIL 2340: The Computational Age
\item PHIL 2420: Introduction to Symbolic Logic
\end{itemlist}


\paragraph{Physics}
\begin{itemlist}
\item PHYS 2660: Fundamentals Scientific Computing
\end{itemlist}


\paragraph{Psychology}
\begin{itemlist}
\item PSYC 2150: Introduction to Cognition
\item PSYC 2200: Survey of the Neural Basis of Behavior
\item PSYC 2300: Introduction to Perception
\item PSYC 4110: Psycholinguistics
\item PSYC 4111: Language Development \& Disorders
\item PSYC 4125: Psychology of Language
\item PSYC 4150: Cognitive Processes
\item PSYC 4200: Neural Mechanisms of Behavior
\item PSYC 4300: Theories of Perception
\item PSYC 4400: Approaches to Quantitative Methods in Psychology
\item PSYC 4682: Mobile Technology in Mental Health Research
\end{itemlist}


\paragraph{Statistics}
\begin{itemlist}
\item STAT 1100: Chance: Intro to Statistics
\item STAT 1120: Intro to Statistics
\item STAT 2020: Statistics for Biologists
\item STAT 2120: Intro to Statistical Analysis
\item STAT 3010: Statist Computing \& Graphics
\item STAT 3080: From Data to Knowledge
\item STAT 3120: Intro to Mathematical Statistics
\item STAT 3220: Introduction to Regression Analysis
\item STAT 3240: Coding in Matlab/Mathematica with Applications
\item STAT 4220: Applied Analytics for Business
\item STAT 4260: Databases (only if CS 4750 has not been taken)
\item STAT 4630: Statistical Machine Learning
\end{itemlist}



\paragraph{Using other courses.}  If a student would like to use a
course not on the above list as an integration elective, they should
first contact their academic advisor.  Their advisor can work with the
student to come up with a good argument as to why the course should
qualify, and once the advisor approves it, send it to the BA CS
Director at \bacsdirectoremail.  Alternatively, if the advisor
prefers, s/he can just send the student to BA CS director to get
approval for a requirement exception.  This will require a SIS
exception to be entered for the student;
\ifthenelse{\boolean{ba-int-list-sis}}{see the full CS undergraduate handbook}{see section~\ref{sec:sisexceptions} (page~\pageref{sec:sisexceptions})}
for the manual SIS exception process.
